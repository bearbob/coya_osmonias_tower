\chapter*{Eine geglückte Rettung}

\block{awakenOutsideTower}{Ich bin erholt}

Du schläfst tief und fest.
Und du hast einen seltsamen Traum, in dem du auf einem Schimmel sitzt und über die Wolken reitest.
All die Menschen, Dörfer und Städte sind so unbeschreiblich winzig unter dir und es ist, als könntest du sie nur mit einem Finger zerquetschen wie Ameisen.
Plötzlich spürst du Finger, die zärtlich über deine Schultern wandern und die Szene verändert sich. Du sitzt nicht mehr auf dem Pferd, sondern auf einem kleinen Bett. Der Raum kommt dir seltsam vertraut vor, doch du kannst dich nicht erinnern, schon einmal hier gewesen zu sein.
Noch mysteriöser sind die Finger, die die streicheln.
Du drehst dich um und siehst den Körper einer Frau, nein, eines Mannes.
Vor deinen Augen verschwimmen und verändern sich die Konturen und Körperformen!
Als du den Blick nach oben wandern lässt, um in das Gesicht deine Begleitung zu sehen, trifft dich ein Schwall kaltes Wasser direkt ins Gesicht!

Als du langsam wieder zu Sinnen kommst, starrt dich eine grüne Fratze mit spitzen Zähnen breit grinsend an. %is it possible that the elf cannot rescue the goblin? Can the goblin die?
Augenblicklich schreckst du hoch!
Der kleine Goblin quiekt kurz erschrocken und stolpert dann nach hinten über einen kleinen Holzeimer, sodass er auf seinem Hintern landet. Verwirrt wischst du dir die kalten Wassertropfen aus dem Gesicht.

Wenn Ereignis \getEvent{iveSeenTheGoblin} eingetreten ist, gehe zu \goto{iKnowThisGoblin}.
Ansonsten gehe zu \goto{iDontKnowThisGoblin}.

\block{iDontKnowThisGoblin}{Ich lerne jemanden kennen}

Du versuchst die Müdigkeit abzuschütteln und zu verstehen, woher der Goblin kommt.
Scheinbar will er dir nichts Böses, doch die letzten Ereignisse haben dir gezeigt, dass man immer auf der Hut sein sollte.
Erst jetzt fällt dir auf, dass du nicht mehr an dem Baum nahe des schwarzen Turms lehnst, sondern dich auf einer kleinen Waldlichtung befindest.
Wildblumen wiegen sich im warmen Wind und ein süßlicher Geruch steigt dir in die Nase. Vögel zwitschern vergnügt und dem leisen Plätschern nach zu urteilen muss sich ein kleiner Bach in der Nähe befinden.

Der Goblin hast sich von seinem Schock erholt und sagt mit einem vorsichtigen Lächeln: ``Du... Hilfe! Du Alba gerettet!''. Kurz darauf hörst du hinter dir eine sanfte Stimme: ``Das ist wahr, du hast mich gerettet.''.

Gehe zu \goto{meetingAlba}.

\block{iKnowThisGoblin}{Ich erinnere mich an dich}

Natürlich erinnerst du dich an den kleinen Goblin.
Zuletzt hast du ihnen im Kerker des Turms gesehen.
Wie er von dort entkommen ist, ist dir allerdings ein Rätsel.
Erst jetzt fällt dir auf, dass du nicht mehr an dem Baum nahe des schwarzen Turms lehnst, sondern dich auf einer kleinen Waldlichtung befindest.
Wildblumen wiegen sich im warmen Wind und ein süßlicher Geruch steigt dir in die Nase. Vögel zwitschern vergnügt und dem leisen Plätschern nach zu urteilen muss sich ein kleiner Bach in der Nähe befinden.

Der Goblin hast sich von seinem Schock erholt und sagt mit einem vorsichtigen Lächeln: ``Du... Hilfe! Du Alba gerettet!''. Kurz darauf hörst du hinter dir eine sanfte Stimme: ``Das ist wahr, du hast mich gerettet.''.

Gehe zu \goto{meetingAlba}.

\block{meetingAlba}{Ich bringe Glück}

Hinter dir befindet sich der kleine Bach, vor dem die Elfe kniet. Ihr Haar ist nass und hängt schwer hinab. Einzelne Wassertropfen fallen herab und brechen die Sonnenstrahlen, wodurch ihre Gestalt in ein leichtes Funkeln gehüllt ist. Sie richtet sich auf und wringt mit beiden Händen ihre Haar aus, bevor sie es nach hinten schwingt. Einige Tropfen fallen hinab und perlen von ihrem blauen Seidengewand ab, das sich an ihren Körper schmiegt.

``Es freut mich, dass du wieder wach bist!'', sagt sie mit einem strahlenden Lächeln zu dir.
``Ich weiß nicht, wie du es geschafft hast, aber ich bin dir so unendlich dankbar, dass du mich gerettet hast! Ich dachte wirklich, dass ich in diesem Turm sterben würde und der arme Heinrich unten im Kerker verhungert. Aber das Glück war auf unserer Seite und hat dich geschickt!''

Während sie spricht, ist sie Schritt für Schritt auf dich zugekommen und kniet nun vor dir, sodass eure Augen auf einer Höhe sind.
Aus der Nähe sieht ihre Haut makelos aus und ihr Mund wirkt wie ein süßes Versprechen.
Nur ihre Augen wollen nicht ganz zum Rest passen, sie wirken kalt und hart.

Wenn du sie fragen willst, wer sie ist, gehe zu \goto{WhoAreYou}. Wenn du fragen willst, in was du da hineingeraten bist, gehe zu \goto{whatHasHappened}. Wenn du wissen willst, wo du bist, gehe zu \goto{whereAmI}. Wenn du dich verabschieden willst, gehe zu \goto{saygoodbyeFast}.

\block{whereAmI}{Ich kenne diesen Ort nicht}

Du lenkst deinen Blick von der Elfe und siehst dich um, während du fragst, wo du bist. Ihr seid offensichtlich nicht mehr im Wald um den Turm herum.

Die Elfe bestätigt diese Vermutung und sagt: ``Ich bin aufgewacht, nachdem du uns aus dem Turm geschliffen hast.
Es grenzt für mich noch immer an ein Wunder, dass du das geschafft hast und noch mehr, dass du mich nicht einfach zurückgelassen hast.
Ich kenne nicht viele, die ihr Leben für jemanden riskieren würden, ganz zu schweigen von jemandem, den sie nicht kennen.
Zum Glück waren meine Kräfte nicht vollkommen erschöpft und ich konnte den kleinen Heinrich befreien und uns weit weg von diesem grausigen Ort bringen.
Wir sind in der Nähe des kleinen Dörfchens Wrakenberg, zwei Tagesreisen östlich von Greifstadt.''

Wenn du sie fragen willst, wer sie ist, gehe zu \goto{WhoAreYou}. Wenn du fragen willst, in was du da hineingeraten bist, gehe zu \goto{whatHasHappened}. Wenn du dich verabschieden willst, gehe zu \goto{saygoodbye}.

\block{WhoAreYou}{Ich will wissen wer sie ist}

Als du fragst, wer sie ist, antwortet die Elfe: ``Eine gute, aber nicht ganz einfach zu beantwortende Frage. Ich lebe jetzt schon so lange, dass man mir viele Namen gegeben hat.
Ich weiß, das klingt wie ein Klischee, aber leider ist es die Wahrheit.
Ein Name, der mir immer gut gefallen hat, ist Alba.
In deiner Sprache bedeutet es 'Die Sanfte', das beschreibt mich ganz gut.''

Während sie spricht, fällt dir auf, dass der Ausschnitt ihres Kleids den Blick auf eine faustgroße, runde Narbe freigibt. Als Alba deinen Blick bemerkt, seufzt sie. ``Ja, das ist eine unschöne Geschichte. Es ist schon einige Monate her, dass mir diese Narbe zugefügt wurde. Eine Hexe hat mich überfallen und versucht zu töten. Wie du siehst, war sie nicht erfolgreich, aber ich spüre noch oft den Schmerz dieser Nacht, wenn ich daran denke. Leider sehe ich in meiner Zukunft, dass ich ihr eines Tages wieder begegnen werde.''

Als sie die leichte Verwirrung in deinen Augen beim letzten Satz bemerkt, fügt sie hinzu bevor du etwas sagen kannst: ``Oh, ja, ich bin eine Hexe, aber keine Sorge, ich setze meine Kräfte für das Gute in der Welt ein!''. Den letzten Satz sagt sie mit einem unbeschwerten Lachen, sodass du versucht bist ihr zu glauben. Vermutlich würdet ihr sonst auch nicht hier sitzen und einfach reden.

``Aber wir reden nur von mir. Wie du dir vorstellen kannst, habe ich auch einige Fragen, allen voran wem ich eigentlich verdanke, dass ich noch lebe.''. Du siehst keinen Grund, warum du ihr nicht antworten solltest und sagst ihr, wer du bist. Natürlich hast du im Laufe deiner Abenteurerkarriere gelernt, dass es immer klugt ist, einige Informationen für dich zu behalten. Alba kommt dir zwar sympathisch vor, doch ein wenig Misstrauen schadet nie. Die Elfe hört dir aufmerksam zu und fragt hin und wieder Kleinigkeiten nach, besonders deine Zeit in Greifenheim scheint sie zu interessieren.

Wenn du fragen willst, in was du da hineingeraten bist, gehe zu \goto{whatHasHappened}. Wenn du wissen willst, wo du bist, gehe zu \goto{whereAmI}.
Wenn du dich verabschieden willst, gehe zu \goto{saygoodbye}.

\block{whatHasHappened}{Ich will wissen was passiert ist}

Es wird Zeit, sich mit den wirklich wichtigen Fragen auseinanderzusetzen. Nämlich was genau eigentlich im Turm passiert ist und ob die Elfe weiß, warum man dich entführt hat. Und natürlich, warum sie dort oben ausbluten sollte.

Sie atmet schwer aus und antwortet bedrückt: ``Ich kann dir sagen, was ich weiß, doch es ist nicht viel. Ähnlich wie du wurden Heinrich und ich entführt, von zwei grausigen Kreaturen. Sie kamen mitten in der Nacht durch das Fenster unserer Herberge. Ich war wach, weil ich in den letzten Monaten unter Schlafstörungen leide. Zuerst dachte, es wären zwei gewöhnliche Einbrecher, zwei Männer, doch das war ein gewaltiger Irrtum. Aus ihren Händen wurden riesige Krallen und ihre Gesichter verzogen sich zu furchtbaren Fratzen mit mehreren Augen, die in der Dunkelheit leuchteten! Ich habe versucht mich zu wehren, doch sie haben mich überwältigt und bewusstlos geschlagen.

Als ich wieder zu Sinnen kam, lag ich gefesselt in einem Raum mit einem riesigen Edelstein auf einem Podest. Neben mir stand diese Missgeburt von einem Hexer, der sich als der große Loedriger vorstellte. Er sagte, er habe in dem Turm endlich das Geheimnis gefunden und mein Blut sei der Schlüssel. Er stach mir eine Nadel in die Brust und ließ mich von seinen beiden Schergen an der Decke aufhängen... danach erinnere ich mich nur noch daran, wie ich neben dir in Freiheit aufgewacht bin.''

Wenn du \getItem{itemFreyaBook} hast und ihr zeigen möchtest, gehe zu \goto{givingAlbaTheBook}.
Wenn du sie fragen willst, wer sie ist, gehe zu \goto{WhoAreYou}. Wenn du wissen willst, wo du bist, gehe zu \goto{whereAmI}.
Wenn du dich verabschieden willst, gehe zu \goto{saygoodbye}.

\block{givingAlbaTheBook}{Ich habe das hier gefunden}

Du zeigst der Elfe das Buch, dass du in der kleinen Bibliothek des Turms gefunden hast. Zwar kannst du mit der Schrift nicht viel anfangen, doch das scheint bei ihr nicht der Fall zu sein. Mit ausgestrecktem Zeigefinger geht sie Rune für Rune ab und formt mit ihren Lippen stumme Worte.
Als du sie fragst, ob sie etwas erkennen kann, schreckt sie hoch und sagt: ``Entschuldige, ich... das ist unglaublich... interessant! Es scheint eine Art Tagebuch zu sein, aber hier sind auch Texte, deren Bedeutung mir noch nicht ganz klar ist.''

Sie blättert weiter durch das Buch und sieht sich auch die Zeichnungen genau an. Nach einigen Minuten sagt sie: ``Es ist schwer, das innerhalb so kurzer Zeit zu sagen, es würde vermutlich Wochen dauern, bis ich alles gelesen habe und noch länger, um es zu verstehen... aber es scheint um eine Frau zu gehen, Freya.''.
Sie deutet mit dem Finger auf die kleinen Runen unter einer Zeichnung und fügt hinzu ``Scheinbar war der Schreiber, Osmonias, ganz besessen von ihr. Er hat für sie einen Tempel errichtet, einen Ort der... vielleicht soll das auch Gruft bedeuten, schwer zu sagen. Auf jeden Fall ist das unglaublich! Siehst du diese Zeichnung hier? Und diese?''

Während sie spricht, zeigt sie immer wieder auf Seiten des Buches, die mit Kritzeleien übersät sind. Zumindest dachtest du das, doch die Elfe erklärt dir, dass es sich um Pläne handelt.
``Ähnlich wie Magier ihre Runenkreise konstruieren, haben die Bewohner der alten Welt die Maschinengötter geschaffen.
Mit Plänen, mit Anleitungen! Dies scheint ein Teil einer solchen Anleitung zu sein. Vielleicht war Osmonios ein Maschinenpriester? Oh, eine Frage jagt die nächste...''

Plötzlich schießt ihr Blick nach oben zu dir. Du kannst ein Funkeln in ihren Augen erkennen, als sie vorsichtig fragt: ``Ich weiß, es ist viel verlangt, aber... würdest du mir dieses Buch überlassen?''

Wenn du ihr das Buch schenken möchtest, gehe zu \goto{giftingTheBook}. Wenn du das Buch behalten möchtest, gehe zu \goto{keepingTheBook}.

\block{giftingTheBook}{Ich kann damit ohnehin nichts anfangen}

Natürlich gibst du der Elfe das Buch, wenn sie es haben will. Es ist offensichtlich, dass sie mehr Nutzen daraus ziehen kann als du. Vor Freude fällt sie dir direkt um den Hals und bedankt sich überschwänglich. Sie hat zwar nichts, dass sie dir im Tausch geben könnte, doch sie hofft, dass sich das Schicksal irgendwie bei dir erkenntlich zeigen wird.

Wenn du sie fragen willst, wer sie ist, gehe zu \goto{WhoAreYou}. Wenn du wissen willst, wo du bist, gehe zu \goto{whereAmI}.
Wenn du dich verabschieden willst, gehe zu \goto{saygoodbye}.

\block{keepingTheBook}{Ich könnte das noch brauchen}

Es ist offensichtlich, dass sie mehr Nutzen aus dem Buch ziehen kann als du.
Aber dein Gefühl sagt dir, dass du das Buch noch behalten solltest und dein Gefühl liegt oft richtig.
Die Elfe wirkt zwar etwas enttäuscht, sagt aber, dass sie dich verstehen kann.
``Vielleicht ist es vom Schicksal so gewollt und gegen das Schicksal sollte man sich nicht stellen. Ich bin schon dankbar, dass du mein Leben gerettet hast, wie könnte ich noch mehr von dir fordern, wo ich doch nichts habe, um mich zu bedanken.''

Wenn du sie fragen willst, wer sie ist, gehe zu \goto{WhoAreYou}. Wenn du wissen willst, wo du bist, gehe zu \goto{whereAmI}.
Wenn du dich verabschieden willst, gehe zu \goto{saygoodbye}.

\block{saygoodbyeFast}{Ich muss gehen}

Du warst noch nie für Zärtlichkeiten und aufgesetzte Freundlichkeit zu haben. Vorsichtig rutschst du ein paar Zentimeter zurück und stehst behutsam auf. Du hast keine Schmerzen, im Gegenteil, du fühlst dich körperlich fit. Der Blick der Elfe folgt deinen Bewegungen und sie fragt: ``Alles in Ordnung bei dir?''. Du nickst ihr zu und erklärst dann, dass du froh bist, dass du helfen konntest. Du willst auch nicht unhöflich sein, aber es ist an der Zeit für dich nach Greifenheim aufzubrechen.

Die Elfe wirkt überrascht, nickt dann aber nach einem Moment und antwortet: ``Ich verstehe. Ich hoffe du hälst mich nicht für undankbar, mir ist bewusst, dass ich dir mein Leben verdanke. Doch ich habe nichts, das ich dir geben könnte, um das zu zeigen. Ich hoffe, dass das Schicksal einen Weg findet, um dich zu belohnen!''

Sie steht auf und deutet tiefer in den Wald hinein. ``Wenn du in diese Richtung gehst, solltest du in ungefähr zwei Tagen in Greifenheim ankommen. Ich hoffe, dass du sicher und wohlbehütet an deinem Ziel ankommst.''

Du nickst der Elfe zu und verabschiedest dich von ihr und dem Goblin. Dann machst du dich auf den Weg in die gezeigte Richtung. Gehe zu \goto{theEndAlone}.

\block{saygoodbye}{Unsere Wege trennen sich}

Du findest die Elfe zwar sehr nett, aber während ihr euch unterhaltet vergeht mehr Zeit, als dir bewusst ist. Der Himmel wird in ein warmes orange getaucht, als dir bewusst ist, dass die Dämmerung eingesetzt hat und ihr irgendwo in der Wildnis seid.

Als du deine Gedanken äußerst, steht deine Gesprächspartnerin auf und deutet tiefer in den Wald hinein. ``Wenn du in diese Richtung gehst, solltest du in ungefähr zwei Tagen in Greifenheim ankommen. Allerdings, wie du so treffend festgestellt hast, wird es schon dunkel. Vielleicht wäre es klüger, wenn du uns nach Wrakenberg begleitest und heute Nacht dort bleibst. Morgen früh sollte es sicherer sein.''

Wenn du mit den beiden nach Wrakenberg gehen willst, gehe zu \goto{restingInWrakenberg}. Wenn du dich trotz der Dunkelheit auf den Weg nach Greifenheim machen willst, gehe zu \goto{backToGreifenheim}.

\block{backToGreifenheim}{Ich fürchte die Dunkelheit nicht}

Natürlich ist es nicht ungefährlich nachts durch den Wald zu marschieren. Aber du willst nicht noch mehr Zeit verschwenden als nötig, sicherlich wirst du unterwegs eine Rastmöglichkeit finden.

Du nickst der Elfe zu und verabschiedest dich von ihr und dem Goblin. Dann machst du dich auf den Weg in die gezeigte Richtung. Du bist vermutlich knapp drei Stunden unterwegs, als du zwischen den Bäumen das tanzende Licht eines Lagerfeuers sehen kannst. Leise dringt das klimpern eines Saiteninstruments an deine Ohren. Vorsichtig näherst du dich und erkennst zwei große Pferdewagen und eine kleine Gruppe, die vor dem Feuer tanzt. Du erkennst drei Männer und zwei Frauen, die alle lange schwarze Haare und ausfallende Kleidung aus Stofftüchern tragen. Überall hängen kleine Talismane und Münzen aus den Haaren und die Stimmung scheint ausgelassen, was wohl der Grund dafür ist, dass sie dich noch nicht bemerkt haben.

Wenn du dich der Gruppe zu erkennen geben willst, gehe zu \goto{helloGypsies}.
Wenn du unerkannt weiterziehen willst, gehe zu \goto{forwardsToGreifenheim}

\block{helloGypsies}{Ich tanze gern}

Du trittst aus dem Schatten der Bäume hervor und gibst dich der Gruppe zu erkennen. Es dauert keine fünf Minuten, bis du lachend zwischen den anderen sitzt, mit einem Bier in einen Hand und einem Brot in der anderen.

Die kleine Gruppe besteht aus Nomaden, die, zu deinem Glück, gerade auf dem Weg nach Greifenheim sind. Die Feier geht über Stunden, bis die Sonne das Dunkel der Nacht durchbricht. Müde, aber zufrieden, sinkst du in einem der weichen Betten in den Schlaf. Als du wieder wach wirst, befindet ihr euch bereits kurz vor Greifenheim. Du dankst deinen Gastgebern überschwänglich und wirst genau überschwänglich von ihnen verabschiedet. Dann wird es Zeit, dass sich eure Wege wieder trennen.

Schnellen Schrittes gehst du zum ``Zum Nimmerleer''. Du hast dir Ruhe und Erholung verdient. Und sicherlich wirst du mit der Geschichte deines Abenteuers das ein oder andere Freigetränk in den Gasthäusern erwerben können. Ganz zu schweigen davon, was du vielleicht noch bei den Händlern verdienen könntest, wenn du deine Beute verkaufst. Doch das sind Aufgaben für einen anderen Tag. Du gehst nach Hause.

\textbf{Ende.}

\block{forwardsToGreifenheim}{Mir steht nicht der Sinn nach lauter Musik}

Du wendest dich von dem bunten Treiben ab und setzt deinen Weg nach Greifenheim fort. Vermutlich würde dich die Gruppe willkommen heißen, aber man weiß nie.

Du kommst ein gutes Stück voran, bevor dich die Müdigkeit einholt. Du findest einen Baum, der sich gut eignet um unerkannt eine Rast einzulegen und bist wenige Stunden später hungrig, aber ausgeruht.
Du kletterst wieder zum Waldboden und streckst dich. Gehe zu \goto{theEndAlone}.

\block{restingInWrakenberg}{Ich sollte rasten}

Natürlich ist es nicht ungefährlich nachts durch den Wald zu marschieren.
Und das ist dir auch klar, du hast bei diesem Abenteuer oft genug dein Glück herausgefordert.
Nun solltest du nicht übermütig werden, das kann schlimm ausgehen.
Nein, du gehst lieber sicher und gehst mit der Elfe und dem Goblin zum kleinen Dorf Wrakenberg.

Nach nichteinmal einer Stunde seid ihr am Wall des Dorfes angekommen. Die Elfe und der Goblin laufen einige Meter vor dir und nach wenigen Minuten seid ihr im Gasthaus des Dorfes. Es ist sehr ruhig, ihr seid die einzigen Gäste wie es scheint.
Dir ist nicht klar, wie sie es anstellt, aber nach einem kurzen Gespräch mit dem Wirt willigt dieser ein euch heute Nacht umsonst hier schlafen zu lassen. Warum solltest du ein kostenloses Bett ausschlagen?

Am nächsten Morgen wachst du erholt auf. Als du dein Zimmer verlässt, warten die Elfe und der Goblin bereits auf dich. Mit einem Lächeln sagt sie: ``Hier trennen sich leider unsere Wege. Wenn du das möchtest, ich habe einen Händler gefunden, der nach Greifenheim aufbricht und dich mitnehmen würde. Unser Weg führt uns leider in eine andere Richtung, doch ich werde nie vergessen, was du für uns getan hast!''

Sie umarmt dich ein letztes Mal und der Goblin winkt dir etwas schüchtern zu, bevor beide aufbrechen. Die versprochene Mitfahrgelegenheit bricht kurze Zeit später auf und schon zwei Tage danach bist du endlich zurück in Greifenheim.

Schnellen Schrittes gehst du zum ``Zum Nimmerleer''. Du hast dir Ruhe und Erholung verdient. Und sicherlich wirst du mit der Geschichte deines Abenteuers das ein oder andere Freigetränk in den Gasthäusern erwerben können. Ganz zu schweigen davon, was du vielleicht noch bei den Händlern verdienen könntest, wenn du deine Beute verkaufst. Doch das sind Aufgaben für einen anderen Tag. Du gehst nach Hause.

\textbf{Ende.}
