\chapter{Die Reise beginnt...}

\block{startpage}{Unsanftes Erwachen}
Leise rauscht der warme Herbstwind durch das vielfarbige Blätterdach des Waldes. Viele Bäume haben bereits den Großteil ihrer Blätter abgeworfen und färben so den schmalen Weg in zarte Rot- und Goldtöne.\\
Das ganze Bild strahl eine tiefe Ruhe aus, die du jedoch nur schwerlich genießen kannst. Du liegst auf einem Pferdekarren, gelenkt von zwei muskelbepackten Männern. Ihre breiten Rücken sind dir zugewandt, sodass du ihre Gesichter nicht erkennen kannst. Beide haben schulterlange blonde Haare, die der rechte zu einem kleinen Knoten geformt hat. Scheinbar haben sie noch nicht bemerkt, dass du wieder zu dir gekommen bist, denn sie unterhalten sich angeregt. Leider sind ihre Worte durch die Geräusche des Karrens schwer zu verstehen. Dir ist noch nicht ganz klar, wie du in diese Situation gekommen bist, aber die dröhnenden Kopfschmerzen lassen nichts gutes vermuten. Schnell stellst du fest, dass deine Hände hinter dem Rücken zusammengebunden sind und deine Knöchel ebenfalls gefesselt sind. Dein Kiefer schmerzt ein wenig von dem dicken Stoffknebel.
\\Wenn du dich bemerkbar machen willst, gehe zu [\ref{getAttention}].
\\Wenn du dich weiter umsehen willst, gehe zu [\ref{lookAtCart}].
\\Wenn du abwarten möchtest, gehe zu [\ref{waitInCart}].
\\Wenn du versuchen willst zu verstehen, worüber die beiden reden, gehe zu [\ref{listenInCart}].

\block{getAttention}{Ich hätte gern Aufmerksamkeit}

Es wird Zeit herauszufinden mit wem du hier eigentlich unterwegs bist. Leider werden deine Worte durch den Knebel zu stark gedämpft, um beim Knarzen und Ächzen des Wagens gehört zu werden.
\\Wenn du gegen den Kutschbock treten möchtes, gehe zu [\ref{kickCart}].
\\Wenn du dich umsehen möchtest, gehe zu [\ref{lookAtCart}].
\\Wenn du abwarten möchtest, gehe zu [\ref{waitInCart}].

\block{kickCart}{Ich brauche Aufmerksamkeit}

Da dir im Moment nichts besseres einfällt, drehst du dich mit den Beinen zum Kutschbock und trittst beherzt gegen das alte Holz. Wie erwartet sorgt das dumpfe Geräusch dafür, dass sich deine Entführer umdrehen. Die beiden könnten Zwillinge sein, so ähnlich sehen sie sich. Beide haben grobschlächtige Gesichter und breiten Nasen. Kleine Bartstoppeln umrahmen den Mund, der bei beiden ein leichtes Grinsen formt. Trotzdem sind sie leicht zu unterscheiden, denn beim rechten Halunken zieht sich eine große Narbe von der linken Wange über die Nase zur rechten Augenbraue.\\
``Sieh dir das an, Torlof, unsere Prinzessin ist aufgewacht!'', sagt er laut mit tiefer Stimme. Torlof antwortet nicht, sondern brummt nur belustigt vor sich hin. Sein Partner fügt hinzu: ``Na, dann tu mal den Rest deiner Reise genießen, ein paar Stunden wird es noch dauern bis wir da sind!''\\
Dein fragender Blick wird keiner Antwort gewürdigt. Die beiden Männer lachen kurz und drehen sich dann wieder nach vorn.
\\Wenn du dich umsehen möchtest, gehe zu [\ref{lookAtCart2}].
\\Wenn du warten möchtest, gehe zu [\ref{waitInCart}].

\block{listenInCart}{Ich belausche die beiden}

Deine Entführer reden weder besonders laut, noch besonders deutlich. Du versuchst vorsichtig dich zu drehen und so etwas näher im vorderen Bereich des Karrens zu liegen. Vielleicht kannst du so etwas mehr verstehen.
\\Lege eine Aufmerksamkeitsprobe mit DC 18 ab. Wenn du erfolgreich bist, gehe zu [\ref{eavesdroppingSuccess}].
Wenn du scheiterst, gehe zu [\ref{eavesdroppingFailed}].

\block{eavesdroppingFailed}{Ich kann nichts verstehen}

Du bemühst dich ein paar sinnvolle Sätze aus den wenigen Worten zu bilden, die du verstehen kannst. Doch abgesehen von ``Aufstand'' und ``kleiner Goblin'' kannst du nichts zuordnen. Und diese beiden Brocken sind wenig hilfreich.
\\Wenn du dich bemerkbar machen willst, gehe zu [\ref{getAttention}].
\\Wenn du dich weiter umsehen willst, gehe zu [\ref{lookAtCart}].
\\Wenn du abwarten möchtest, gehe zu [\ref{waitInCart}].

\block{eavesdroppingSuccess}{Ich kann ein wenig verstehen}

Du schaffst es dich so zu drehen, dass du die Stimmen etwas klarer verstehst. Hin und wieder fehlen dir einige Worte, doch du glaubst trotzdem zu erkennen worum es geht. Die beiden Männer unterhalten sich über einen Aufstand einiger Dorfbewohner, nachdem ihr Meister mehr Gold verlangte. Scheinbar ging die Geschichte für die Dorfbewohner nicht gut aus. Das kurze Lachen der beiden Schurken lässt kein fröhliches Ende vermuten. Außerdem sprechen sie kurz über eine Frau, eine Elfe, die scheinbar eine sehr reizvolle Figur hat. Und über einen kleinen Goblin, allerdings bist du dir nicht sicher wie die letzten beiden Teile zusammen passen - die beiden scheinen Gäste des Mannes zu sein, der deine Entführung beauftragt hat?\\
Danach schweigen die beiden für eine Weile und du überlegst, was du tun solltest.
\\Wenn du dich bemerkbar machen willst, gehe zu [\ref{getAttention}].
\\Wenn du dich weiter umsehen willst, gehe zu [\ref{lookAtCart}].
\\Wenn du abwarten möchtest, gehe zu [\ref{waitInCart}].

\block{waitInCart}{Ich akzeptiere mein Schicksal}

Du bist dir nicht sicher ob es nur ein paar Stunden sind, doch es fühlt sich wie eine Ewigkeit an. Du liegst mit dem Rücken im Wagen und beobachtest den Himmel, an dem hin und wieder dicke weiße Wolken vorbeiziehen. Scheinbar seid ihr inzwischen in einem Wald, denn um euch herum sind immer mehr Bäume zu sehen und der Weg ist noch holpriger geworden als zuvor. Man muss kein Genie sein um zu wissen, dass es sich nicht um eine vielbefahrene Handelsstraße handelt.\\
Wirf einen W20. Ist das Ergebnis mindestens 16, gehe zu [\ref{banditEncounter}].\\
Ansonsten gehe zu [\ref{arrivalAtTower}].

\block{lookAtCart}{Ich sehe mich um}

Du lässt deinen Blick schweifen und versuchst deine aktuelle Lage besser zu erfassen. Der Karren ist aus Holz und hat schon bessere Tage gesehen. Am hinteren Ende befindet sich eine Ladeklappe, die von Metallriegeln auf beiden Seiten gehalten wird. Neben dir befinden sich zwei gefüllte braune Leinensäcke.
\\Wenn du zur Ladeklappe kriechen willst, gehe zu [\ref{loadramp}].
\\Wenn du warten möchtest, gehe zu [\ref{waitInCart}].
\\Wenn du versuchen willst herauszufinden was sich in den Säcken befindet, gehe zu [\ref{inspectCargo}].

\block{loadramp}{Ich krieche zur Ladeklappe}

Es ist ein kleines Kunststück und dauert einige Minuten, aber du schaffst es dich zu drehen und zur Klappe zu kriechen. Die Bretter werden von Nägeln zusammengehalten, denen Wind und Wetter stark zugesetzt haben. Trotzdem glaubst du, dass es kein Kinderspiel wäre die Klappe mit Gewalt zu öffnen - was vor allem an deiner eingeschränkten Bewegungsmöglichkeit liegt.
\\Wenn du versuchen willst die Klappe aufzutreten, gehe zu [\ref{forceLoadramp}].
\\Wenn du versuchen willst die Riegel zu öffnen, gehe zu [\ref{unlockLoadramp}].
\\Wenn du versuchen willst herauszufinden was sich in den Säcken befindet, gehe zu [\ref{inspectCargoFromloadramp}].
\\Wenn du warten möchtest, gehe zu [\ref{waitInCart}].

\block{forceLoadramp}{Ich trete gegen die Ladeklappe}

Es braucht wieder einige Minuten der Vorbereitung, bis du dich auf den Rücken gedreht hast und mit angewinkelten Knien vor der Holzklappe liegst.\\
Lege eine Stärkeprobe mit DC 16 ab. Wenn du erfolgreich bist, gehe zu [\ref{forceLoadrampsuccess}].
Wenn du scheiterst, gehe zu [\ref{forceLoadrampfailed}].

\block{forceLoadrampsuccess}{Ich bin stärker als Holz}

Du konzentrierst dich und sammelst deine Kräfte für einen gezielten Tritt gegen die Seite der Holzklappe. Das Glück ist endlich auf deiner Seite, das Material gibt nach und bricht am Metallriegel. Ein nervöser Blick auf deine Entführer zeigt dir, dass sie überhaupt nicht bemerken, was du da treibst. Schnellst sammelst du dich für einen zweiten Tritt auf der anderen Seite und die Klappe fällt nach unten ab. Einige Herzschläge später rollst du seitwärts vom Wagen und fällst auf den steinigen Boden. Dabei verlierst du 2 Lebenspunkte.\\
Während du versuchst nicht vor Schmerz zu stöhnen, rollt der Wagen weiter den Weg entlang. Die beiden Halunken ahnen nicht einmal, dass du nicht wie ein Lamm auf die Schlachtung wartest. Schnell kriechst du vom Weg zu einem der Bäume, während sich der Wagen Meter um Meter von dir entfernt.

Es dauert einige Stunden, doch irgendwann kannst du deine Fesseln lösen. Du machst dich auf den Rückweg, wobei du darauf achtest dem Weg fern zu bleiben.
Du wirst zwar nie erfahren, warum du entführt wurdest, aber manche Geheimnisse können auch ungelöst bleiben.

\textbf{Ende.}

\block{inspectCargoFromloadramp}{Ich wende mich den Säcke zu}

Es ist ein kleines Kunststück und dauert einige Minuten, aber du schaffst es dich wieder zurück zu drehen und zu den Säcken zu kriechen. Es ist nicht gerade einfach herauszufinden was sich in den Säcken befindet, wenn die Hände hinter dem Rücken gefesselt sind. Trotzdem schaffst du es dich so zu positionieren, dass einer der Säcke hinter dir liegt und du ihn berühren kannst. Du spürst einige runde Stellen, die mit ausreichend Druck nachgeben. Wenn dich nicht alles täuscht, sind das Kartoffeln in den Säcken.
\\Wenn du versuchen willst dir eine der Kartoffeln zu nehmen, gehe zu [\ref{stealPotatoAfterRamp}].
\\Wenn du zurück zur Ladeklappe kriechen willst, gehe zu [\ref{returnLoadramp}].
\\Wenn du warten möchtest, gehe zu [\ref{waitInCart}].

\block{stealPotatoAfterRamp}{Ich benutze meine Finger}

Natürlich sind die Säcke zugeknotet. Normalerweise wäre das vermutlich kein Hindernis, doch leider ist diese Situation nicht ganz normal. Mit dem Rücken zum Knoten versuchst du zu ertasten wo du wie ziehen musst.
\\Lege eine Fingerfertigkeitsprobe mit DC 15 ab. Wenn du Erfolg hast, gehe zu [\ref{gotPotatoRamp}].
\\Wenn es dir nicht gelingt, gehe zu [\ref{failedPotatoRamp}].

\block{gotPotatoRamp}{Mein Triumph-Moment}

Ein Gefühl des Stolzes überkommt dich, als du spürst wie sich das Seil lockert und du den Knoten lösen kannst. Du ziehst die Öffnung zu Boden und ein paar Kartoffeln rollen über den Boden. Dir ist bewusst, dass du nur an eine deiner Taschen herankommst und dort nicht mehr als eine Kartoffel hineinpasst. Du suchst dir ein Exemplar heraus, dass sich gut anfühlt und packst es ein.
\\Du erhälst eine Kartoffel (1)!
\\Wenn du zurück zur Ladeklappe kriechen willst, gehe zu [\ref{returnLoadramp}].
\\Wenn du warten möchtest, gehe zu [\ref{waitInCart}].

\block{failedPotatoRamp}{Vergebene Mühe}

Ein Gefühl der Hilflosigkeit überkommt dich, als du spürst, dass es keinen Sinn hat sich weiter an dem Seil zu versuchen. Solange deine Hände gefesselt sind und du den Knoten nicht sehen kannst, wirst du ihn nicht öffnen können.
\\Wenn du zurück zur Ladeklappe kriechen willst, gehe zu [\ref{returnLoadramp}].
\\Wenn du warten möchtest, gehe zu [\ref{waitInCart}].

\block{returnLoadramp}{Ich krieche zurück zur Ladeklappe}

Du lässt die Kartoffeln Kartoffeln sein und wendest dich wieder den Holzbrettern zu.
\\Wenn du versuchen willst die Klappe aufzutreten, gehe zu [\ref{forceLoadramp}].
\\Wenn du versuchen willst die Riegel zu öffnen, gehe zu [\ref{unlockLoadramp}].
\\Wenn du warten möchtest, gehe zu [\ref{waitInCart}].

\block{forceLoadrampfailed}{Das Holz hat gewonnen}

Du konzentrierst dich und sammelst deine Kräfte für einen gezielten Tritt gegen die Seite der Holzklappe. Zu deinem Pech rührt sich überhaupt nichts, woran auch ein zweiter und ein dritter Tritt nichts ändern. Als du zum vierten Tritt ansetzt, bemerkst du plötzlich einen Schatten über dir und hörst eine tiefe Stimme ``Na, sowas sehen wir hier aber gar nicht gern!'' brummen, bevor du einen harten Schlag auf den Kopf bekommst und das Bewusstsein verlierst.\\
Du verlierst 5 Lebenspunkte.
\\Gehe zu [\ref{wakeUpInCell}].

\block{lookAtCart2}{Ich sehe mich um}
% copy of lookAtCart after the hero has seen the kidnappers
Du lässt deinen Blick schweifen und versuchst deine aktuelle Lage besser zu erfassen. Der Karren ist aus Holz und hat schon bessere Tage gesehen. Am hinteren Ende befindet sich eine Ladeklappe, die von Metallriegeln auf beiden Seiten gehalten wird. Neben dir befinden sich zwei gefüllte braune Leinensäcke. Torlof und sein Kumpane scheinen wieder in ihr Gespräch vertieft zu sein.
\\Wenn du zur Ladeklappe kriechen willst, gehe zu [\ref{loadramp}].
\\Wenn du warten möchtest, gehe zu [\ref{waitInCart}].
\\Wenn du versuchen willst herauszufinden was sich in den Säcken befindet, gehe zu [\ref{inspectCargo}].

\block{inspectCargo}{Ich untersuche die Säcke}

Es ist nicht gerade einfach herauszufinden was sich in den Säcken befindet, wenn die Hände hinter dem Rücken gefesselt sind. Trotzdem schaffst du es dich so zu drehen, dass einer der Säcke hinter dir liegt und du ihn berühren kannst. Du spürst einige runde Stellen, die mit ausreichend Druck nachgeben. Wenn dich nicht alles täuscht, sind das Kartoffeln in den Säcken.
\\Wenn du versuchen willst dir eine der Kartoffeln zu nehmen, gehe zu [\ref{stealPotato}].
\\Wenn du zur Ladeklappe kriechen willst, gehe zu [\ref{loadramp}].
\\Wenn du warten möchtest, gehe zu [\ref{waitInCart}].

\block{stealPotato}{Ich erprobe mein Fingerspitzengefühl}

Natürlich sind die Säcke zugeknotet. Normalerweise wäre das vermutlich kein Hindernis, doch leider ist diese Situation nicht ganz normal. Mit dem Rücken zum Knoten versuchst du zu ertasten wo du wie ziehen musst.
\\Lege eine Fingerfertigkeitsprobe mit DC 15 ab. Wenn du Erfolg hast, gehe zu [\ref{gotPotato}].
\\Wenn es dir nicht gelingt, gehe zu [\ref{failedPotato}].

\block{gotPotato}{Der Knoten löst sich}

Ein Gefühl des Stolzes überkommt dich, als du spürst wie sich das Seil lockert und du den Knoten lösen kannst. Du ziehst die Öffnung zu Boden und ein paar Kartoffeln rollen über den Boden. Dir ist bewusst, dass du nur an eine deiner Taschen herankommst und dort nicht mehr als eine Kartoffel hineinpasst. Du suchst dir ein Exemplar heraus, dass sich gut anfühlt und packst es ein.
\\Du erhälst eine Kartoffel (1)!
\\Wenn du zur Ladeklappe kriechen willst, gehe zu [\ref{loadramp}].
\\Wenn du warten möchtest, gehe zu [\ref{waitInCart}].

\block{failedPotato}{Der Knoten sitzt zu fest}

Ein Gefühl der Hilflosigkeit überkommt dich, als du spürst, dass es keinen Sinn hat sich weiter an dem Seil zu versuchen. Solange deine Hände gefesselt sind und du den Knoten nicht sehen kannst, wirst du ihn nicht öffnen können.
\\Wenn du zur Ladeklappe kriechen willst, gehe zu [\ref{loadramp}].
\\Wenn du warten möchtest, gehe zu [\ref{waitInCart}].
