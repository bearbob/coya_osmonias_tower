\chapter{Die Reise beginnt...}

\block{startpage}{Unsanftes Erwachen}
Leise rauscht der warme Herbstwind durch das vielfarbige Blätterdach des Waldes. Viele Bäume haben bereits den Großteil ihrer Blätter abgeworfen und färben so den schmalen Weg in zarte Rot- und Goldtöne.\\
Das ganze Bild strahl eine tiefe Ruhe aus, die du jedoch nur schwerlich genießen kannst. Du liegst auf einem Pferdekarren, gelenkt von zwei muskelbepackten Männern. Ihre breiten Rücken sind dir zugewandt, sodass du ihre Gesichter nicht erkennen kannst. Beide haben schulterlange blonde Haare, die der rechte zu einem kleinen Knoten geformt hat. Scheinbar haben sie noch nicht bemerkt, dass du wieder zu dir gekommen bist, denn sie unterhalten sich angeregt. Leider sind ihre Worte durch die Geräusche des Karrens schwer zu verstehen. Dir ist noch nicht ganz klar, wie du in diese Situation gekommen bist, aber die dröhnenden Kopfschmerzen lassen nichts gutes vermuten. Schnell stellst du fest, dass deine Hände hinter dem Rücken zusammengebunden sind und deine Knöchel ebenfalls gefesselt sind. Dein Kiefer schmerzt ein wenig von dem dicken Stoffknebel.
\\Wenn du dich bemerkbar machen willst, gehe zu [\ref{getAttention}].
\\Wenn du dich weiter umsehen willst, gehe zu [\ref{lookAtCart}].
\\Wenn du abwarten möchtest, gehe zu [\ref{waitInCart}].
\\Wenn du versuchen willst zu verstehen, worüber die beiden reden, gehe zu [\ref{listenInCart}].

\block{getAttention}{Ich hätte gern Aufmerksamkeit}

Es wird Zeit herauszufinden mit wem du hier eigentlich unterwegs bist. Leider werden deine Worte durch den Knebel zu stark gedämpft, um beim Knarzen und Ächzen des Wagens gehört zu werden.
\\Wenn du gegen den Kutschbock treten möchtes, gehe zu [\ref{kickCart}].
\\Wenn du dich umsehen möchtest, gehe zu [\ref{lookAtCart}].
\\Wenn du abwarten möchtest, gehe zu [\ref{waitInCart}].

\block{kickCart}{Ich brauche Aufmerksamkeit}

Da dir im Moment nichts besseres einfällt, drehst du dich mit den Beinen zum Kutschbock und trittst beherzt gegen das alte Holz. Wie erwartet sorgt das dumpfe Geräusch dafür, dass sich deine Entführer umdrehen. Die beiden könnten Zwillinge sein, so ähnlich sehen sie sich. Beide haben grobschlächtige Gesichter und breiten Nasen. Kleine Bartstoppeln umrahmen den Mund, der bei beiden ein leichtes Grinsen formt. Trotzdem sind sie leicht zu unterscheiden, denn beim rechten Halunken zieht sich eine große Narbe von der linken Wange über die Nase zur rechten Augenbraue.\\
``Sieh dir das an, Torlof, unsere Prinzessin ist aufgewacht!'', sagt er laut mit tiefer Stimme. Torlof antwortet nicht, sondern brummt nur belustigt vor sich hin. Sein Partner fügt hinzu: ``Na, dann tu mal den Rest deiner Reise genießen, ein paar Stunden wird es noch dauern bis wir da sind!''\\
Dein fragender Blick wird keiner Antwort gewürdigt. Die beiden Männer lachen kurz und drehen sich dann wieder nach vorn.
\\Wenn du dich umsehen möchtest, gehe zu [\ref{lookAtCart2}].
\\Wenn du warten möchtest, gehe zu [\ref{waitInCart}].

\block{listenInCart}{Ich belausche die beiden}

Deine Entführer reden weder besonders laut, noch besonders deutlich. Du versuchst vorsichtig dich zu drehen und so etwas näher im vorderen Bereich des Karrens zu liegen. Vielleicht kannst du so etwas mehr verstehen.
\\Lege eine Aufmerksamkeitsprobe mit DC 18 ab. Wenn du erfolgreich bist, gehe zu [\ref{eavesdroppingSuccess}].
Wenn du scheiterst, gehe zu [\ref{eavesdroppingFailed}].

\block{eavesdroppingFailed}{Ich kann nichts verstehen}

Du bemühst dich ein paar sinnvolle Sätze aus den wenigen Worten zu bilden, die du verstehen kannst. Doch abgesehen von ``Aufstand'' und ``kleiner Goblin'' kannst du nichts zuordnen. Und diese beiden Brocken sind wenig hilfreich.
\\Wenn du dich bemerkbar machen willst, gehe zu [\ref{getAttention}].
\\Wenn du dich weiter umsehen willst, gehe zu [\ref{lookAtCart}].
\\Wenn du abwarten möchtest, gehe zu [\ref{waitInCart}].

\block{eavesdroppingSuccess}{Ich kann ein wenig verstehen}

Du schaffst es dich so zu drehen, dass du die Stimmen etwas klarer verstehst. Hin und wieder fehlen dir einige Worte, doch du glaubst trotzdem zu erkennen worum es geht. Die beiden Männer unterhalten sich über einen Aufstand einiger Dorfbewohner, nachdem ihr Meister mehr Gold verlangte. Scheinbar ging die Geschichte für die Dorfbewohner nicht gut aus. Das kurze Lachen der beiden Schurken lässt kein fröhliches Ende vermuten. Außerdem sprechen sie kurz über eine Frau, eine Elfe, die scheinbar eine sehr reizvolle Figur hat. Und über einen kleinen Goblin, allerdings bist du dir nicht sicher wie die letzten beiden Teile zusammen passen - die beiden scheinen Gäste des Mannes zu sein, der deine Entführung beauftragt hat?\\
Danach schweigen die beiden für eine Weile und du überlegst, was du tun solltest.
\\Wenn du dich bemerkbar machen willst, gehe zu [\ref{getAttention}].
\\Wenn du dich weiter umsehen willst, gehe zu [\ref{lookAtCart}].
\\Wenn du abwarten möchtest, gehe zu [\ref{waitInCart}].

\block{waitInCart}{Ich akzeptiere mein Schicksal}

Du bist dir nicht sicher ob es nur ein paar Stunden sind, doch es fühlt sich wie eine Ewigkeit an. Du liegst mit dem Rücken im Wagen und beobachtest den Himmel, an dem hin und wieder dicke weiße Wolken vorbeiziehen. Scheinbar seid ihr inzwischen in einem Wald, denn um euch herum sind immer mehr Bäume zu sehen und der Weg ist noch holpriger geworden als zuvor. Man muss kein Genie sein um zu wissen, dass es sich nicht um eine vielbefahrene Handelsstraße handelt.\\
Wirf einen W20. Ist das Ergebnis mindestens 16, gehe zu [\ref{banditEncounter}].\\
Ansonsten gehe zu [\ref{arrivalAtTower}].

\block{lookAtCart}{Ich sehe mich um}

Du lässt deinen Blick schweifen und versuchst deine aktuelle Lage besser zu erfassen. Der Karren ist aus Holz und hat schon bessere Tage gesehen. Am hinteren Ende befindet sich eine Ladeklappe, die von Metallriegeln auf beiden Seiten gehalten wird. Neben dir befinden sich zwei gefüllte braune Leinensäcke.
\\Wenn du zur Ladeklappe kriechen willst, gehe zu [\ref{loadramp}].
\\Wenn du warten möchtest, gehe zu [\ref{waitInCart}].
\\Wenn du versuchen willst herauszufinden was sich in den Säcken befindet, gehe zu [\ref{inspectCargo}].

\block{loadramp}{Ich krieche zur Ladeklappe}

Es ist ein kleines Kunststück und dauert einige Minuten, aber du schaffst es dich zu drehen und zur Klappe zu kriechen. Die Bretter werden von Nägeln zusammengehalten, denen Wind und Wetter stark zugesetzt haben. Trotzdem glaubst du, dass es kein Kinderspiel wäre die Klappe mit Gewalt zu öffnen - was vor allem an deiner eingeschränkten Bewegungsmöglichkeit liegt.
\\Wenn du versuchen willst die Klappe aufzutreten, gehe zu [\ref{forceLoadramp}].
\\Wenn du versuchen willst die Riegel zu öffnen, gehe zu [\ref{unlockLoadramp}].
\\Wenn du versuchen willst herauszufinden was sich in den Säcken befindet, gehe zu [\ref{inspectCargoFromloadramp}].
\\Wenn du warten möchtest, gehe zu [\ref{waitInCart}].

\block{forceLoadramp}{Ich trete gegen die Ladeklappe}

Es braucht wieder einige Minuten der Vorbereitung, bis du dich auf den Rücken gedreht hast und mit angewinkelten Knien vor der Holzklappe liegst.\\
Lege eine Stärkeprobe mit DC 16 ab. Wenn du erfolgreich bist, gehe zu [\ref{forceLoadrampsuccess}].
Wenn du scheiterst, gehe zu [\ref{forceLoadrampfailed}].

\block{forceLoadrampsuccess}{Ich bin stärker als Holz}

Du konzentrierst dich und sammelst deine Kräfte für einen gezielten Tritt gegen die Seite der Holzklappe. Das Glück ist endlich auf deiner Seite, das Material gibt nach und bricht am Metallriegel. Ein nervöser Blick auf deine Entführer zeigt dir, dass sie überhaupt nicht bemerken, was du da treibst. Schnellst sammelst du dich für einen zweiten Tritt auf der anderen Seite und die Klappe fällt nach unten ab. Einige Herzschläge später rollst du seitwärts vom Wagen und fällst auf den steinigen Boden. Dabei verlierst du 2 Lebenspunkte.\\
\\Gehe zu [\ref{rampEscape}].

\block{inspectCargoFromloadramp}{Ich wende mich den Säcke zu}

Es ist ein kleines Kunststück und dauert einige Minuten, aber du schaffst es dich wieder zurück zu drehen und zu den Säcken zu kriechen. Es ist nicht gerade einfach herauszufinden was sich in den Säcken befindet, wenn die Hände hinter dem Rücken gefesselt sind. Trotzdem schaffst du es dich so zu positionieren, dass einer der Säcke hinter dir liegt und du ihn berühren kannst. Du spürst einige runde Stellen, die mit ausreichend Druck nachgeben. Wenn dich nicht alles täuscht, sind das Kartoffeln in den Säcken.
\\Wenn du versuchen willst dir eine der Kartoffeln zu nehmen, gehe zu [\ref{stealPotatoAfterRamp}].
\\Wenn du zurück zur Ladeklappe kriechen willst, gehe zu [\ref{returnLoadramp}].
\\Wenn du warten möchtest, gehe zu [\ref{waitInCart}].

\block{stealPotatoAfterRamp}{Ich benutze meine Finger}

Natürlich sind die Säcke zugeknotet. Normalerweise wäre das vermutlich kein Hindernis, doch leider ist diese Situation nicht ganz normal. Mit dem Rücken zum Knoten versuchst du zu ertasten wo du wie ziehen musst.
\\Lege eine Fingerfertigkeitsprobe mit DC 15 ab. Wenn du Erfolg hast, gehe zu [\ref{gotPotatoRamp}].
\\Wenn es dir nicht gelingt, gehe zu [\ref{failedPotatoRamp}].

\block{gotPotatoRamp}{Mein Triumph-Moment}

Ein Gefühl des Stolzes überkommt dich, als du spürst wie sich das Seil lockert und du den Knoten lösen kannst. Du ziehst die Öffnung zu Boden und ein paar Kartoffeln rollen über den Boden. Dir ist bewusst, dass du nur an eine deiner Taschen herankommst und dort nicht mehr als eine Kartoffel hineinpasst. Du suchst dir ein Exemplar heraus, dass sich gut anfühlt und packst es ein.
\\Du erhälst eine Kartoffel \getItem{simplePotato}!
\\Wenn du zurück zur Ladeklappe kriechen willst, gehe zu [\ref{returnLoadramp}].
\\Wenn du warten möchtest, gehe zu [\ref{waitInCart}].

\block{failedPotatoRamp}{Vergebene Mühe}

Ein Gefühl der Hilflosigkeit überkommt dich, als du spürst, dass es keinen Sinn hat sich weiter an dem Seil zu versuchen. Solange deine Hände gefesselt sind und du den Knoten nicht sehen kannst, wirst du ihn nicht öffnen können.
\\Wenn du zurück zur Ladeklappe kriechen willst, gehe zu [\ref{returnLoadramp}].
\\Wenn du warten möchtest, gehe zu [\ref{waitInCart}].

\block{returnLoadramp}{Ich krieche zurück zur Ladeklappe}

Du lässt die Kartoffeln Kartoffeln sein und wendest dich wieder den Holzbrettern zu.
\\Wenn du versuchen willst die Klappe aufzutreten, gehe zu [\ref{forceLoadramp}].
\\Wenn du versuchen willst die Riegel zu öffnen, gehe zu [\ref{unlockLoadramp}].
\\Wenn du warten möchtest, gehe zu [\ref{waitInCart}].

\block{forceLoadrampfailed}{Das Holz hat gewonnen}

Du konzentrierst dich und sammelst deine Kräfte für einen gezielten Tritt gegen die Seite der Holzklappe. Zu deinem Pech rührt sich überhaupt nichts, woran auch ein zweiter und ein dritter Tritt nichts ändern. Als du zum vierten Tritt ansetzt, bemerkst du plötzlich einen Schatten über dir und hörst eine tiefe Stimme ``Na, sowas sehen wir hier aber gar nicht gern!'' brummen, bevor du einen harten Schlag auf den Kopf bekommst und das Bewusstsein verlierst.\\
Du verlierst 5 Lebenspunkte.
\\Gehe zu [\ref{wakeUpInCell}].

\block{lookAtCart2}{Ich sehe mich um}
% copy of lookAtCart after the hero has seen the kidnappers
Du lässt deinen Blick schweifen und versuchst deine aktuelle Lage besser zu erfassen. Der Karren ist aus Holz und hat schon bessere Tage gesehen. Am hinteren Ende befindet sich eine Ladeklappe, die von Metallriegeln auf beiden Seiten gehalten wird. Neben dir befinden sich zwei gefüllte braune Leinensäcke. Torlof und sein Kumpane scheinen wieder in ihr Gespräch vertieft zu sein.
\\Wenn du zur Ladeklappe kriechen willst, gehe zu [\ref{loadramp}].
\\Wenn du warten möchtest, gehe zu [\ref{waitInCart}].
\\Wenn du versuchen willst herauszufinden was sich in den Säcken befindet, gehe zu [\ref{inspectCargo}].

\block{inspectCargo}{Ich untersuche die Säcke}

Es ist nicht gerade einfach herauszufinden was sich in den Säcken befindet, wenn die Hände hinter dem Rücken gefesselt sind. Trotzdem schaffst du es dich so zu drehen, dass einer der Säcke hinter dir liegt und du ihn berühren kannst. Du spürst einige runde Stellen, die mit ausreichend Druck nachgeben. Wenn dich nicht alles täuscht, sind das Kartoffeln in den Säcken.
\\Wenn du versuchen willst dir eine der Kartoffeln zu nehmen, gehe zu [\ref{stealPotato}].
\\Wenn du zur Ladeklappe kriechen willst, gehe zu [\ref{loadramp}].
\\Wenn du warten möchtest, gehe zu [\ref{waitInCart}].

\block{stealPotato}{Ich erprobe mein Fingerspitzengefühl}

Natürlich sind die Säcke zugeknotet. Normalerweise wäre das vermutlich kein Hindernis, doch leider ist diese Situation nicht ganz normal. Mit dem Rücken zum Knoten versuchst du zu ertasten wo du wie ziehen musst.
\\Lege eine Fingerfertigkeitsprobe mit DC 15 ab. Wenn du Erfolg hast, gehe zu [\ref{gotPotato}].
\\Wenn es dir nicht gelingt, gehe zu [\ref{failedPotato}].

\block{gotPotato}{Der Knoten löst sich}

Ein Gefühl des Stolzes überkommt dich, als du spürst wie sich das Seil lockert und du den Knoten lösen kannst. Du ziehst die Öffnung zu Boden und ein paar Kartoffeln rollen über den Boden. Dir ist bewusst, dass du nur an eine deiner Taschen herankommst und dort nicht mehr als eine Kartoffel hineinpasst. Du suchst dir ein Exemplar heraus, dass sich gut anfühlt und packst es ein.
\\Du erhälst eine Kartoffel \getItem{simplePotato}!
\\Wenn du zur Ladeklappe kriechen willst, gehe zu [\ref{loadramp}].
\\Wenn du warten möchtest, gehe zu [\ref{waitInCart}].

\block{failedPotato}{Der Knoten sitzt zu fest}

Ein Gefühl der Hilflosigkeit überkommt dich, als du spürst, dass es keinen Sinn hat sich weiter an dem Seil zu versuchen. Solange deine Hände gefesselt sind und du den Knoten nicht sehen kannst, wirst du ihn nicht öffnen können.
\\Wenn du zur Ladeklappe kriechen willst, gehe zu [\ref{loadramp}].
\\Wenn du warten möchtest, gehe zu [\ref{waitInCart}].

\block{unlockLoadramp}{Ich versuche mich an den Riegeln}

Dir ist klar, dass es vermutlich am einfachsten ist die Riegel zu benutzen, wenn du die Ladeklappe öffnen willst. Wenn die nur nicht so weit in der Ecke sitzen würden. Wieder scheitert eine eigentlich einfache Aufgabe daran, dass du gefesselt bist. \\
Wirf eine Akrobatikprobe mit DC 16. Wenn du erfolgreich bist, gehe zu [\ref{rampUnlocked}].
\\Wenn du es nicht schaffst, gehe zu [\ref{rampStaysLocked}].

\block{rampStaysLocked}{Ich nicht sehr gelenkig}

Vermutlich wäre es für einen Zuschauer ein kleines Spektakel gewesen. Du drückst dich mit dem Rücken in die Ecke des Wagens bis deine Fingerspitzen das kühle Metall des linken Riegels spüren. Du schiebst den Riegel vorsichtig mit Zeigefinger und Mittelfinger nach oben, als der Wagen plötzlich über einen Stein fährt und wackelt. Du verlierst den Halt und der kleine Metallstift rutscht wieder zurück. Ein nervöser Blick auf deine Entführer zeigt dir, dass sie überhaupt nicht bemerken, was du da treibst.

Du versuchst dich erneut am Riegel und stellst verbittert fest, dass er noch tiefer nach unten gerutscht ist. Deine Fingerspitzen erreichen ihn nicht mehr und du schaffst es auch nicht noch näher an den Riegel heranzurutschen, du bist einfach nicht gelenkig genug. Nach einigen Minuten erfolgloser Versuche musst du dir eingestehen, dass du hier nichts ausrichten kannst.
\\Wenn du versuchen willst die Klappe aufzutreten, gehe zu [\ref{forceLoadramp}].
\\Wenn du warten möchtest, gehe zu [\ref{waitInCart}].

\block{rampUnlocked}{Ich bin sehr gelenkig}

Vermutlich wäre es für einen Zuschauer ein kleines Spektakel gewesen. Du drückst dich mit dem Rücken in die Ecke des Wagens bis deine Fingerspitzen das kühle Metall des linken Riegels spüren. Du schiebst den Riegel vorsichtig mit Zeigefinger und Mittelfinger nach oben, als der Wagen plötzlich über einen Stein fährt und wackelt. Du verlierst den Halt und der kleine Metallstift rutscht wieder zurück. Ein nervöser Blick auf deine Entführer zeigt dir, dass sie überhaupt nicht bemerken, was du da treibst.

Du sammelst dich einen Moment lang und versuchst es erneut. Diesmal sind deine Bemühungen von Erfolg gekrönt! Der Metallstift fällt zur Seite und der linke Riegel ist offen. Schnell kriechst du zur anderen Seite, bringst dich in Position und kannst auch diesen Riegel öffnen. Die Klappe fällt nach unten ab. Einige Herzschläge später rollst du seitwärts vom Wagen und fällst auf den steinigen Weg. Dabei verlierst du 2 Lebenspunkte.\\
\\Gehe zu [\ref{rampEscape}].

\block{rampEscape}{Ich bin allein}

Während du versuchst nicht vor Schmerz zu stöhnen, rollt der Wagen weiter den Weg entlang. Die beiden Halunken ahnen nicht einmal, dass du nicht wie ein Lamm auf die Schlachtung wartest. Schnell kriechst du vom Weg zu einem der Bäume, während sich der Wagen Meter um Meter von dir entfernt.

Es dauert einiges an Zeit, doch irgendwann kannst du deine Fesseln lösen. Zufrieden mit dir überlegst du, was du nun tun solltest. Wenn du dem Weg folgen willst um herauszufinden wohin man dich gebracht hätte, gehe zu [\ref{iamverycurious}]. Wenn du lieber zurück nach Greifenheim gehen möchtest, gehe zu [\ref{headingHome}].

\block{headingHome}{Zu Hause ist es am schönsten}

Du machst dich auf den Rückweg nach Greifenheim, wobei du darauf achtest dem Weg fern zu bleiben.
Du wirst zwar nie erfahren, warum du entführt wurdest, aber manche Geheimnisse können auch ungelöst bleiben. Dafür wartet ein warmes Essen auf dich.

\textbf{Ende.}

\block{iamverycurious}{Ich bin neugierig}

Du könntest morgen schlecht in den Spiegel sehen, wenn du jetzt nach Hause gehen würdest. Nein, hier gilt es ein Geheimnis zu lüften! Wagemutig machst du dich auf den Weg durch den Wald, immer in der Nähe des Pfades, aber außer Sichtweite. Irgendwann werden deine Entführer sicher bemerken, dass du verschwunden bist. Fürs erste ist es schlauer, den beiden aus dem Weg zu gehen.

Gehe zu [\ref{infiltrateTower}].

\block{banditEncounter}{Ich sehe dunkle Wolken}

Du denkst darüber nach, ob es nur eine Einbildung ist, dass die Wolken immer dunkler werden je länger ihr unterwegs seid. Du versuchst dich aufzurichten, um zu sehen ob sich erkennen lässt wohin die Reise geht, als du ein Funkeln im Wald bemerkst. Gerade als du versuchst zu erkennen worum es sich handelt, verschwindet es.
\\Wirf eine Probe auf Warnehmung mit DC 14. Wenn du erfolg hast, gehe zu [\ref{seeVillagers}].
\\Wenn du es nicht schaffst, gehe zu [\ref{surpriseVillagers}].

\block{seeVillagers}{Wir sind nicht allein}

Du kneifst die Augenlieder zusammen und konzentrierst dich auf die schattigen Stellen des Waldes. Langsam wird das Bild deutlicher - du kannst mehrere Männer mit Waffen erkennen, die sich auf den Angriff vorbereiten. Zwei von ihnen spannen gerade ihre Bögen und zielen auf den Wagen.
\\Wenn du deine Entführer ablenken willst, gehe zu [\ref{distractKidnappers}].
\\Gehe zu [\ref{warnKidnappers}], wenn du deine Entführer warnen willst.
\\Wenn du in Deckung kriechen willst, gehe zu [\ref{villagerAttack}].

\block{warnKidnappers}{Stockholm-Syndrom}

Du reißt deine Beine herum und trittst heftig gegen den Kutschbock. Als sich die beiden Männer herumdrehen, versuchst du durch den Knebel zu schreien und deutest mit dem Kopf wild in Richtung der Angreifer. Gerade als die beiden in diese Richtung sehen, zischen zwei Pfeile durch die Luft, direkt in ihre Richtung. Durch deine Warnung können die beiden Ausweichen und springen vom Karren. Durch den plötzlichen Ruck, der damit einhergeht, landest du mit dem Rücken auf den Holzbrettern.

Wenige Herzschläge später kannst du das Klirren von Waffen hören, gefolgt von panischen Schreien und einem monströsen Gebrüll, das dir das Blut in den Adern gefrieren lässt.
\\Wenn du dich aufrichten willst, um zu sehen was vor sich geht, gehe zu [\ref{seeDefeatedVillagers}].
\\Wenn du am Boden liegen bleiben willst, gehe zu [\ref{stayHiddenVillagers}].

\block{seeDefeatedVillagers}{Blut, Blut überall}

Vorsichtig richtest du dich wieder auf und siehst über den Rand des Karrens. Die Schreie haben aufgehört und du siehst die Entführer, die von den Bäumen zurück zum Wagen laufen. Ihre Kleidung ist blutbeschmiert und einer der beiden hält etwas in der Hand, das du für einen ausgerissenen Arm hälst. Von den Angreifern ist kein Lebenszeichen zu erkennen.

Ohne ein Wort zu sagen schwingen sich beide Männer wieder auf den Kutschbock. In ihrem Blick glaubst du etwas kaltes, furchteinflößendes zu erkennen. Sie drehen sich wieder nach vorn und setzen die Reise wortlos fort.

Gehe zu [\ref{arrivalAtTower}].

\block{villagerAttack}{Ich ziehe den Kopf ein}

Du lässt dich wieder nach hinten fallen, um in Deckung zu gehen. Wenig später kannst du ein Zischen hören. Und dann noch eins. Du drehst den Kopf und erkennst, dass jedem deiner Entführer einen Pfeil im Rücken steckt und die beiden vor Schmerzen stöhnen.

Gehe zu [\ref{villagersAttack}].

\block{surpriseVillagers}{Ich bin überrascht}

Du starrst in den dunklen Wald und glaubst für einen Moment etwas in den Schatten zu erkennen. Allerdings kann es auch einfach nur ein Busch gewesen sein, der sich im Wind bewegt hat. Du lässt dich wieder nach hinten fallen, um abzuwarten, was sich das Schicksal für dich ausgedacht hat, als du plötzlich ein Zischen hörst. Und dann noch eins. Du drehst den Kopf und erkennst, dass jedem deiner Entführer einen Pfeil im Rücken steckt und die beiden vor Schmerzen stöhnen.

Gehe zu [\ref{villagersAttack}].

\block{villagersAttack}{Ein harter Kampf}

Aus dem Wald ertönt Klirren von Metall und wütendes Gebrüll. Kurz darauf siehst du ein Dutzend bewaffneter Männer aus dem Wald rennen, direkt auf den Karren zu. Ein plötzlicher Ruck geht durch den Wagen, als deine beiden Entführer abspringen. Unsanft landest du mit dem Rücken auf den Holzbrettern.

Wenige Herzschläge später kannst du das Klirren von Waffen hören, gefolgt von panischen Schreien und einem monströsen Gebrüll, das dir das Blut in den Adern gefrieren lässt.
\\Wenn du dich aufrichten willst, um zu sehen was vor sich geht, gehe zu [\ref{seeDefeatedVillagersBloody}].
\\Wenn du am Boden liegen bleiben willst, gehe zu [\ref{stayHiddenVillagersBloody}].

\block{stayHiddenVillagers}{Schreie und Ruhe}

Du bleibst versteckt, bis die Kampfgeräusche abklingen. Eine quälend lange Zeit ist gar nichts zu hören, bis du schwere Schritte vernehmen kannst. Dann tauchen deine beiden Entführer am Karren auf, die Gesichter und Kleidung mit dunklem Blut beschmiert.

Ohne ein Wort zu sagen schwingen sie sich wieder auf den Kutschbock. In ihrem Blick glaubst du etwas kaltes, furchteinflößendes zu erkennen. Sie drehen sich wieder nach vorn und setzen die Reise wortlos fort.

Gehe zu [\ref{arrivalAtTower}].

\block{seeDefeatedVillagersBloody}{Es kann nur einen geben}

Vorsichtig richtest du dich wieder auf und siehst über den Rand des Karrens. Die Schreie haben aufgehört und du siehst einen der beiden Entführer, der von den Bäumen zurück zum Wagen läuft. Seine Kleidung ist blutbeschmiert und in der Hand hält er etwas, das du für einen ausgerissenen Arm hälst. Der zweite Entführer liegt mit dem Gesicht nach unten auf dem Boden, regungslos. Von den Angreifern ist kein Lebenszeichen zu erkennen.

Ohne ein Wort zu sagen schwingt sich der Mann wieder auf den Kutschbock. In seinem Blick glaubst du etwas kaltes, furchteinflößendes zu erkennen. Er dreht sich wieder nach vorn und setzt die Reise wortlos fort.

Gehe zu [\ref{arrivalAtTowerSingle}].

\block{stayHiddenVillagersBloody}{Eiskalter Blick}

Du bleibst versteckt, bis die Kampfgeräusche abklingen. Eine quälend lange Zeit ist gar nichts zu hören, bis du schwere Schritte vernehmen kannst. Dann taucht einer der beiden Entführer am Karren auf, Gesicht und Kleidung mit dunklem Blut beschmiert. Von seinem Kumpanen fehlt jede Spur.

Ohne ein Wort zu sagen schwingt er sich wieder auf den Kutschbock. In seinem Blick glaubst du etwas kaltes, furchteinflößendes zu erkennen. Er dreht sich wieder nach vorn und setzt die Reise wortlos fort.

Gehe zu [\ref{arrivalAtTowerSingle}].

\block{distractKidnappers}{Ich bin ein Unruhestifer}

Du reißt deine Beine herum und trittst heftig gegen den Kutschbock. Als sich die beiden Männer herumdrehen, trittst du erneut so heftig du kannst gegen das Holz und siehst sie böse an. Einer von ihnen seufzt und öffnet den Mund um etwas zu sagen, als sich plötzlich ein Pfeil durch seine Kehle bohrt. Bevor sein Kumpane reagieren kann, ereilt ihn das gleiche Schicksal und die beiden Männer sacken leblos vom Karren.

Aus Richtung des Waldes kannst du laute Jubelschreie hören. Wenig später tauchen Männer in deinem Blickfeld auf, die die Kleidung einfacher Bauern tragen. Freudig blicken sie dich an und sagen: ``Heute ist dein Glückstag!'', während sie deine Fesseln lösen. Du erfährst, dass die Männer sich selbst als Widerstandsgruppe bezeichnen, die gegen den dunklen Hexer Osmonias kämpft. Leider, so sagen sie, ist es unmöglich den Hexer in seinem finsteren Turm anzugreifen. Deswegen versuchen sie ihn so gut es geht zu stören, indem sie seine Lieferungen abfangen. Warum du entführt wurdest, können sie dir nicht sagen, doch ihrer Meinung nach war es der Wille der Götter, dass du gerettet wirst.
\\Wenn du versuchen willst zum Turm zu gelangen, gehe zu [\ref{notLeavingNow}].
\\Wenn du der Sache nicht weiter nachgehen willst, gehe zu [\ref{justLeave}].
\\Wenn du die Männer überzeugen willst mit dir zum Turm zu gehen, gehe zu [\ref{letsGoTogether}].

\block{justLeave}{Ich lass es auf sich beruhen}

Vielleicht haben die Dorfbewohner Recht. Vielleicht ist es besser, der Sache nicht weiter nachzugehen und froh zu sein, dass du mit dem Leben davon gekommen bist. Du bedankst dich bei deinen unverhofften Rettern und trittst dann den Heimweg an. Du wirst zwar nie erfahren, warum du entführt wurdest, aber manche Geheimnisse können auch ungelöst bleiben.

\textbf{Ende.}

\block{letsGoTogether}{Wir sollten gemeinsam gehen}

Du erklärst den Widerstandskämpfern, dass es nichts bringt den Hexer ein wenig zu stören. Solche Gefahren müssen erschlagen werden! Du bietest an, sie bei diesem Abenteuer zu begleiten.

Wirf eine Probe auf Überzeugen mit DC 16. Wenn du erfolgreich bist, gehe zu [\ref{aNewAdventure}].
\\Wenn du es nicht schaffst, gehe zu [\ref{allAloneVillagers}].

\block{allAloneVillagers}{Ich bin nicht überzeugend}

Deine Überredungskünste finden leider keinen fruchtbaren Boden. Die Männer erklären dir, dass der Hexer schreckliche Monster in seinem Dienst hat, die einem Mann mit bloßen Händen das Fleisch von den Knochen reißen und verspeisen! Nein, auch wenn sie ihn aufhalten wollen, sie wissen wo ihre Grenzen liegen. Und es wäre auch das beste für dich, wenn du nach Hause gehst.

Wenn du trotzdem heimlich versuchen willst zum Turm zu gelangen, gehe zu [\ref{notLeavingNow}].
\\Wenn du der Sache nicht weiter nachgehen willst, gehe zu [\ref{justLeave}].

\block{aNewAdventure}{Ich bin überzeugend}

Deine Überredungskünste finden zwar fruchtbaren Boden, doch die Männer scheinen von einer tief sitzenden Angst beseelt zu sein. Sie erklären dir, dass der Hexer schreckliche Monster in seinem Dienst hat, die einem Mann mit bloßen Händen das Fleisch von den Knochen reißen und verspeisen! Nein, auch wenn sie ihn aufhalten wollen, sie wissen wo ihre Grenzen liegen.

Aber du, du bist anders. Du siehst erfahren aus, vielleicht kannst du das unmögliche wagen. Doch dann solltest du heimlich sein, denn wer den Hexer offen angreift, der kann wirklich nicht mit einem Sieg rechnen. Sie geben dir einen kleinen Dolch \getItem{simpleDagger}, ein Kurzschwert \getItem{shortSword} und einen Trank der größeren Heilung \getItem{GreaterHealingPotion}. Dir ist klar, dass besonders der Heiltrank für die Männer ein wertvolles Gut ist. Dankbar verabschiedest du dich von deinen Helfern und läufst durch den Wald weiter in Richtung dieses mysterösen Turms.

Gehe zu [\ref{infiltrateTower}].

\block{notLeavingNow}{Ich lass es nicht auf sich beruhen}

Vielleicht haben die Dorfbewohner Recht. Vielleicht ist es besser, der Sache nicht weiter nachzugehen und froh zu sein, dass du mit dem Leben davon gekommen bist.

Leider bist du zu neugierig und es sicher nicht verkehrt, wenn man weiß, warum einen ein Paar Halunke entführen wollte. Außerdem bist du jetzt in der Situation, dass du selbst einschätzen kannst, welches Risiko du eingehst. Die Abenteuerlust packt dich förmlich.

Du verabschiedest dich von deinen Rettern und sagst ihnen, dass du nach Hause gehen wirst. Nachdem sie außer Sichtweite sind, drehst du jedoch um und läufst durch den Wald weiter in Richtung dieses mysterösen Turms.

Gehe zu [\ref{infiltrateTower}].
