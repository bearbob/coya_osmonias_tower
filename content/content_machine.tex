\chapter*{Der Maschinengott}
%im obersten Zimmer des Turms verbirgt sich ein Maschinengott, der dem Hexer Befehle erteilt
%die maschine ist eine magische kriegsmaschine und benötigt seelenenergie um wieder zu funktionieren
% nachtrag: der teil der geschichte hat sich irgendwie doch nicht so ergeben ;)

\block{theUpperLevelWithWarlock}{Die oberen Stockwerke}

Irgendwann kommt ihr an einer weiteren Tür an, vor der ihr stehen bleibt. Dein neuer Meister wühlt in seinem Umhang herum und zückt schließlich einen kleinen Schlüssel, den er in das Schloss der Tür steckt. Du hörst, wie sich ein schwerer Riegel hinter dem Holz bewegt und die Tür sich mit überraschender Leichtigkeit öffnet.
Gehe zu \goto{roomWithPillars}.

\block{roomWithPillars}{Ich sehe Säulen}

Der nachfolgende Raum ist kreisrund und muss ungefähr so groß sein wie die Eingangshalle. Allerdings ist er nicht leer, sondern gefüllt mit knapp einem Dutzend gläsernern Säulen. Im Inneren dieser Säulen wabert eine vioelette Flüssigkeit, die den Raum in ein bedrohliches Licht taucht. Als der Hexer den Raum betritt, flammen an den Wänden Fackeln auf und erhellen die Szenerie. Du kannst nun deutlich erkennen, dass sich in der violetten Flüssigkeit humanoide Formen befinden, die in ihr zu schweben scheinen.

Wenn Ereignis \getEvent{ihaveseentheirfaces} eingetreten ist, gehe zu [\ref{iKnowTheseFaces}]. Ansonsten gehe zu [\ref{weproceedThroughTheRoom}].

\block{weproceedThroughTheRoom}{Durch den Raum}

Entschlossen geht der Hexer durch den Raum und du folgst ihm. Als du einigen der Säulen näher kommst, erkennst du, dass die humanoiden Formen, die du zuvor gesehen hast, Männer mittleren Alters mit breiten Schultern sind. Es ist schwer mehr Details zu erkennen, aber du glaubst, dass die Männer sich alle sehr stark ähneln. Beinahe als wären sie Brüder. Allerdings glaubst du nicht, dass du eine große Familie vor dir hast. Vielmehr muss hier Magie im Spiel sein.

Auf der anderen Seite des Raums befindet sich ein Fenster und daneben eine weitere Treppe, die höher führt. Ihr folgt dieser Treppe.
Gehe zu [\ref{togetherToLevel2}].

\block{iKnowTheseFaces}{Ich kenne diese Gesichter}

Entschlossen geht der Hexer durch den Raum und du folgst ihm. Als du einigen der Säulen näher kommst, erkennst, dass die humanoiden Formen die du zuvor gesehen hast Männer mittleren Alters mit breiten Schultern sind. Es ist schwer mehr Details zu erkennen, aber du bist dir völlig sicher: es sind immer wieder die gleichen zwei Gesichter. Die Gesichter deiner Entführer, Torlof und sein Kumpane.

Wirf eine Arkanaprobe mit DC 14. Wenn du es nicht schaffst, gehe zu [\ref{arcanaFailedMe}]. Wenn du Erfolg hast, und das Ereignis \getEvent{oneKidnapperDiedA}, \getEvent{oneKidnapperDiedB} oder \getEvent{bothKidnappersDied} eingetreten ist, gehe zu [\ref{theseAreClonesAndOneIsMissing}]. Wenn keins dieser Ereignisse eingetreten ist, gehe zu [\ref{theseAreClones}].

\block{arcanaFailedMe}{Ich kann das nicht zuordnen}

Du kannst nur vermuten, dass es einen dunklen Zauber oder eine ähnliche Hexerei gibt, mit der das alles zu tun hat. Aber was für Pläne hat der Hexer, wenn er sich offensichtlich eine Schar von Handlangern heranzüchtet?

Während du noch darüber nachdenkst, erreicht ihr die andere Seite des Raums. Hier befindet sich ein Fenster und eine weitere Treppe, die höher führt. Ihr folgt dieser Treppe.
Gehe zu [\ref{togetherToLevel2}].

\block{theseAreClones}{Ich habe davon gehört}

Du kannst dich düster erinnern, dass es einen Zauber gibt, mit dem man Klone erschaffen kann. Wenn einer der beiden Entführer stirbt, fährt seine Seele in einen der schlafenden Körper. Aber was für Pläne hat der Hexer, wenn er nicht nur einmal, sondern so oft auf den Tod seiner Handlanger vorbereitet sein will?

Während du noch darüber nachdenkst, erreicht ihr die andere Seite des Raums. Hier befindet sich ein Fenster und eine weitere Treppe, die höher führt. Ihr folgt dieser Treppe.
Gehe zu [\ref{togetherToLevel2}].

\block{theseAreClonesAndOneIsMissing}{Mir fehlt etwas}

Du kannst dich düster erinnern, dass es einen Zauber gibt, mit dem man Klone erschaffen kann. Wenn einer der beiden Entführer stirbt, fährt seine Seele in einen der schlafenden Körper. Diese Theorie wird bestätigt, als du eine Säule entdeckst, die keinen Inhalt hat. Vermutlich wurde einer der beiden nach dem Angriff im Wald wieder zum Leben erweckt. Aber was für Pläne hat der Hexer, wenn er nicht nur einmal, sondern so oft auf den Tod seiner Handlanger vorbereitet sein will?

Während du noch darüber nachdenkst, erreicht ihr die andere Seite des Raums. Hier befindet sich ein Fenster und eine weitere Treppe, die höher führt. Ihr folgt dieser Treppe.
Gehe zu [\ref{togetherToLevel2}].

\block{stairsToTheUpperLevels}{Die Treppe nach oben}

Du folgst den Treppenstufen nach oben. Wenig später kommst du zu einer kleinen Tür, die deinen Weg blockiert. Ein kurzer Versuch am Türgriff zeigt, dass die Tür verschlossen ist.

Wenn du \getItem{warlocksKey} besitzt, kannst du zu [\ref{unlockTheDoorWithWarlocksKey}] gehen.
Ansonsten kannst du vor der Tür warten und zu \goto{campingInFrontOfDoor} gehen, oder du musst umkehren und zu [\ref{goToCrossroom}] gehen.

\block{campingInFrontOfDoor}{Ich warte ab}

Du beschließt vor der Tür zu warten. Nach einigen Minuten setzt du dich auf die Treppenstufen und starrst die Wand an. Als dein Blick über die großen Steine wandert, denkst du darüber nach, wie lange dieser Turm hier wohl schon steht. Nach weiteren Minuten legst du vorsichtig dein Ohr an die Tür, doch du kannst nichts hören.

Wenn du weiter warten möchtest, gehe zu \goto{waitingLonger}.
Wenn du nicht noch mehr Zeit verschwenden und die Treppe wieder hinabgehen willst, gehe zu \goto{goToCrossroom}.

\block{togetherToLevel2}{Höher hinauf}

Als ihr am Fenster vorbei kommt, siehst du, dass es inzwischen dunkel geworden ist. Noch immer scheinen die dichten Wolken den Himmel zu bedecken, denn du kannst in diesem flüchtigen Moment keine Sterne oder den Schein des Mondes erkennen.

Euer Weg führt euch die Treppe hinauf in die nächste Etage des Turms. Es geht einen langen Gang entlang, an dessen Ende sich ein kleiner Runde Raum mit 4 Türen befindet, eine in jede Himmelsrichtung. Der Hexer greift nach dem Türgriff der östlichen Tür und öffnet sie. Ihr durchquert ein kleines Lager, du siehst einige Fässer, Säcke und gefüllte Gläser. Am Ende wartet eine weitere Treppe darauf erklommen zu werden. Vom Ende der Treppe strahlt euch bereits ein rötliches Licht entgegen, das wild flackert und furchterregende Schatten an die Wand wirft.

Ihr befindet euch nun im obersten Zimmer des Turms. Keine weitere Treppe, keine Türen, keine Fenster. Nur ein großer, runder Raum, der ungefähr 10 Meter im Durchmesser misst. An der Außenwand sind in kurzem Abstand hunderte schwarzer Kerzen aufgereiht, deren kleine Flammen sich zuckend hin und her bewegen.
Doch die größte Lichtquelle befindet sich mitten im Raum. Auf einem Podest, dass sich wie eine Pyramide zuspitzt, befindet sich in Brusthöhe ein faustgroßer Edelstein. Er ist in eine kleine Metallkralle gefasst, von der Metallfäden verschiedenster Farbe ausgehen. Du hast selten größere Juwelen gesehen, doch besonders das rote Licht, welches dieser Stein ausstrahlt, zieht dich in seinen Bann.

Wirf einen Charismarettungswurf mit DC 16. Wenn du es nicht schaffst, gehe zu [\ref{cannotresistthegem}]. Wenn du Erfolg hast, gehe zu [\ref{resistingTheGem}].

\block{resistingTheGem}{Ich kann mich abwenden}

Für einen Moment bist du wie verzaubert. Als deine Augen den Edelstein erblicken, vergisst du die Welt um dich herum. Doch dann schaffst du es, deine Gedanken wegzulenken und abwärts zu sehen. Erst jetzt fällt dir auf, dass sich am Podest und auf dem Boden schwarze Flecken befinden. Unter den Flecken, über den gesamten Boden verteilt, befindet sich ein riesiger Runenkreis mit merkwürdigen Symbolen, die mit Sicherheit keinem guten Zweck gewidmet sind.

Als du den Blick zur Decke schweifen lässt, erkennst du eine Frau, eine Elfe, die dort in einigen Metern Höhe kopfüber aufgehangen ist. Sie scheint bewusstlos, ihr blondes Haar ist blutverschmiert und hängt nach unten. Immer wieder fallen kleine Bluttropfen von ihr hinab auf den Edelstein, zweifelsohne absichtlich.

Die krächzende Stimme des Hexers reißt dich aus deinen Gedanken. ``Nur keine Scheu, fass den Stein an.'', sagt er mit einem bösartigen Unterton.
Wenn Ereignis \getEvent{deceivedTheWarlock} eingetreten ist, gehe zu [\ref{idonthavetotouchthegem}]. Wenn nicht, gehe zu [\ref{hewantsmetotouchthegem}].

\block{idonthavetotouchthegem}{Die Scharade hat ein Ende}

Es reicht, das Schauspiel endet jetzt! Ruckartig drehst du dich um und gehst auf den Hexer los. Der Ausdruck der Überraschung in seinen Augen weicht schnell dem Schock, als er realisiert, dass er keine Macht über dich hat!

\monsterWarlock{ikilledthewarlockInTheHighTower}{anEvilWarlockKilledMe}

\block{ikilledthewarlockInTheHighTower}{Nur einer von uns wird leben}

Der Hexer ist dem Tode geweiht, daran besteht für keinen von euch beiden ein Zweifel. Ein atemloser Schrei versucht sich aus seinem Mund zu quälen, doch er bringt nicht mehr als ein Keuchen hervor. Während du zusiehst kehrt sich das Weiße in seinen Augen nach oben und er sinkt auf die Knie. Noch bevor sein Gesicht auf dem harten Steinboden aufschlägt, hört er auf zu zucken. Reglos bleibt er liegen. Er ist tot.

Markiere das Ereignis \getEvent{theWarlockDied}.
Wenn du den Hexer durchsuchen willst, gehe zu [\ref{lootTheWarlockInTheTower}].
Wenn du das Podest mit dem Edelstein untersuchen willst, gehe zu [\ref{inspectTheGem}].
Wenn du die Elfe genauer betrachten willst, gehe zu [\ref{inspectTheElf}].
Wenn du den Raum verlassen willst, gehe zu [\ref{downToLayer2}].

\block{lootTheWarlockInTheTower}{Was sein war ist mein}

Du zögerst keine Sekunde und durchsuchst den Toten nach Dingen, die dir helfen könnten.
Kurze Zeit später kannst du einen Zauberstab \getItem{warlockStaff}, einen geschwungenden Dolch \getItem{warlocksDagger}, einen kleinen Kupferschlüssel \getItem{warlocksKey} und einen goldenen Ring \getItem{goldenWarlockRing} dein Eigen nennen.

Außerdem ist dir aufgefallen, dass der linke Arm des Mannes von seltsamen Tätowierungen übersät ist, die dir wie Runen vorkommen. Du erkennst die gleichen Muster auf dem Boden um den Edelstein wieder, offensichtlich besteht ein Zusammenhang. Viele kleine Narben und Schnitte durchbrechen die Muster, einige scheinen noch recht frisch und gerade erst verheilt zu sein.

Wenn du das Podest mit dem Edelstein untersuchen willst, gehe zu \goto{inspectTheGem}.
Wenn du die Elfe genauer betrachten willst, gehe zu \goto{inspectTheElf}.
Wenn du den Raum verlassen willst, gehe zu \goto{downToLayer2}.

\block{inspectTheElf}{Ich sehe mir die Elfe genauer an}

Du betrachtest die Elfe über die genauer. Sie ist mit deinem dicken Seil gefesselt und mit den Füßen nach oben an der Decke aufgehangen. Ihre sonst feinen Gesichtszüge wirken wie von Schmerzen verzerrt und immer wieder tropft Blut von ihr herab auf den Edelstein.
Du kannst nicht erkennen, wo sie verwundet ist.
Die Teile ihres blauen Seidengewands, die du durch die Fesseln sehen kannst, wirken sauber und unbeschadet.

Zwischen euch liegen mehr als zwei Körperlängen, sodass du keine Chance hast, sie zu erreichen. Wenn du dir das Podest genauer ansehen willst, gehe zu \goto{inspectTheGem}.
Wenn du den Raum verlassen willst, gehe zu \goto{downToLayer2}.

\block{waitingLonger}{Ich warte weiter ab}

Du weißt nicht, wie lange du schon auf der Treppe sitzt, aber es erscheint dir noch nicht sinnvoll, jetzt zu gehen. Also harrst du weiter aus. Und weiter. Für einen Moment merkst du, wie deine Gedanken abschweifen und du an den Abend in Greifenheim denken musst, bevor die ganzen Ereignisse ihren Lauf nahmen, die dich hier her geführt haben.

Ein leichter Schmerz im Nacken lässt dich hochschrecken. Dein Kopf ist nach vorne gekippt, als du eingeschlafen bist. Du atmest tief ein und aus und reibst dir die Augen. Plötzlich hörst du das knackende Geräusch des Türschlosses und springst hoch. Die Tür wird aufgezogen und auf der anderen Seite starrt dich ein bleiches, eingefallenes Gesicht erstaunt an. Alles an dem Mann schreit nach ``Hexer''. Die knochigen Arme, die schwarze, lange Robe, die trotzdem nicht die jämmerliche Statur verbergen kann, die eingefallene, blasse Haut und der grünlich leuchtende Zauberstab in seiner Hand.

Der Moment der Überraschung ist eindeutig auf deiner Seite, der Hexer ist mit der Situation völlig überfordert.
Wenn du mit ihm reden willst, gehe zu \goto{talkingWithWarlock}.
Wenn du ihn die Treppe hinabreißen willst, gehe zu \goto{pullinWarlock}.
Wenn du ihn angreifen willst, gehe zu \goto{attackingWarlock}.
Wenn du wegrennen willst, gehe zu \goto{escapingWarlock}.

\block{attackingWarlock}{Ich greife ihn an}

Du nutzt deinen Vorteil und springst die letzten beiden Stufen hinauf, um den Hexer anzugreifen bevor er die Chance hat das gleiche mit dir zu machen.

\textit{Durch den Überraschungsmoment kannst du zuerst angreifen. Du hast Vorteil auf deinen Geschicklichkeitswurf.}

\monsterWarlock{ikilledthewarlockOnTheStairs}{anEvilWarlockKilledMe}

\block{ikilledthewarlockOnTheStairs}{Einer muss fallen}

Der Hexer ist dem Tode geweiht, daran besteht für keinen von euch beiden ein Zweifel. Ein atemloser Schrei versucht sich aus seinem Mund zu quälen, doch er bringt nicht mehr als ein Keuchen hervor. Während du zusiehst, kehrt sich das Weiße in seinen Augen nach oben und er fällt vorwärts. Du machst einen Schritt zur Seite und lässt ihn an dir vorbei die Stufen hinab fallen.
Sein Körper fällt über einige Stufen hinweg und lautes Knacken sagt dir, dass mehrere seiner Knochen gerade gebrochen sind. Reglos bleibt er einige Meter unter dir liegen. Er ist tot.

Markiere Ereignis \getEvent{theWarlockDied}.
Wenn du in den Raum gehen willst, gehe zu \goto{roomWithPillarsAlone}.
Wenn du den Hexer durchsuchen willst, gehe zu [\ref{lootTheWarlockOnStairs}].

\block{pullinWarlock}{Ich packe zu}

Du verschwendest keinen Herzschlag! So schnell du kannst packst du die dunkle Robe des Mannes und ziehst ihn die Treppe hinab. Du hörst ihn überrascht schreien und spürst, wie sich seine knochigen Finger an dir festkrallen.

Wirf eine Akrobatikprobe mit DC 13. Wenn du erfolgreich bist, gehe zu \goto{surpisePullinWarlock}.
Wenn du es nicht schaffst, gehe zu \goto{bothgetpulled}.

\block{bothgetpulled}{Wir fallen}

Der Hexer hat keine Chance seinem Schicksal zu entgehen. Als du nach ihm greifst und ihn die Treppe hinabreißt, finden seine Hände leider für einen kurzen Moment Halt an deinem Arm, bevor ihn die Schwerkraft hinabreißt. Der Schwung, den er hatte, reicht, um dich aus dem Gleichgewicht zu bringen. Du kämpfst dagegen an, doch schlussendlich zieht es auch dich nur wenige Herzschläge später hinab in die Tiefe!

Als der Hexer auf der ersten Stufe aufschlägt, gibt er noch einen grellen, schmerzerfüllten Schrei von sich. Beim zweiten, tieferen Aufprall kannst du schon keine Laute mehr hören.
Mit einem dumpfen Poltern wird sein Leib noch weitere Stufen hinabgerissen, Meter um Meter.
Reglos und mit verdrehten Gliedern bleibt er auf dem kalten Stein liegen. Du schaffst es zum Glück den ersten Aufprall besser abzufedern. Trotzdem brichst du dir den Arm. Dann die Rippen und das linke Schienbein. Dein letzter Aufschlag wäre sicher tödlich, würde nicht der Hexer bereits tot am Boden liegen und deinen Fall bremsen. Stechender Schmerz durchfährt dich, als du auf dem harten Stein aufkommst. Stöhnend schaffst du es dich abzurollen und aufrecht zu setzen.

Wenn du \getItem{GreaterHealingPotion} hast, kannst du zu \goto{healingAfterFalling} gehen.
Ansonsten gehe zu \goto{myEndAfterFalling}.

\block{myEndAfterFalling}{Ich stehe das nicht durch}

Mit zitternder Hand wischst du dir Blut aus den Mundwinkeln. Jeder Atemzug schmerzt furchtbar und du findest nicht die Kraft um aufzustehen. Vermutlich würden deine gebrochenen Beine sofort nachgeben. Plötzlich musst du husten und würgst hellrotes Blut hervor.

Während du überlegst, wie du aus dieser Situation entkommen kannst, verschwimmt deine Sicht und deine Hand wird schwerer und schwerer. Keuchend drehst du deinen Kopf zu dem traurigen Haufen, der mal der Hexer war. Für ihn ist jede Hoffnung vergebens. So wie es aussieht, musst du ihm leider in Kürze folgen. Deine Brust hebt und senkt sich immer weniger. Ein letztes Zucken geht durch deine Finger, bevor dein Kopf schwer zur Seite sackt. Leider wird der schwarze Turm dein düsteres Grab.

\textit{Ende.}

\block{healingAfterFalling}{Ich stehe das durch}

Mit zitternder Hand greifst du nach dem Heiltrank und betest zu den Göttern, dass er nicht zerbrochen ist. Scheinbar wurdest du erhöhrt, denn die Flasche ist beinahe unbeschädigt. Du versuchst den Schmerz zu ignorieren und die Flasche zu öffnen. Eilig trinkst du den Trank und lehnst dich stöhnend zurück.

Es vergehen einige Minuten, bis dich eine wohlige Wärme erfüllt. Du spürst deinen Herzschlag stärker werden und hörst deine Knochen knacken. Dann überkommt dich eine Welle noch größeren Schmerzes, bevor alles abebbt und du dich wieder besser fühlst. Keuchend drehst du deinen Kopf zu dem traurigen Haufen, der mal der Hexer war. Für ihn ist jede Hoffnung vergebens.
Markiere Ereignis \getEvent{theWarlockDied}.

Wenn du die Treppe hinauf in den Raum gehen willst, gehe zu \goto{roomWithPillarsAlone}.
Wenn du den Hexer durchsuchen willst, gehe zu [\ref{lootTheWarlockOnStairs}].

\block{surpisePullinWarlock}{Er fällt}

Der Hexer hat keine Chance. Als du nach ihm greifst und ihn die Treppe hinabreißt, suchen seine Hände verzweifelt nach Halt an dir, doch die Schwerkraft arbeitet gegen ihn - und schnell.
Als er auf der ersten Stufe aufschlägt, gibt er noch einen grellen, schmerzerfüllten Schrei von sich. Beim zweiten, tieferen Aufprall kannst du schon keine Laute mehr hören.
Mit einem dumpfen Poltern wird sein Leib noch weitere Stufen hinabgerissen, Meter um Meter.
Reglos und mit verdrehten Gliedern bleibt er auf dem kalten Stein liegen.
Für ihn ist jede Hoffnung vergebens.
Markiere Ereignis \getEvent{theWarlockDied}.

Wenn du die Treppe hinauf in den Raum gehen willst, gehe zu \goto{roomWithPillarsAlone}.
Wenn du den Hexer durchsuchen willst, gehe zu [\ref{lootTheWarlockOnStairs}].

\block{lootTheWarlockOnStairs}{Ich sollte das behalten}

Du zögerst keine Sekunde und durchsuchst den Toten nach Dingen, die dir helfen könnten.
Du versuchst dabei zu ignorieren, dass der Sturz einige Spitze Knochen durch die Haut des Hexers gejagt hat und dickes, rotes Blut auf die Treppenstufen läuft.
Kurze Zeit später kannst du einen Zauberstab \getItem{warlockStaff}, einen geschwungenden Dolch \getItem{warlocksDagger}, einen kleinen Kupferschlüssel \getItem{warlocksKey} und einen goldenen Ring \getItem{goldenWarlockRing} dein Eigen nennen.

Außerdem ist dir aufgefallen, dass der linke Arm des Mannes von seltsamen Tätowierungen übersät ist, die dir wie Runen vorkommen. Viele kleine Narben und Schnitte durchbrechen die Muster, einige scheinen noch recht frisch und gerade erst verheilt zu sein.

Deine Neugier ist geweckt und du blickst die Treppe hinauf zu der Tür, aus der der Hexer kam. Ohne zu zögern läufst du die Stufen wieder hinauf. Gehe zu \goto{roomWithPillarsAlone}.

\block{talkingWithWarlock}{Ich will nur reden}

Aus irgendeinem Grund hälst du es für sinnvoll, ein Gespräch mit dem Mann zu suchen. Als du den Mund öffnest, um etwas zu sagen, reißt er seinen Zauberstab empor.
Sofort spürst du eine Kraft auf dich einwirken, die deinen Körper gegen deinen Willen bewegen will.
Du vergisst, was du sagen wolltest und versuchst dich dem Angriff auf deinen Geist mit aller Mühe zu erwehren.

Wirf einen Rettungswurf auf Charisma mit DC 18. Wenn du erfolgreich bist, gehe zu [\ref{youShallNotControlMeInTower}]. Wenn du es nicht schaffst, gehe zu [\ref{yesMasterControlMeInTower}].

\block{youShallNotControlMeInTower}{Ich bleibe mein eigener Herr}

Keine Fremde Macht soll dich kontrollieren! Angestrengt versuchst du den Angriff auf deinen Geist abzuwehren. Wie in Wellen prallt es auf dich ein und versucht Stück für Stück dir die Sinne zu rauben. Du hast das Gefühl tausende Stimmen zu hören, die direkt in deinem Kopf flüstern und dir befehlen einfach aufzugeben. Zuerst verspürst du ein seltsames Kribbeln in den Fingern, dann verschwimmt für einen Moment deine Sicht. Doch du bist nicht bereit aufzugeben! Endlos scheinende Sekunden drängst du zurück, bis du plötzlich Klarheit verspürst. Das Wispern hört auf. Du bist wieder allein in deinem Kopf.

Der Hexer scheint nicht bemerkt zu haben, dass sein Zauber fehlgeschlagen ist.
Er fängt an höhnisch zu gackern: ``Sieh an, sieh an, ist das Lamm allein zur Schlachtbank gelaufen!''.
Als er sich umdreht und sagt ``Folge mir.'' überlegst du, ob das deine große Chance ist.

Wenn du so tun willst als hätte der Zauber Erfolg gehabt, gehe zu [\ref{yesMasterControlMeButInTruthYouDontInTower}].
Wenn du den Hexer von hinten angreifen willst, gehe zu [\ref{surpisePullinWarlock}].

\block{yesMasterControlMeButInTruthYouDontInTower}{Ich täusche ihn}

Du beschließt das Spiel weiter mitzuspielen. Schweigend setzt du einen Fuß vor den anderen und folgst dem Hexer.
Markiere Ereignis \getEvent{deceivedTheWarlock} und gehe zu \goto{roomWithPillars}.

\block{yesMasterControlMeInTower}{Ich bin nicht mein eigener Herr}

Keine Fremde Macht soll dich kontrollieren! Angestrengt versuchst du den Angriff abzuwehren. Wie in Wellen prallt es auf dich ein und versucht Stück für Stück dir die Sinne zu rauben. Du hast das Gefühl tausende Stimmen zu hören, die direkt in deinem Kopf flüstern und dir befehlen einfach aufzugeben. Zuerst verspürst du ein seltsames Kribbeln in den Fingern, dann verschwimmt für einen Moment deine Sicht. Du versuchst die Stimmen aus deinem Kopf zu verbannen, doch vergebens.

Deine Muskeln entspannen sich. Der Hexer fängt an höhnisch zu gackern: ``Sieh an, sieh an, ist das Lamm allein zur Schlachtbank gelaufen!''.
Als er sich umdreht und sagt ``Folge mir.'' hast du keine Wahl. Du fühlst dich wie ein Passagier in deinem eigenen Körper, der anfängt sich zu bewegen und dem Mann folgt.
Gehe zu \goto{roomWithPillars}.

\block{escapingWarlock}{Ich nehme die Beine in die Hand}

Du weißt nicht genau, was du erwartet hast, aber dein Bauchgefühl schreit dich förmlich an, dass du aus dieser Situation entkommen musst. Du willst nichts mit diesem Mann zu tun haben! Eilig drehst du dich um und versuchst die Treppe hinabzurennen. Doch noch bevor du den ersten Schritt machen kannst, hälst du plötzlich inne. Deine Beine gehorchen dir nicht!

Der Hexer fängt an höhnisch zu gackern: ``Sieh an, sieh an, ist das Lamm allein zur Schlachtbank gelaufen!''. Gegen deinen Willen drehst du dich um und starrst in die zufrieden grinsende Fratze des Mannes, der seinen Zauberstab auf Augenhöhe hält. ``Na dann, wenn du es schon nicht erwarten kannst, dann lass uns gehen!''

Er dreht sich ohne ein weiteres Wort wieder um und läuft zurück in den Raum, aus dem er kam. Obwohl du dich mit aller Mühe dagegen wehrst, ergreift eine unsichtbare Macht wieder Besitz von deinem Körper und zwingt dich ihm zu folgen.
Gehe zu \goto{roomWithPillars}.

\block{inspectTheGem}{Ich sehe mir den Edelstein genauer an}

Du widmest deine Aufmerksamkeit dem Edelstein. Als du ihn ansiehst, spürst du sofort wieder das Verlangen ihn zu berühren. Schnell wendest du deine Augen nach unten ab. Dein Blick fällt auf das Podest. Erst jetzt bemerkst du, dass es aus Metall besteht und scheinbar nicht auf dem Boden steht, sondern \textit{durch} den Boden geht. Knapp über dem Boden siehst du eine kleine Platte, auf der die Zeichen ``F.R.E-Y4'' eingraviert sind.

Wirf eine Weisheitsprobe mit DC 12. Wenn du es schaffst, gehe zu \goto{iHaveMachineKnowledge}.
Wenn nicht, gehe zu \goto{skippingMachineKnowledge}.

\block{skippingMachineKnowledge}{Ich kann das nicht zuordnen}

Du kannst mit dieser Gravur nicht viel anfangen. Dir ist klar, dass das Gebilde vor dir nicht nur ein einfacher Podest ist, aber wer kann schon sagen, welchem Zweck es wirklich dient. Tatsache ist, dass das Leuchten des Edelsteins noch immer ein mulmiges Gefühl in deiner Magengegend auslöst. Vielleicht ist es besser zu gehen.

Wenn du versuchen willst dir den Edelstein zu nehmen, gehe zu \goto{grabbingTheGem}.
Wenn du lieber die Finger von der ganzen Sache lassen und den Raum verlassen willst, gehe zu \goto{downToLayer2}.

\block{iHaveMachineKnowledge}{Das kommt mir bekannt vor}

Du kennst diese Art von Gravur. Du hast eine ähnliche Platte am Bauch des Nimmerleer in Greifenheim gesehen! In deinem Kopf fügen sich gerade einige Puzzelteile zusammen - du stehst auf einem Maschinengott! Mit Sicherheit gehört dieses Podest bereits zu dem Konstrukt. Du kannst zwar nicht sagen, was der Hexer vorhatte, aber mit Sicherheit war es nichts Gutes. Und du bist dir genauso sicher, dass dieser Maschinengott keine guten Neuigkeiten bringen wird.

Wenn du den Podest genauer untersuchen willst, gehe zu \goto{inspectingThePodest}.
Wenn du versuchen willst dir den Edelstein zu nehmen, gehe zu \goto{grabbingTheGem}.
Wenn du lieber die Finger von der ganzen Sache lassen und den Raum verlassen willst, gehe zu \goto{downToLayer2}.

\block{grabbingTheGem}{Ich nehme den Edelstein}

Du hast zwar kein gutes Gefühl dabei, aber du willst auch sicher gehen, dass niemand das Werk des Hexers fortführen kann. Und dieser Edelstein scheint eine Schlüsselrolle zu spielen. Eilig streckst du die Hand aus und packst den Stein. Sofort umgibt dich Stille. Alles ist schwarz.

Gehe zu \goto{deathStranding}.

\block{hewantsmetotouchthegem}{Ich soll es berühren}

Du stehst noch immer unter dem Zauber des Hexers, der deinen Körper kontrolliert. Gegen deinen Willen bewegen sich deine Füße in Richtung des Edelsteins! Du hörst den Hexer noch sagen: ``Nur keine Scheu, erfülle deinen Zweck!''

Nenn es Todesahnung, nenn es schlechtes Bauchgefühl, doch dein Geist sträubt sich vehement gegen den Befehl! Wirf einen Charismarettungswurf mit DC 12. Wenn du erfolgreich bist, gehe zu [\ref{noiwillnottouchthisgem}]. Wenn du versagst, gehe zu [\ref{deathbyGem}].

\block{noiwillnottouchthisgem}{Ich werde das nicht tun}

Du bist dir sicher, dass es dein Todesurteil ist, wenn du den Stein berührst. Magische Kontrolle oder nicht, du wirst dein Leben nicht ohne einen Kampf opfern! Während sich deine Füße wie von allein weiter bewegen, tobt in dir ein Gefecht mit der dunklen Magie, die dich beherrscht. Schlussendlich kommst du wieder zu Sinnen und hast die volle Kontrolle zurück.

Du bist frei. Und das keinen Moment zu spät, als du nach unten siehst, hast du schon die Hand ausgestreckt um den Edelstein zu berühren. Doch dazu wird es nicht kommen.

Gehe zu [\ref{idonthavetotouchthegem}]

\block{cannotresistthegemAlone}{Ich sollte es berühren}

Du bist wie verzaubert. Als deine Augen den Edelstein erblicken, vergisst du alles um dich herum. Du achtest gar nicht darauf, was sich noch im Raum befindet.
In deinen Gedanken ist nur dieser riesige, wunderschöne Edelstein.

Wenn du deinem Gefühl folgen und den Stein nehmen willst, gehe zu \goto{whyfightagainstthegem}.
Wenn du stattdessen versuchen willst an etwas anderes zu denken, gehe zu \goto{thinkaboztsomethinelse}.

\block{thinkaboztsomethinelse}{Ich muss den Kopf frei bekommen}

Du bemerkst, dass dein Körper nicht mehr dir gehört und ein dunkler Zauber seine Macht auf dich ausübt! Du siehst dir selbst dabei zu, wie du die Hand ausstreckst und ein schreckliches Gefühl überkommt dich.
Du bist dir sicher, dass es dein Todesurteil ist, wenn du den Stein berührst! Magische Kontrolle oder nicht, du wirst dein Leben nicht ohne einen Kampf opfern! Während sich deine Füße wie von allein weiter bewegen, tobt in dir ein Gefecht mit der dunklen Magie, die dich beherrscht.
Voller Verzweiflung beißt du dir auf die Zunge, bis dickes rotes Blut deinen Mund füllt und der stechende Schmerz deinen Körper zusammenkrümmt. In dem Moment, in dem sich deine Augen von dem Stein abwenden, kommst du wieder zu Sinnen und hast die volle Kontrolle zurück.

Du verlierst 13 Lebenspunkte.

Es dauert einige Minuten, bis der Schmerz abebbt und die Blutung abnimmt. Der Geschmack deines eigenen Blutes ist widerlich, doch ein geringer Preis für dein Leben.
Erst jetzt fällt dir auf, dass sich am Podest und auf dem Boden schwarze, krustige Flecken befinden. Unter den Flecken und über den gesamten Boden verteilt, befindet sich ein riesiger Runenkreis mit merkwürdigen Symbolen, die mit Sicherheit keinem guten Zweck gewidmet sind.

Als du den Blick zur Decke schweifen lässt, erkennst du eine Frau, eine Elfe, die dort in einigen Metern Höhe kopfüber aufgehangen ist. Sie scheint bewusstlos, ihr blondes Haar ist blutverschmiert und hängt nach unten. Immer wieder fallen kleine Bluttropfen von ihr hinab auf den Edelstein, zweifelsohne absichtlich.

Wenn du das Podest mit dem Edelstein genauer untersuchen willst, gehe zu [\ref{inspectTheGem}].
Wenn du die Elfe genauer betrachten willst, gehe zu [\ref{inspectTheElf}].
Wenn du den Raum lieber wieder verlassen willst, gehe zu [\ref{downToLayer2}].

\block{whyfightagainstthegem}{Es ist mein gutes Recht}

Du musst ihn haben, du sollst ihn haben! Das Schicksal hat dich hier her geführt, damit du dir diesen Stein nimmst. Dein ganzes Leben lief auf diesen Moment zu. Und gleich gehört er dir. Du musst nur noch zupacken. Ja, du musst den Edelstein berühren. Dann hast du es geschafft.
Gehe zu [\ref{deathbyGem}].

\block{cannotresistthegem}{Ich will es berühren}

Du bist wie verzaubert. Als deine Augen den Edelstein erblicken, vergisst du alles um dich herum. Du achtest gar nicht darauf, was sich noch im Raum befindet. Unterbewusst bemerkst du, dass der Hexer mit dir spricht, doch es ist wie ein unverständliches Rauschen in deinen Ohren. Und du willst ihm nicht zuhören. Du musst ihm nicht zuhören. Nein, du musst gar nichts. Nur zum Edelstein, ja. Du musst den Edelstein berühren.
Gehe zu [\ref{deathbyGem}].

\block{deathbyGem}{Ich habe es berührt}

Wie in Trance kommst du dem Stein näher, Schritt um Schritt, bis deine Finger ihn berühren können. Irgendetwas in dir wehrt sich, ist unruhig. Doch niemand wird dich jetzt noch aufhalten, nicht einmal du selbst! Du streckst die Hand aus und berührst den Stein. Sofort umgibt dich Stille. Die Unruhe hört auf. Alles ist schwarz.

Gehe zu \goto{deathStranding}.

\block{deathStranding}{Ich bin verloren}

Du weißt nicht, wie viel Zeit vergeht. Dein Körper ist wie betäubt, du spürst nichts, siehst nichts und hörst nichts. Dann wird es plötzlich wärmer. Erst jetzt bemerkst du, wie kalt dir war. Doch auch die Kälte ist bald vergessen, als die Wärme nicht mehr abnimmt und sich zu Hitze entwickelt. Und schon bald wird aus der Hitze ein schmerzvolles Brennen. Du kannst das Gefühl nicht beschreiben, doch es will einfach nicht aufhören. Wenn du sterben könntest, wärst du inzwischen gestorben. So viel Schmerz kann keine Seele ertragen. Doch irgendetwas hält dich auf! Du kannst nicht in die nächste Welt, der Weg ist versperrt!

Du weißt nicht, wie viel Zeit vergeht. Tage? Jahre? Jahrhunderte? Deine Welt besteht nur noch aus Verzweiflung und Schmerz, der jeden klaren Gedanken verhindert.

\textbf{Ende.}

\block{reachingTheWindow}{Wenn sich eine Tür schließt, öffnet sich ein Fenster}

Du erreichst mit schmerzenden Fingern den Fenstersims und kannst dich endlich wieder richtig festhalten. Dein Kletterpfad war vermutlich nicht der einfachste, aber er hat dich ans Ziel geführt. Über die beiden Momente, in denen du fast den Halt verloren und metertief gefallen wärst, denkst du nicht weiter nach.

Du atmest tief ein und ziehst dich nach oben. Vor dir liegt ein großer, kreisrunder Raum, der von einem violetten Schimmer erhellt wird. Knapp ein Dutzend gläserner Säulen sind die Quellen dieses Lichts. In ihnen wabert eine dunkle Flüssigkeit, in der du jeweils ein menschliche Form erkennen kannst. Irgendetwas... oder irgendjemand steckt in diesen Säulen.

Auf deiner rechten Seite befindet sich eine Treppe, die entlang der Außenwand in ein höheres Stockwerk führt. Auf der dir gegenüber liegenden Seite des Raums kannst du eine kleine Tür erkennen.

Wenn du der Treppe nach oben folgen willst, gehe zu [\ref{goToLevel2Alone}].
Wenn du die Säulen genauer betrachten willst, gehe zu [\ref{inspectTheVioletPillars}].
Wenn du zur Tür gegen willst, gehe zu [\ref{theLockedDoorFromTheOtherSide}].

\block{theLockedDoorFromTheOtherSide}{Ich sehe mir die Tür an}

Du gehst zielorientiert durch den Raum und gelangst zu der Tür. Ein kurzer Griff am Türknauf zeigt dir, dass sie verschlossen ist. Da das gute Stück sehr stabil aussieht, wirst du hier ohne passendes Werkzeug oder den richtigen Schlüssel keine Chance haben.

Wenn du der Treppe nach oben folgen willst, gehe zu [\ref{goToLevel2Alone}].
Wenn du die Säulen genauer betrachten willst, gehe zu [\ref{inspectTheVioletPillars}].

\block{unlockTheDoorWithWarlocksKey}{Jede Tür hat einen Schlüssel}

Wie es das Schicksal will, scheint der Schlüssel, den du dem Hexer genommen hast, genau in das Schloss zu passen. Weder ein Zufall, noch sonderlich überraschend. Du öffnest die Tür. Gehe zu \goto{roomWithPillarsAlone}.

\block{roomWithPillarsAlone}{Ich bin nicht allein}

Vor dir liegt ein großer, kreisrunder Raum, der von einem violetten Schimmer erhellt wird. Knapp ein Dutzend gläserner Säulen sind die Quellen dieses Lichts. In ihnen wabert eine dunkle Flüssigkeit, in der du jeweils ein menschliche Form erkennen kannst. Irgendetwas... oder irgendjemand steckt in diesen Säulen.

Auf der dir gegenüberliegenden Seite, vorbei an diesen Säulen, befindet sich ein Fenster. Links neben dem Fenster kannst du Treppenstufe erkennen, die entlang der Außenwand in ein höheres Stockwerk führen.

Wenn du der Treppe nach oben folgen willst, gehe zu [\ref{goToLevel2Alone}].
Wenn du die Säulen genauer betrachten willst, gehe zu [\ref{inspectTheVioletPillars}].

\block{inspectTheVioletPillars}{Ich untersuche die Säulen}

Als du einigen der Säulen näher kommst, erkennst, dass die humanoiden Formen, die du zuvor gesehen hast, Männer mittleren Alters mit breiten Schultern sind. Es ist schwer mehr Details zu erkennen, aber du bist dir völlig sicher: es sind immer wieder die gleichen zwei Gesichter. Die Gesichter deiner Entführer!

Wirf eine Arkanaprobe mit DC 14. Wenn du es nicht schaffst, gehe zu [\ref{arcanaFailedMeAndImAlone}]. Wenn du Erfolg hast, gehe zu \goto{theseAreClonesAndSomeAreMissingAndIKnowWhy} wenn \getEvent{bothKidnappersDied} eingetreten ist, sonst gehe zu [\ref{theseAreClonesAndSomeAreMissing}].

\block{arcanaFailedMeAndImAlone}{Ich kann damit nichts anfangen}

Du kannst nur vermuten, dass es einen dunklen Zauber oder eine ähnliche Hexerei gibt, mit der das alles zu tun hat. Aber um was für einen Plan kann es sich handeln, wenn man dafür offensichtlich eine Schar von Handlangern heranzüchten muss?

Du beschließt, dass es wenig Sinn macht zu lange darüber nachzudenken. Du durchquerst den Raum und gelangst zur Treppe, die nach oben führt.
Gehe zu [\ref{goToLevel2Alone}].

\block{theseAreClonesAndSomeAreMissingAndIKnowWhy}{Hier fehlen zwei}

Du kannst dich düster erinnern, dass es einen Zauber gibt, mit dem man Klone erschaffen kann. Wenn einer der beiden Männer stirbt, fährt seine Seele in einen der hier schlafenden Körper.
Diese Theorie wird bestätigt, als du zwei Säulen entdeckst, die keinen Inhalt haben.
Vermutlich wurden die beiden nach dem Angriff im Wald hier wieder zum Leben erweckt.
Aber um was für einen Plan kann es sich handeln, wenn man dafür offensichtlich eine Schar von Handlangern heranzüchten muss?

Du beschließt, dass es wenig Sinn macht zu lange darüber nachzudenken. Du durchquerst den Raum und gelangst zur Treppe, die nach oben führt.
Gehe zu [\ref{goToLevel2Alone}].

\block{theseAreClonesAndSomeAreMissing}{Hier fehlt etwas}

Du kannst dich düster erinnern, dass es einen Zauber gibt, mit dem man Klone erschaffen kann. Wenn einer der beiden Männer stirbt, fährt seine Seele in einen der hier schlafenden Körper.
Du siehst deine Theorie als bestätigt an, als du zwei leere Säulen entdeckst.
Direkt vor den Säulen befindet sich eine dickliche, schleimige Flüssigkeit am Boden.
Vermutlich wurden zwei der Kreaturen hier vor kurzem wieder zum Leben erweckt. Aber um was für einen Plan kann es sich handeln, wenn man dafür offensichtlich eine Schar von Handlangern heranzüchten muss?

Du beschließt, dass es wenig Sinn macht zu lange darüber nachzudenken. Du durchquerst den Raum und gelangst zur Treppe, die nach oben führt.
Gehe zu [\ref{goToLevel2Alone}].

\block{goToLevel2Alone}{Ich steige hinauf}

Ein kurzer Blick aus dem Fenster zeigt dir, dass es inzwischen dunkel geworden ist. Noch immer scheinen die dichten Wolken den Himmel zu bedecken, denn du kannst in diesem flüchtigen Moment keine Sterne oder den Schein des Mondes erkennen.
Du gehst über die Treppenstufen weiter nach oben, bis du im nächsten Stockwerk angekommen bist.

Vor dir liegt ein langer Gang, an dessen Ende sich ein kleiner Runde Raum mit 4 Türen befindet, eine in jede Himmelsrichtung.
Wenn du die östliche Tür öffnen willst, gehe zu \goto{stockDoor}.
Wenn du die südliche Tür öffnen willst, gehe zu \goto{mastersBedroom}.
Wenn du die westliche Tür öffnen willst, gehe zu \goto{bathRoom}.
Wenn du die nördliche Tür öffnen willst, gehe zu \goto{libray}.

\block{mastersBedroom}{Ich gehe nach Süden}

Du entschließt dich dazu die südliche Tür zu öffnen.
Wenn Ereignis \getEvent{theWarlockDied} bereits eingetreten ist, gehe zu \goto{emptyMastersBedroom}. Ansonsten gehe zu \goto{dangerMastersBedroom}.

\block{dangerMastersBedroom}{Ich zögere}

Du streckst die Hand aus, um den Türknauf der südlichen Tür zu drehen, als du plötzlich ein kurzes Klappern hinter der Tür hörst. Unter dem Türschlitz leuchtet der Boden kurz grünlich auf und deine Nackenhaare stellen sich auf. Du hast plötzlich wieder dieses Gefühl, das dir im Laufe deines Lebens schon ein paar Mal die Haut gerettet hat. Wer oder was auch immer sich hinter dieser Tür verbirgt, wenn du sie jetzt öffnest, würde es vermutlich kein gutes Ende nehmen. Du hast gelernt, in solchen Momenten auf dein Bauchgefühl zu hören.

Wenn du die Tür trotzdem öffnen willst, gehe zu \goto{openDoorToWarlock}.
Wenn du lieber die östliche Tür öffnen willst, gehe zu \goto{stockDoor}.
Wenn du lieber die westliche Tür öffnen willst, gehe zu \goto{bathRoom}.
Wenn du lieber die nördliche Tür öffnen willst, gehe zu \goto{libray}.

\block{openDoorToWarlock}{Ich komme unangekündigt}

Du öffnest die Tür und siehst sofort einen alten Mann in einer dunklen Robe vor dir. Alles an ihm schreit nach ``Hexer''. Die knochigen Arme, die schwarze, lange Robe, die trotzdem nicht die jämmerliche Statur verbergen kann, die eingefallene, blasse Haut und der grünlich leuchtende Zauberstab in seiner Hand. Er hat sofort bemerkt, dass jemand eingetreten ist und sagt krächzend: ``Ich hatte euch doch befohlen...''

Als er sich umsieht, realisiert er erschrocken, dass du nicht die Person bist, mit der er gerechnet hatte. Du spürst sofort, dass er dich angreifen wird, sobald er die Chance dazu bekommt. Ein Kampf ist unvermeidbar!

\textit{Du nutzt diesen Moment der Überraschung, um zuerst anzugreifen. Du hast Vorteil auf deinen Geschicklichkeitswurf.}

\monsterWarlock{ikilledthewarlockInHisRoom}{anEvilWarlockKilledMe}

\block{ikilledthewarlockInHisRoom}{Ich sterbe hier nicht}

Der Hexer ist dem Tode geweiht, daran besteht für keinen von euch beiden ein Zweifel. Ein atemloser Schrei versucht sich aus seinem Mund zu quälen, doch er bringt nicht mehr als ein Keuchen hervor. Während du zusiehst, kehrt sich das Weiße in seinen Augen nach oben und er sinkt auf die Knie. Noch bevor sein Gesicht auf dem harten Steinboden aufschlägt, hört er auf zu zucken. Reglos bleibt er liegen. Er ist tot.

Markiere Ereignis \getEvent{theWarlockDied}.

Wenn du den Hexer durchsuchen willst, gehe zu [\ref{lootTheWarlockInHisRoom}].
Wenn du dich in dem Raum umsehen willst, gehe zu [\ref{emptyMastersBedroom}].

\block{lootTheWarlockInHisRoom}{Er braucht das nicht mehr}

Du zögerst keine Sekunde und durchsuchst den Toten nach Dingen, die dir helfen könnten.
Kurze Zeit später kannst du einen Zauberstab \getItem{warlockStaff}, einen geschwungenden Dolch \getItem{warlocksDagger}, einen kleinen Kupferschlüssel \getItem{warlocksKey} und einen goldenen Ring \getItem{goldenWarlockRing} dein Eigen nennen.

Außerdem ist dir aufgefallen, dass der linke Arm des Mannes von seltsamen Tätowierungen übersät ist, die dir wie Runen vorkommen. Viele kleine Narben und Schnitte durchbrechen die Muster, einige scheinen noch recht frisch und gerade erst verheilt zu sein.

Gehe zu [\ref{emptyMastersBedroom}].

\block{emptyMastersBedroom}{Ich sehe mich um}

Vor dir liegt ein kleiner Raum, der sowohl als Schlafgemach, als auch als Labor zu dienen scheint.
Das ausladende Bett mit hölzernem Baldachin war einst sicher ein kostbares Stück, doch es ist in die Jahre gekommen und wurde nicht gepflegt. Viele kleine Löcher weisen auf Holzwürmer hin und die Tücher sind von Motten zerfressen. Daneben steht ein langer, robuster Holztisch, auf dem dutzende Phiolen und Gläser mit Tränken in verschiedensten Farben aufgestellt sind. Eine kleine Kerze erhitzt eines der Gläser, in dem eine blaue Flüssigkeit vor sich hinblubbert.

All das ist zwar interessant, für dich aber nicht von sonderlich großer Bedeutung. Spannender sind die Papierfetzen, die auf dem Boden verstreut liegen. Du bist dir nicht sicher, was du dort siehst, denn das Gekrakel wirkt wirr und ist mit unleserlichen Runen übersäht. Doch auf fast jedem der Blätter kannst du den Schriftzug ``F.R.E-Y4'' erkennen. Auf einem der Blätter ist eine Art Tier abgebildet, mit harten Kanten. Wenn du raten müsstest, würdest du vermuten, dass dies die Abbildung eines Maschinengottes ist. Auch die Karte von Greifenheim, auf der einige Gebäude mit einem X markiert sind, lässt dich nichts Gutes vermuten.

Während du dich bückst um die Karte aufzuheben, bemerkst du eine große Truhe direkt unter dem Labortisch. Sie sieht alt und schwer aus, hat aber kein Schloss.

Wenn du versuchen willst die Truhe zu öffnen, gehe zu \goto{openTheDamnChest}. Wenn nicht, gehe zu \goto{leaveTheBedroom}.

\block{openTheDamnChest}{Ich öffne die Truhe}

Es kostet dich einiges an Mühe die schwere Truhe vorzuziehen. Behutsam öffnest du den Deckel, um nicht irgendeinen Sicherheitsmechanismus auszulösen. Doch deine Sorge ist unbegründet, die kannst die Truhe ohne Probleme öffnen. Darin findest du zwischen einigen Stofffetzen und einer toten Ratte ein kleines Säckchen \getItem{itemGemStones} voller Diamanten.

Gehe zu \goto{leaveTheBedroom}.

\block{leaveTheBedroom}{Ich verlasse den Raum}

Du verlässt das Schlafgemach und stehst wieder in dem kleinen Vorraum. Wenn du nach Osten gehen willst, gehe zu \goto{stockDoor}.
Wenn du den westlichen Raum noch nicht erkundet hast und diese Tür nutzen willst, gehe zu \goto{bathRoom}.
Wenn du den nördlichen Raum noch nicht erkundet hast und diese Tür nutzen willst, gehe zu \goto{libray}.
%Wenn du dem Gang zur Treppe folgen willst, gehe zu \goto{downToLayer1}.

\block{leaveTheLibrary}{Ich gehe zurück}

Du verlässt die Bibliothek und stehst wieder in dem kleinen Vorraum. Wenn du nach Osten gehen willst, gehe zu \goto{stockDoor}.
Wenn du den westlichen Raum noch nicht erkundet hast und diese Tür nutzen willst, gehe zu \goto{bathRoom}.
Wenn du den südlichen Raum noch nicht erkundet hast und diese Tür nutzen willst, gehe zu \goto{mastersBedroom}.
%Wenn du dem Gang zur Treppe folgen willst, gehe zu \goto{downToLayer1}.

\block{leaveTheBathroom}{Ich kehre zurück}

Du stehst wieder in dem kleinen Vorraum. Wenn du nach Osten gehen willst, gehe zu \goto{stockDoor}.
Wenn du den südlichen Raum noch nicht erkundet hast und diese Tür nutzen willst, gehe zu \goto{mastersBedroom}.
Wenn du den nördlichen Raum noch nicht erkundet hast und diese Tür nutzen willst, gehe zu \goto{libray}.
%Wenn du dem Gang zur Treppe folgen willst, gehe zu \goto{downToLayer1}.

\block{takeAnotherDoor}{Ich versuche etwas anderes}

Du verlässt den Lagerraum und stehst wieder in dem kleinen Vorraum. Wenn du zurück gehen willst, gehe zu \goto{stockDoor}.
Wenn du den südlichen Raum noch nicht erkundet hast und diese Tür nutzen willst, gehe zu \goto{mastersBedroom}.
Wenn du den nördlichen Raum noch nicht erkundet hast und diese Tür nutzen willst, gehe zu \goto{libray}.
Wenn du den westlichen Raum noch nicht erkundet hast und diese Tür nutzen willst, gehe zu \goto{bathRoom}.
%Wenn du dem Gang zur Treppe nach unten folgen willst, gehe zu \goto{downToLayer1}.

%\block{downToLayer1}{Ich kehre um}

%Nach einem kurzen Moment, in dem du überlegst, was du hier eigentlich suchst, schüttelst du den Kopf %und gehst zurück zur Treppe nach unten.
%TODO continue
% MEMO: excluded from current story because it leaves to many options

\block{stockDoor}{Ich gehe nach Osten}

Du versuchst die östliche Tür zu öffnen und kommst in einen kleinen Lagerraum, an dessen Außenwand sich eine weitere Treppe nach oben befindet. Vom Ende der Treppe strahlt dir bereits ein rötliches Licht entgegen, das wild flackert und furchterregende Schatten an die Wand wirft.
Du erkennst links von dir einige Fässer und Leinensäcke, während sich rechts ein Regal mit einer Vielzahl von vollen und leeren Gläsern und Flaschen befindet.

Wenn du der Treppe folgen willst, gehe zu \goto{goToLevel3Alone}.
Wenn du den Raum durchsuchen willst, gehe zu \goto{searchStockRoom}.
Wenn du eine der anderen Türen ausprobieren willst, gehe zu \goto{takeAnotherDoor}.

\block{searchStockRoom}{Ich durchwühle das Lager}

Du siehst dir den Inhalt des Raums genauer an. Nachdem du die Deckel von drei Fässern abgenommen hast, vermutest du, dass auch die restlichen Fässer den gleichen Inhalt haben - säuerlich riechender Wein und leicht abgestandenes Wasser. Beides ist nicht von Interesse für dich. Die Leinensäcke sind ähnlich, du findest Kartoffeln über Kartoffeln. Auch, wenn du gern etwas essen würdest, nützen sie dir im Rohzustand nicht viel.

Mehr Erfolg hast du bei den Gläsern. Bei vielen stellen sich deine Nackenhaare auf, denn du erkennst abgetrennte Finger, Augen irgendwelcher armen Kreaturen und wer weiß was noch für Scheußlichkeiten. Aber auch zwei kleinere Fläschen mit einer rötlich schimmernden Flüssigkeit. Du ziehst bei beiden den Korken und machst eine kurze Geruchs- und Geschmacksprobe, dann ist klar, dass du zwei Heiltränke gefunden hast. Du steckst beide Fläschen (\getItem{itemRegularHealingPotion1} und \getItem{itemRegularHealingPotion2}) in deine Tasche.

Wenn du der Treppe folgen willst, gehe zu \goto{goToLevel3Alone}.
Wenn du eine der anderen Türen ausprobieren willst, gehe zu \goto{takeAnotherDoor}.

\block{libray}{Ich gehe nach Norden}

Du öffnest die nördliche Tür und gibst den Weg in eine kleine Bibliothek frei. Der Raum hat die Form eines Kreisviertels und die Wände sind von hohen Regalen voller Bücher verdeckt.
Einige der Buchrücken sind alt und abgegriffen, andere sehen aus als wären sie heute erst gekauft worden. Immer wieder sind Papierfetzen und Notizen dazwischen gepresst. Auf dem Boden liegen weitere Bücher aufgeschlagen herum.
In der Mitte des Raums steht ein roter Ledersessel, auf dem eine leichte Staubschicht liegt.
Über dem Sessel schwebt wie von Geisterhand eine kleine Laterne in der Luft und wirft ein warmes Licht auf die Szenerie.

Wenn du dir den Raum genauer ansehen willst, gehe zu \goto{inspectTheLibrary}.
Wenn du dir einen der anderen Räume ansehen willst, gehe zu \goto{leaveTheLibrary}.

\block{inspectTheLibrary}{Ich interessiere mich für Bücher}

Vorsichtig betrittst du die kleine Bibliothek. Bei näherer Betrachtung sind viele der Bücher mit Spinnweben und Staub bedeckt. Die meisten Buchrücken tragen merkwürdige Runen, mit denen du nichts anfangen kannst. Es ist nicht, dass du sie nicht lesen kannst - du hast derartige Runen sogar noch nie zuvor gesehen. Du zweifelst daran, dass sie von einem Lebewesen geschrieben sein können. In gewisser Weise ähneln sie der harten Keilschrift der Zwerge, doch da ist noch mehr. Die Runen sind... zu sauber. Zu gleichmäßig. Es ist ein leichtes gleiche Runen zu finden, denn sie sind wie exakte Kopien voneinander. Selbst ihre Größe unterscheidet sich innerhalb einer Seite nicht!

Du blätterst durch einige Bücher, doch weder der Text noch die Abbildungen helfen wir weiter. Dann weckt ein Buch dein Interesse, denn es unterscheidet sich von den anderen. Schon der Buchrücken ist nicht so hart und gleichmäßig, sondern in dunkles Leder gebunden. Auf der ersten Seite erkennst du die Zeichen ``vyr freya'', die vermutlich mit einer heißen Nadel in das Leder gebrannt wurden. Interessiert schlägst du das Buch auf. Zwar kannst du auch hier die Schrift nicht lesen, doch diese Zeichen wurden eindeutig mit einer Feder geschrieben. Mehrere Seiten sind mit überaus kunstvollen Handzeichnungen von Frauen versehen. Nein, Moment, von einer einzigen Frau. Es scheint immer die selbe Person zu sein. Ein Frau mit langen, dunklen Haaren und einem bezaubernden Lächeln.

Je weiter du blätterst, desto weniger Text ist auf den Seiten und die Zeichnungen der Frau nehmen zu. In verschiedenen Posen, aus unterschiedlichen Winkeln und mit anderen Gesichtsausdrücken. Wer auch immer der Autor dieses Buches ist, war offensichtlich von dieser Frau besessen. Auf der letzten Seite findest du eine Zeichnung, die die Frau schlafend darstellt. Eine kleine fein verzierte goldene Haarnadel wurde in das Papier gesteckt. Du denkst kurz nach beschließt dann, dass du das Buch \getItem{itemFreyaBook} und die Haarnadel \getItem{itemHairNeedle} mitnimmst. Dein Gefühl sagt dir, dass du es nicht hier lassen solltest.

Du schlägst das Buch zu und steckst es ein. Dein Blick schweift wieder durch den Raum, doch du kannst nichts mehr erkennen, das von Interesse wäre.
Gehe zu \goto{leaveTheLibrary}.

\block{bathRoom}{Ich gehe nach Westen}

Du versuchst dich an der westlichen Tür. Doch die Tür rührt sich nach den ersten paar Zentimetern nicht mehr, sie klemmt. Und zwar ziemlich fest. Vielleicht könntest du sie mit etwas Gewalt dazu bewegen dir den Weg freizugeben.

Wenn du es lieber nicht versuchen willst, gehe zu \goto{leaveTheBathroom}.
Wenn du versuchen willst die Tür gewaltsam zu öffnen, gehe zu \goto{forceOpenBathroom}.

\block{forceOpenBathroom}{Ich breche durch}

Du versuchst die Tür mit Gewalt zu öffnen und wirfst dich mit der Schulter dagegen. Langsam bewegt sich die Tür und knallt beim dritten Versuch weit auf. Ein modriger Geruch steigt dir in die Nase, wie von abgestandenem Wasser. Wenig überraschend, denn vor dir siehst du eine Art Badezimmer. In der Mitte steht eine randvolle große Wanne aus weißem Stein, auf der sich eine grüne Decke aus kleinen Wasserpflanzen ausgebreitet hat. Am Stein entlang ranken sich Schlingpflanzen und die Wände sind mit Moos und mehrfarbigem Schimmel bedeckt. Aus einem goldenen Wasserhahn tropft gemächlich Wasser in die Wanne und schickt leichte Wellen durch die grünliche Suppe.

Nicht nur die klemmende Tür, auch die unbändige Vegetation und der mordige Geruch schreien danach, dass dieser Raum seit Jahren von niemandem mehr betreten wurde. Du kannst dir vorstellen, dass er einst sehr prunkvoll und ansprechend war, doch davon ist nicht mehr als eine Ahnung geblieben.

Wenn du dich hier genauer umsehen willst, gehe zu \goto{inspectBathroom}.
Wenn nicht, gehe zu \goto{nothingToDoInTheBath}.

\block{inspectBathroom}{Wenn die Tür schon offen ist, kann ich mich auch umsehen}

Vorsichtig trittst du in den Raum. Unter deinen Füßen spürst du, wie das weiche Moos nachgibt und du ein kleines Stück versinkst. Der große Spiegel an der Wand ist komplett beschlagen und lässt dich nichts erkennen. Ohne groß darüber nachzudenken wischst du mit der Handfläche darüber, um ein kleines Stück freizugeben. Als du dein Spiegelbild siehst, erschrickst du und taumelst einen Schritt zurück. Die verzerrte, von Maden zerfressene Fratze die dich im Spiegel ansieht ähnelt dir zwar und macht deine Bewegungen nach, sieht aber aus wie eine über Jahre vergessene Leiche.

Als du versuchst dir das Bild etwas genauer anzusehen, fühlt es sich an, als würde sich dein Magen umdrehen. Das sollst defintiv du sein, selbst die zerissene Kleidung ähnelt deiner. Sicherlich nur eine Illusion, doch eine finstere! Angewiedert wendest du dich ab.

Wenn du den Wasserhahn zudrehen möchtest, gehe zu \goto{closeFaucet}.
Wenn du den Raum lieber verlassen willst, gehe zu \goto{nothingLeftToDoInTheBath}.

\block{closeFaucet}{Ich halte das nicht aus}

Das beständige Tropfen des Wasserhahns fängt an dich zu nerven. Du beugst dich über die Wanne und drehst den Hahn zu. Du wartest kurz ab, ob weitere Tropfen ihren Weg finden, doch du hast den Wasserhahn zum schweigen gebracht. Zufrieden drehst du dich um, als du plötzlich hinter dir ein lautes Platschen hörst. Wasser läuft in einem Schwall über den Rand der Wanne und noch bevor du dich umdrehen kannst packen dich zwei kräftige Arme und ziehen dich nach hinten!

Du wirst in die Wanne gezogen und warmes Wasser umhüllt deinen Körper. Die Wanne muss magisch sein, denn das Innenleben ist größer, viel, viel größer - du fühlst dich wie in einem See. Du windest und wehrst dich, noch immer umklammern dich die kräftigen Arme und reißen dich unerbittlich Meter um Meter in die Tiefe!

Du windest dich und trittst nach hinten, bis dein Fuß etwas Hartes trifft. Sofort trittst du erneut zu und kannst den Griff ausreichend lockern, um dich umzudrehen. Dein Verstand braucht einen Moment, um zu verarbeiten, was du siehst. Eine nackte Frau, mit langen, schwarzen Haaren starrt dich mit leeren, weißen Augen an. Ihre Haut ist selbst in dem grünlichen Licht des dreckigen Wassers bleich und ihr Griff fühlt sich an, als würde dich ein großer Ork packen!

Wir einen Stärkerettungswurf mit DC 15. Wenn du es schaffst, gehe zu \goto{sweetEscape}.
Wenn nicht, gehe zu \goto{deathByWater}.

\block{sweetEscape}{Ich werde nicht ertrinken}

Du kämpfst um dein Leben und stößt dem Weib mehrmals mit deiner Stirn ins Gesicht, bis dunkles Blut herausläuft. Nur für den Bruchteil einer Sekunde lockert sich ihre Umarmung. Sofort schlägst du um dich, bis du frei bist, und schwimmst zur Oberfläche! Als das Weib unter dir nach deinem Fuß packt, trittst du erneut nach ihr und kannst so etwas Abstand zwischen euch bringen. Gerade als dir die Luft ausgeht erreichst du die rettende Oberfläche. Deine Hand packt den Rand der Wanne und mit einem einzigen Schwung ziehst du dich aus dem Wasser und über den Wannenrand!

Du keuchst angestrengt und spuckst mehrmals Wasser aus, dass seinen Weg in deine Lungen gefunden hat. Du hast 8 Lebenspunkte verloren. Schnell krabbelst du weg von der Wanne! Es dauert mehrere Minuten, bis du wieder ruhiger atmen kannst.
Erst jetzt fällt dir auf, dass deine Kleidung durch die Wärme schon beinahe wieder trocken ist. Angestrengt richtest du dich auf und beobachtest misstrauisch die Wanne.
Der Wasserhahn hat wieder angefangen zu tropfen, aber du wirst den Teufel tun das Ding noch einmal anzufassen!

Gehe zu \goto{nothingLeftToDoInTheBath}.

\block{deathByWater}{Ich will nicht ertrinken}

Du kämpfst um dein Leben und stößt dem Weib mehrmals mit deiner Stirn ins Gesicht, bis dunkles Blut herausläuft. Nur für den Bruchteil einer Sekunde lockert sich ihre Umarmung. Sofort schlägst du um dich, bis du frei bist, und schwimmst zur Oberfläche! Als das Weib unter dir nach deinem Fuß packt, trittst du erneut nach ihr doch du kannst sie nicht abschütteln. Mit unerbitterlicher Kraft zieht sie dich wieder in die Tiefe!

Verzweifelt greifst du nach der rettenden Oberfläche, als sie ihre kalten Lippen auf deine presst. Sofort spürst du, wie die Luft aus deinen Lungen gesaugt wird und deine Kraft schwindet.
Das grausame Weib lässt von dir ab, denn es ist hoffnungslos.
Langsam sinkst du hinab in dein feuchtes Grab.

\textbf{Ende.}

\block{nothingLeftToDoInTheBath}{Ich muss mich nicht weiter umsehen}

Du hast nicht das Bedürfnis länger als nötig in diesem Raum zu bleiben. Wenn hier so lange niemand war, wird es vermutlich ohnehin nichts wertvolles zu holen geben.

Wenn Ereignis \getEvent{theWarlockDied} bereits eingetreten ist, gehe zu \goto{leaveTheBathroom}.
Ansonsten gehe zu \goto{warlockSupriseAttack}.

\block{nothingToDoInTheBath}{Ich muss mich nicht umsehen}

Du hast nicht das Bedürfnis dich durch die blühende Fauna dieses Raums zu kämpfen. Wenn hier so lange niemand war, wird es vermutlich auch nichts wertvolles zu holen geben.

Wenn Ereignis \getEvent{theWarlockDied} bereits eingetreten ist, gehe zu \goto{leaveTheBathroom}.
Ansonsten gehe zu \goto{warlockSupriseAttack}.

\block{warlockSupriseAttack}{Ich war zu laut}

Offensichtlich ist nicht unbemerkt geblieben, dass du die Tür eingerammt hast. Die südliche Tür ist offen und ein blasser, dünner Mann in schwarzer Robe blickt dich verbissen an. Wenn du jemals jemanden gesehen hast, auf den die Beschreibung ``Hexer'' gepasst hat, dann muss es dieser alte Mann sein. Bevor du reagieren kannst, hebt er seinen Arm, in der Hand einen Zauberstab, und wirft dir mit einem Schrei ein magisches Geschoss entgegen!

\textit{Der Hexer hat dich überrascht und greift zuerst mit einem Eldritch Blast an. Du hast Nachteil auf deinen Stärkewurf.}

\monsterWarlock{ikilledthewarlockInHisRoom}{anEvilWarlockKilledMe}

\block{goToLevel3Alone}{Ich gehe weiter}

Du gehst die Treppe hinauf und befindest dich nun im obersten Zimmer des Turms. Keine weitere Treppe, keine Türen, keine Fenster. Nur ein großer, runder Raum, der ungefähr 10 Meter im Durchmesser misst. An der Außenwand sind in kurzem Abstand hunderte schwarzer Kerzen aufgereiht, deren kleine Flammen sich zuckend hin und her bewegen.
Doch die größte Lichtquelle befindet sich mitten im Raum.
Instinktiv siehst du dir an, wovon das Licht ausgeht.
Auf einem Podest, dass sich wie eine Pyramide zuspitzt, befindet sich in Brusthöhe ein faustgroßer Edelstein. Er ist in eine kleine Metallkralle gefasst, von der Metallfäden verschiedenster Farbe ausgehen. Du hast selten größere Juwelen gesehen, doch besonders das rote Licht, welches dieser Stein ausstrahlt, zieht dich in seinen Bann.

Wirf einen Charismarettungswurf mit DC 16. Wenn du es nicht schaffst, gehe zu [\ref{cannotresistthegemAlone}]. Wenn du Erfolg hast, gehe zu [\ref{resistingTheGemAllAlone}].

\block{resistingTheGemAllAlone}{Ich muss mich abwenden}

Für einen Moment bist du wie verzaubert. Als deine Augen den Edelstein erblicken, vergisst du die Welt um dich herum. Doch dann schaffst du es, deine Gedanken wegzulenken und abwärts zu sehen. Der Zwang wird schwächer und du bekommst langsam wieder einen klaren Kopf. Erst jetzt fällt dir auf, dass sich am Podest und auf dem Boden schwarze Flecken befinden. Unter den Flecken, über den gesamten Boden verteilt, befindet sich ein riesiger Runenkreis mit merkwürdigen Symbolen, die mit Sicherheit keinem guten Zweck gewidmet sind.

Als du den Blick zur Decke schweifen lässt, erkennst du eine Frau, eine Elfe, die dort in einigen Metern Höhe kopfüber aufgehangen ist. Sie scheint bewusstlos, ihr blondes Haar ist blutverschmiert und hängt nach unten. Immer wieder fallen kleine Bluttropfen von ihr hinab auf den Edelstein, zweifelsohne absichtlich.

Wenn du das Podest mit dem Edelstein genauer untersuchen willst, gehe zu [\ref{inspectTheGem}].
Wenn du die Elfe genauer betrachten willst, gehe zu [\ref{inspectTheElf}].
Wenn du den Raum lieber wieder verlassen willst, gehe zu [\ref{downToLayer2}].

\block{inspectingThePodest}{Ich sehe mir das ganz genau an}

Wenn es sich wirklich um eine Maschine handelt, dann kannst du vielleicht mit viel Glück... ja, da ist ein kleiner Schalter. Als du mit dem Finger über die kleine Platte fährst, gibt diese leicht nach. Du suchst noch etwas weiter, kannst aber keine anderen Schalter finden. Du könntest natürlich auch an einem der bunten Metalldrähte ziehen.

Wenn du den Schalter drücken willst, gehe zu \goto{pushTheButtonAtTheGem}.
Wenn du dich an den Drähten versuchen willst, gehe zu \goto{secretMechanic}.
Wenn du lieber nichts machst und den Raum verlassen willst, gehe zu [\ref{downToLayer2}].

\block{secretMechanic}{Ich weiß, was ich tue}

Um das Podest herum sind verschiedene Metalldrähte zu erkennen, die vom Boden zum Podest verlaufen. Du siehst zwei rote, einen schwarzen, vier graue und einen gelben Draht.

Wenn du einen roten Draht herausreißen willst, gehe zu \goto{redHeringCable}.
Wenn du am schwarzen Draht ziehen willst, gehe zu \goto{pullingBlackCable}.
Wenn du einen der grauen Drähte auswählen willst, gehe zu \goto{pullinGreyCable}.
Wenn du dein Glück mit dem gelben Draht versuchen willst, gehe zu \goto{pullingYellowCable}.
Wenn du doch lieber den Schalter drücken willst, gehe zu \goto{pushTheButtonAtTheGem}.
Wenn du lieber nichts machst und den Raum verlassen willst, gehe zu [\ref{downToLayer2}].

\block{pushTheButtonAtTheGem}{Ich drücke zu}

Du siehst nicht viel Sinn darin an den Drähten herumzuspielen. Am Ende richtest du noch größeren Schaden an, als du dir vorstellen kannst. Aber der kleine Schalter scheint dir ein vertretbares Risiko zu sein. Wenn es einen Schalter gibt, wird er dafür gedacht sein, dass man ihn benutzt.

Behutsam legst du deinen Finger auf die kleine Metallplatte und atmest tief ein und aus. Dann drückst du zu. Sofort hörst du ein leichtes Surren, das den Raum erfüllt. Dann beginnt das Podest langsam im Boden zu versinken. Bevor du dich aber genauer damit befassen kannst, hörst du plötzlich ein lautes Knacken über dir. Als du hochsiehst, siehst du, dass dir die gefesselte Elfe entgegenfällt.

Wirf eine Athletikprobe mit DC 14. Wenn du Erfolg hast, gehe zu \goto{catchingTheElf}. Wenn nicht, gehe zu \goto{notCatchingTheElf}.

\block{redHeringCable}{Ich sehe rot}

Du zögerst kurz und greifst dann nach einem der roten Drähte. Es erfordert mehr Kraft, als du gedacht hättest, doch als du kräftig an dem Draht ziehst, reißt er aus der Verbindung am Podest. Erwartungsvoll hälst du die Luft an.

Es passiert nichts. Sekunden vergehen und du bist dir nicht sicher, ob das etwas bewirkt hat. Du weißt auch nicht genau, was du erwartet hast. Vielleicht einfach... \textit{irgendetwas}. Doch es passiert nichts. Also greifst du zum zweiten roten Draht und reißt auch diesen aus der Halterung, wieder ohne erkennbaren Effekt.

Wenn du jetzt am schwarzen Draht ziehen willst, gehe zu \goto{pullingBlackCable}.
Wenn du einen der grauen Drähte herausreißen willst, gehe zu \goto{pullinGreyCable}.
Wenn du dich am gelben Draht versuchen willst, gehe zu \goto{pullingYellowCable}.
Wenn du jetzt den Schalter drücken willst, gehe zu \goto{pushTheButtonAtTheGem}.
Wenn du es dabei belassen und den Raum verlassen willst, gehe zu [\ref{downToLayer2}].

\block{pullingBlackCable}{Ich sehe schwarz}

Du ziehst kräftig am schwarzen Draht. Noch während du das Ende des Drahts ansiehst, nachdem du es aus seiner Halterung gelöst hast, spürst du ein Zittern im Boden. Innerhalb eines Herzschlags wandert dein Blick nach unten und du siehst, wie dir einer der riesigen Steine geradewegs entgegen kommt! Benommen fällts du rückwärts, doch du landest nicht. Unter dir tut sich die Pforte zur Hölle auf, eine sengende Flammenwand schießt nach oben und verwandelt alles in ihrem Weg zu Asche. Auch dich.

\textbf{Ende.}

\block{pullinGreyCable}{Ich sehe grau}

Wahllos nimmst du einen der grauen Drähte und ziehst an ihm, sodass er sich aus der Halterung am Podest löst. Noch während du das Ende des Drahts ansiehst, spürst du ein Zittern im Boden. Kein Zweifel, der ganze Turm bebt!

Du verschwendest keinen Moment und rennst sofort zur Treppe. Wirf einen Geschicklichkeitsrettungswurf mit DC 16. Wenn du Erfolg hast, gehe zu \goto{barelyEscaping}. Ansonsten gehe zu \goto{deathByStones}.

\block{pullingYellowCable}{Ich sehe gelb}

Wie heißt es im Volksmund? ``Alle guten Dinge sind gelb''... oder so ähnlich? Du greifst nach dem gelben Draht und ziehst kräftig, sodass er aus der Halterung am Podest gerissen wird. Ein kurzes Zittern geht durch das Podest, dann siehst du, wie sich die Metallkralle um den Edelstein langsam zurückzieht. Der Stein verliert seinen Halt und fällt zu Boden. Du willst instinktiv zugreifen und ihn auffangen, hälst dann aber doch kurz inne. Ein lautes Klirren durchdringt den Raum, als der Edelstein auf dem harten Steinboden auftrifft und mehrmals auf und ab springt. Als er endlich ruhig liegen bleibt, siehst du, wie das rötliche Leuchten langsam verschwindet und der Steine eine weißliche, trübe Färbung annimmt.

Wenn du den Edelstein vorsichtig aufheben willst, gehe zu \goto{pickUpTheGem}.
Wenn du den Schalter drücken willst, gehe zu \goto{pushTheButtonWithoutGem}.
Wenn du es dabei belassen und den Raum verlassen willst, gehe zu [\ref{downToLayer2}].

\block{pickUpTheGem}{Ich hebe ihn auf}

Du überlegst kurz, ob es eine gute Idee ist den Stein aufzuheben. Allerdings hast du nicht mehr dieses unbändige Verlangen ihn anzufassen, wenn du ihn betrachtest. Und ohne das Leuchten wirkt er auch nicht mehr so bedrohlich. Vorsichtig tippst du den Stein mit dem Finger an und zuckst sofort zurück.

Nachdem eine Weile nichts passiert, atmest du tief ein und greifst nach dem Stein. Es passiert nichts und du steckst da gute Stück ein \getItem{itemGiantGemstone}.

Wenn du den Schalter drücken willst, gehe zu \goto{pushTheButtonWithoutGem}.
Wenn du es dabei belassen und den Raum verlassen willst, gehe zu [\ref{downToLayer2}].

\block{pushTheButtonWithoutGem}{Ich drücke endlich drauf}

Du siehst nicht viel, was du hier noch tun könntest. Da ist nur noch der Schalter am Podest. Du bist darauf gefasst, dass gleich etwas passiert, als sich dein Finger dem Schalter nähert. Du siehst dich noch einmal im Raum um, dann drückst du den Schalter nach Innen.

Du wartest einen Moment. Doch nichts passiert. Also drückst du erneut. Und nochmal. Und nochmal. Doch das Ergebnis bleibt das gleiche, deine Aktion hat keinen für dich erkennbaren Effekt. Du siehst noch einmal hoch zur Elfe, die noch immer an der Decke aufgehangen ist. Dann wieder zum Podest. Ein wenig enttäuscht machst du dich auf den Weg um den Raum zu verlassen. Gehe zu [\ref{downToLayer2}].

\block{deathByStones}{Ich renne um mein Leben}

Du rast die Treppe hinab und durch den kleinen Lagerraum. Um dich herum fallen Steine aus den Wänden, die Decke und der Boden scheinen sich in ihre Einzelteile aufzulösen und du hörst ein ohrenbetäubendes Dröhnen. Im Gang angekommen hast du die nächste Treppe im Blick, als dir plötzlich ein großer Stein vor den Fuß fällt und dich ins Stolpern bringt. Verzweifelt kommst du wieder auf die Beine, doch da trifft dich der nächste Stein, diesmal auf die Schulter. Und noch einer. Dir bleibt keine Zeit, dich zu erholen.

Es dauert einige Minuten, bis sich die gigantische Staubwolke legt. Vom einst bedrohlichen schwarzen Turm ist nur noch ein riesiger Trümmerhaufen geblieben. Für dich gibt es keine Hilfe mehr. Dein lebloser Körper liegt unter mehreren Metern Gestein begraben.

\textbf{Ende.}

\block{barelyEscaping}{Ich fliehe sofort}

Du rast die Treppe hinab und durch den kleinen Lagerraum. Um dich herum fallen Steine aus den Wänden, die Decke und der Boden scheinen sich in ihre Einzelteile aufzulösen und du hörst ein ohrenbetäubendes Dröhnen. Im Gang angekommen hast du die nächste Treppe im Blick, als dir plötzlich ein großer Stein vor den Fuß fällt. Im letzten Moment kannst du ausweichen und rennst weiter. Bei der nächsten Treppe gehst du mehr Risiko ein und nimmst mehrere Stufen auf einmal. Mit Erfolg, du kommst in den Raum mit den leuchtenden Säulen. Von diesen ist nicht mehr viel geblieben, überall liegen große Glassplitter und die reglosen, schleimbedeckten Körper von nackten Männern auf dem Boden in einer riesigen Lache aus der seltsamen Flüssigkeit.

Du siehst durch den Raum und begreifst sofort, dass du den Turm nicht lebend verlassen wirst. Plötzlich gerät das offene Fenster wieder in dein Blickfeld. Du siehst nicht einmal nach, ob der Sprung sicher wäre, du springst einfach. Dir bleibt keine Wahl, jede noch so kleine Chance ist besser als der sichere Tod!

Du landest in einem Dornenbusch und spürst den heftigen Aufprall in jedem Knochen. Wirf 3D6 und zieh dir das Ergebnis von den Lebenspunkten ab. Wenn du den Aufprall überlebst, gehe zu \goto{survivedTheJump}. Andernfalls gehe zu \goto{killedByTheJump}.

\block{killedByTheJump}{Ich bin heftig gestürzt}

Du bist dir nicht sicher, was es am Ende war. Welcher Knochen gebrochen ist und dein Schicksal besiegelt hat. Während der Turm hinter dir in einer riesigen Staubwolke versinkt, die dich komplett umhüllt, kriechst du voller Schmerzen über den Boden und versuchst zu entkommen. Einige Minuten kämpfst du noch qualvoll mit deinem Schicksal, bis deine Kräfte schwinden und du erschöpft die Augen schließt.

\textbf{Ende.}

\block{survivedTheJump}{Ich bin tief gefallen}

Du bist dir sicher, dass du dir einige Knochen gebrochen hast, als du auf dem Boden aufkommst. Mit letzter Kraft zerrst du dich vom Turm weg und kriechst so weit du kommst, als dich die riesige Staubwolke des Zusammenbruchs einhüllt. Es dauert eine Weile, bis sich der Staub legt und der Krach abebbt. Vom vormals furchteinflößenden schwarzen Turm ist nichts als ein gigantiger Schutthaufen geblieben.

Du bist noch nicht außer Gefahr. Hier draußen bist du leichte Beute für Wölfe, besonders in deinem angeschlagenen Zustand. Verzweifelt versuchst du zum nächsten Baum zu kommen, doch dich verlassen vorher die Kräfte. Ohnmächtig sackst du zusammen.

Gehe zu \goto{rescuedByTortle}.

\block{rescuedByTortle}{Ich wurde gerettet}

Als du wieder zu dir kommst, starrst du eine Zimmerdecke aus Holz an. Es braucht einen Moment, bis du dir deiner Situation bewusst wirst. Du liegst in einem weichen Bett in einem Zimmer im... ja, im ``Zum Nimmerleer'' in Greifenheim! Deine verletzten Glieder wurden versorgt, du bist in dicke Bandagen gehüllt. Als du den Kopf nach rechts drehst und leicht stöhnst, hörst du plötzlich eine helle Stimme: ``Bei den Göttern, was für ein Glück, ihr seid wach!''. Du siehst den Sohn des Wirts, der gerade durch die Tür kam und dich anstrahlt. ``Vater war sehr besorgt, ihr habt ganze drei Tage geschlafen! Wir fürchteten schon, dass ihr nicht mehr aufwacht.''

Nach einem kurzen Gespräch bringst du in Erfahrung, dass ein anderer Abenteurer, eine große Schildkröte, dich gefunden und nach Greifenheim gebracht hat. Er hat auch für deine Versorgung und das Zimmer bezahlt für eine ganze Woche bezahlt. Leider weiß niemand, wie dein mysteriöser Retter heißt, denn nachdem er dich ablieferte verschwand er sogleich wieder. Scheinbar ist es wie der junge Wirtssohn gesagt hat, die Götter waren dir gnädig gestimmt. Du hast dieses Abenteuer überlebt und kannst die Geschichte weitererzählen.

\textbf{Ende.}

\block{notCatchingTheElf}{Ich reagiere nicht schnell genug}

Du fängst die Elfe in letzter Sekunde, allerdings mit deinem vollen Körper. Du verlierst 2 Lebenspunkte, als sie aus mehreren Metern Höhe auf dich fällt. Nachdem du dich davon erholt hast, prüfst du die Atmung der Elfe. Endlich eine gute Nachricht, sie lebt, ist aber bewusstlos. Eilig löst du die Fesseln und entdeckst dabei ein dicke Nadel, die mittig in ihrer Brust steckt. Vorsichtig entfernst du das Werkzeug \getItem{itemBloodyNeedle} und drückst auf die kleine Wunde. Sie ist tief, aber nicht besonders breit, sodass du dir keine großen Sorgen machst. Es ist offensichtlich, dass das Ziel war die Elfe so langsam wie möglich ausbluten zu lassen.

Wenn \getEvent{theWarlockDied} bereits eingereten ist, gehe zu \goto{noIntermissionByWarlock}. Wenn keines der Ereignisse eingetreten ist, gehe zu \goto{warlockFoundMe}.

\block{catchingTheElf}{Ich reagiere sofort}

Du kannst gerade noch die Arme ausstrecken und die Elfe abfangen, bevor sie auf dem Boden aufschlägt. Zum Glück ist sie relativ leicht, ansonsten wäre dir dieses Kunststück vermutlich nicht ohne weiteres gelungen. Eilig prüfst du die Atmung der Elfe. Endlich eine gute Nachricht, sie lebt, ist aber bewusstlos. Du löst ihre Fesseln und entdeckst dabei ein dicke Nadel, die mittig in ihrer Brust steckt und aus der frisches Blut tropft. Vorsichtig entfernst du das Werkzeug \getItem{itemBloodyNeedle} und drückst auf die kleine Wunde. Sie ist tief, aber nicht besonders breit, sodass du dir keine großen Sorgen machst. Es ist offensichtlich, dass das Ziel war die Elfe so langsam wie möglich ausbluten zu lassen.

Wenn \getEvent{theWarlockDied} bereits eingereten ist, gehe zu \goto{noIntermissionByWarlock}.
Wenn nicht, gehe zu \goto{warlockFoundMe}.

\block{noIntermissionByWarlock}{Wir sollten gehen}

Du siehst dich im Raum um. Hier gibt es nichts mehr, was du noch tun könntest.
Gehe zu \goto{downToLayer2WithElf}, um die Turmspitze mit der Elfe zu verlassen.

\block{warlockFoundMe}{Ich wurde entdeckt}

Bevor du weiter darüber nachdenken kannst, was du jetzt tun solltest, hörst du plötzlich einen entsetzten Schrei von der Treppe. Du erkennst einen dürren, alten Mann in einer langen, dunklen Robe, der dich wütend anschreit: ``WIE KANNST DU ES WAGEN?''. Das hagere, eingefallene Gesicht ist von tiefen Augenringen gekennzeichnet, die bei der blassen Haut besonders auffallen. Auch die weite Robe kann nicht verbergen, welch knochiger Körper sich unter ihr befindet. Du hast selten einen Mann gesehen, auf den die Beschreibung ``Hexer'' so gut gepasst hat.

Es ist offensichtlich, dass der Hexer nicht einverstanden ist mit dem was du getan hast. Mit wackligen Schritten rennt er auf dich zu und hebt dabei seinen Arm. In seiner Hand hält er einen kleinen Holzstab, dessen Spitze grün zu leuchten beginnt. Doch du lässt dich nicht einfach überraschen, du bist längst aufgestanden und zum Kampf bereit!

\monsterWarlock{ikilledthewarlockInTheHighTowerAfterHeSuprisedMe}{anEvilWarlockKilledMe}

\block{ikilledthewarlockInTheHighTowerAfterHeSuprisedMe}{Ich lasse mich nicht überraschen}

Der Hexer ist dem Tode geweiht, daran besteht für keinen von euch beiden ein Zweifel. Ein atemloser Schrei versucht sich aus seinem Mund zu quälen, doch er bringt nicht mehr als ein Keuchen hervor. Während du zusiehst, kehrt sich das Weiße in seinen Augen nach oben und er sinkt auf die Knie. Noch bevor sein Gesicht auf dem harten Steinboden aufschlägt, hört er auf zu zucken. Reglos bleibt er liegen. Er ist tot.

Markiere Ereignis \getEvent{theWarlockDied}.

Wenn du den Hexer durchsuchen willst, gehe zu [\ref{lootTheWarlockInTheTowerAfterTheSurpise}].
Wenn du den Raum mit der Elfe verlassen willst, gehe zu [\ref{downToLayer2WithElf}].

\block{lootTheWarlockInTheTowerAfterTheSurpise}{Was sein war soll mir gehören}

Du zögerst keine Sekunde und durchsuchst den Toten nach Dingen, die dir helfen könnten.
Kurze Zeit später kannst du einen Zauberstab \getItem{warlockStaff}, einen geschwungenden Dolch \getItem{warlocksDagger}, einen kleinen Kupferschlüssel \getItem{warlocksKey} und einen goldenen Ring \getItem{goldenWarlockRing} dein Eigen nennen.

Außerdem ist dir aufgefallen, dass der linke Arm des Mannes von seltsamen Tätowierungen übersät ist, die dir wie Runen vorkommen. Du erkennst die gleichen Muster auf dem Boden wieder, offensichtlich besteht ein Zusammenhang. Viele kleine Narben und Schnitte durchbrechen die Muster, einige scheinen noch recht frisch und gerade erst verheilt zu sein.

Gehe zu \goto{downToLayer2WithElf}, um den Raum mit der Elfe zu verlassen.

\block{downToLayer2WithElf}{Wir sollten gehen}

Vorsichtig hebst du die Elfe über deine Schulter, sodass du sie einigermaßen tragen kannst. Du kannst so zwar nicht rennen, aber du kannst halbwegs sicher laufen. Du trägst jetzt die Verantwortung für euch beide, wortwörtlich. Markiere das Ereignis \getEvent{iGotTheElf}.

Du gehst zurück zur Treppe und bist kurz darauf ein Stockwerk tiefer. Vor dir liegt der kleine Lagerraum mit der Tür auf der anderen Seite. Rechts von dir befinden sich einige Fässer und braune Leinensäcke. Auf der linken Seite steht ein größeres Regal, in dem sich dutzende Gläser befinden, manche befüllt, andere leer und verstaubt. Da du die Elfe auf der Schulter hast, hast du aber wenig Spielraum um den Raum zu erkunden. Du gehst zur gegenüber liegenden Tür.

Gehe zu \goto{leavingTheStockroomAfterDescentFast}.
%Wenn Ereignis \getEvent{intoTheTowerAlone} eingetreten ist, gehe zu \goto{leavingTheStockroomAfterDescentFast}.
%Ansonsten gehe zu \goto{leavingTheStockroomAfterDescent}.

\block{downToLayer2}{Es geht die Treppe hinab}

Du gehst zurück zur Treppe und bist kurz darauf ein Stockwerk tiefer.

Wenn Ereignis \getEvent{intoTheTowerAlone} eingetreten ist, gehe zu \goto{inTheStockroomAfterDescentFast}.
Ansonsten kannst gehe zu \goto{inTheStockroomAfterDescent}.

\block{inTheStockroomAfterDescentFast}{Ich wieder im Lager}

Du befindest dich wieder in dem kleinen Lagerraum, den du auf dem Weg in das obere Stockwerk durchqueren musstest. Ohne lange zu zögern machst du ein paar große Schritte durch den Raum zur gegenüberliegenden Tür. Die Fässer und Säcke sind für dich nicht interessant.

Gehe zu \goto{leavingTheStockroomAfterDescentFast}.

\block{inTheStockroomAfterDescent}{Ich bin in einem Lager}

Vor dir liegt der kleine Lagerraum mit der Tür auf der anderen Seite. Rechts von dir befinden sich einige Fässer und braune Leinensäcke. Auf der linken Seite steht ein größeres Regal, in dem sich dutzende Gläser befinden, manche befüllt, andere leer und verstaubt.

%Wenn du zur Tür gehen willst, gehe zu \goto{leavingTheStockroomAfterDescent}. %currently to many options. This path is considered to give the option to search the other rooms, we skip this for the time being
Wenn du zur Tür gehen willst, gehe zu \goto{leavingTheStockroomAfterDescentFast}.
Wenn du dich im Raum umsehen willst, gehe zu \goto{searchStockRoomAfterDescent}.

\block{searchStockRoomAfterDescent}{Ich durchsuche den Raum}

Du nimmst dir einen Moment Zeit und siehst dir den Raum genauer an. Nachdem du die Deckel von drei Fässern abgenommen hast, vermutest du, dass auch die restlichen Fässer den gleichen Inhalt haben - säuerlich riechender Wein und leicht abgestandenes Wasser. Beides ist nicht von Interesse für dich. Die Leinensäcke sind ähnlich, du findest Kartoffeln über Kartoffeln. Auch, wenn du gern etwas essen würdest, nützen sie dir im Rohzustand nicht viel.

Mehr Erfolg hast du bei den Gläsern. Bei vielen stellen sich deine Nackenhaare auf, denn du erkennst abgetrennte Finger, Augen irgendwelcher armen Kreaturen und wer weiß was noch für Scheußlichkeiten. Aber auch zwei kleinere Fläschen mit einer rötlich schimmernden Flüssigkeit. Du ziehst bei beiden den Korken und machst eine kurze Geruchs- und Geschmacksprobe, dann ist klar, dass du zwei Heiltränke gefunden hast. Du steckst beide Fläschen (\getItem{itemRegularHealingPotion1} und \getItem{itemRegularHealingPotion2}) in deine Tasche. Danach gehst du zur Tür.

%Gehe zu \goto{leavingTheStockroomAfterDescent}. %currently to many options. This path is considered to give the option to search the other rooms, we skip this for the time being
Gehe zu \goto{leavingTheStockroomAfterDescentFast}.

\block{leavingTheStockroomAfterDescentFast}{Ich kenne den Weg}

Du bist wieder im kleinen Vorraum mit den vier Türen und dem Gang. Da dein Bauchgefühl dir sagt, dass du nicht mehr Zeit als nötig an diesem Ort verbringen solltest, gehst du direkt den Gang entlang und die anschließende Treppe hinab.

Am Fuß der Treppe siehst du durch das Fenster kurz nach draußen. Inzwischen ist finstere Nacht und du kannst nur mit Mühe den Boden am Fuß des Turms erkennen. Nein, bei dieser Dunkelheit ist der Weg durch das Fenster keine Option. Du wendest dich ab und läufst durch den Raum, vorbei an den schimmernden Säulen, die ein Gefühl des Unbehagens in dir auslösen. Glücklicherweise lässt sich die nächste Tür von dieser Seite ganz einfach öffnen. Wieder führt dich dein Weg über deine Treppe und wenig später bist du im Treppenzimmer des Erdgeschosses.

Wenige Schritte vor dir führt eine weitere Treppe tiefer hinab, doch du wendest dich der Tür zu und öffnest sie. Vor dir liegt die Eingangshalle des Turms. Direkt gegenüber, knapp 10 Meter entfernt, befindet sich das große Eingangstor. Wenn Ereignis \getEvent{SpiderAttacks} bereits eingetreten ist, gehe zu [\ref{iKnowAboutTheSpiderAfterDescent}]. Wenn das Ereignis \getEvent{ikilledthedamnspiderMarker} bereits eingetreten ist, gehe zu [\ref{iKnowAboutTheSpiderBecauseIKilledItAfterDescent}].
Wenn beides nicht zutrifft, gehe zu [\ref{iHaveNoIdeaAboutTheSpiderAfterDescent}].

\block{iHaveNoIdeaAboutTheSpiderAfterDescent}{Ich eile zum Tor}

Mit eiligen Schritten durchquerst du den Raum, um zum Tor zu gelangen. Plötzlich hörst du über dir ein lautes Knacken und das Rasseln einer Metallkette.
Das Geräusch lässt dich nach oben blicken. Dein Magen zieht sich zusammen als du erkennst, dass dort gar kein Kronleuchter hängt sondern eine \textbf{riesige Wolfspinne}, die sich gerade zu dir herablässt! Ein Kampf ist unvermeidbar...

\monsterSpider{ikilledthedamnspiderAfterDescent}{poisonedAndDying}{theSpiderKilledMe}

\block{iKnowAboutTheSpiderAfterDescent}{Mein Spinnensinn warnt mich}

Natürlich hast du die riesige Spinne noch nicht vergessen. Doch du kannst kein Zeichen des riesigen Monsters an der spinnwebenverhangenen Decke erkennen. Vorsichtig versuchst du am Rand des Raums entlang zu schleichen, um einer zweiten Konfrontation aus dem Weg zu gehen. Wirf eine Schleichenprobe mit DC 16 (oder DC 19, wenn Ereignis \getEvent{iGotTheElf} eingetreten ist). Wenn du erfolgreich bist, gehe zu [\ref{stealthToTheMainDoorAfterDescent}].
Wenn du es nicht schaffst, gehe zu [\ref{theSpiderKnowsWhereIAmAfterDescent}].

\block{iKnowAboutTheSpiderBecauseIKilledItAfterDescent}{Alle Beine zum Himmel}

Vor dir liegt der monströse Kadaver der Wolfsspinne, deren Leben du so grausam beendet hast. Allerdings ist das keine Situation für Mitleid, sie hätte dich mit Sicherheit ohne Zögern gefressen, wenn du ihr die Chance gegeben hättest. Zumindest musst du dir jetzt keine Gedanken mehr um das Untier machen.

Du gehst ungehindert zum geschlossenen Tor und ziehst mit aller Kraft. Zentimeter um Zentimeter öffnet sich der Weg in die Freiheit. Als der Spalt groß genug ist, um bequem hindurch zu schlüpfen, verlässt du den Turm.

Gehe zu [\ref{longRoadHomeAfterDescent}].

\block{ikilledthedamnspiderAfterDescent}{Ab jetzt spinnenfreie Zone}

Du lieferst dir den Kampf deines Lebens mit der Spinne. Du weichst Biss um Biss der Kreatur aus, doch langsam gewinnt die Spinne die Oberhand. Plötzlich erkennst du einen unachtsamen Moment, den du sofort ausnutzt. Du triffst die Spinne schwer und sie taumelt wie benommen zurück. Die haarigen Beine zittern, als sie versucht vor dir zurückzuweichen, doch ihre Kraft reicht nicht mehr aus um die Decke emporzusteigen. Mit einem letzten Kreischen sackt der schwere Leib auf den Boden und die Beine werden angezogen. Dann Ruhe und Regungslosigkeit. Du hast die Spinne getötet.

Wenn Ereignis \getEvent{iGotTheElf} eingetreten ist, gehe zu \goto{leavingTheTowerNowWithElf}. Ansonsten gehe zu \goto{leavingTheTowerNow}

\block{leavingTheTowerNow}{Ich sollte endlich verschwinden}

Du wartest nicht, um zu sehen, was der Turm noch für dich bereit hält. Du gehst zum geschlossenen Tor und ziehst mit aller Kraft. Zentimeter um Zentimeter öffnet sich der Weg in die Freiheit. Als der Spalt groß genug ist, um bequem hindurch zu schlüpfen, verlässt du den Turm.

Gehe zu [\ref{longRoadHomeAfterDescent}].

\block{leavingTheTowerNowWithElf}{Wir sollten endlich verschwinden}

Du wartest nicht, um zu sehen, was der Turm noch für dich bereit hält. Du gehst zum geschlossenen Tor und ziehst mit aller Kraft. Zentimeter um Zentimeter öffnet sich der Weg in die Freiheit, bis die Tür weit genug geöffnet ist. Du nimmst die Elfe wieder auf deine Schulter und trägst sie aus dem Turm in die Freiheit.

Gehe zu [\ref{longRoadHomeAfterDescent}].

\block{stealthToTheMainDoorAfterDescent}{Ich berühre den Boden nichteinmal}

Behutsam bewegst du dich Zentimeter um Zentimeter zur Tür, die Decke des Raums immer im Blick. Dabei übersiehst du fast einen kleinen Kieselstein, kannst aber gerade noch rechtzeitig anhalten. Nervös wandert dein Blick zu den Spinnennetzen, doch du kannst keinerlei Bewegung erkennen.
Du setzt deinen Weg fort und bist wenig später am Tor angekommen.

Um die Tür zu öffnen wirst du kräftig ziehen müssen. Dir ist klar, dass das unweigerlich Geräusche machen wird.
Wenn Ereignis \getEvent{iGotTheElf} eingetreten ist, gehe zu \goto{openDoorWithElf}. Ansonsten
wirf eine Athletikprobe mit DC 14. Wenn du Erfolg hast, gehe zu [\ref{escapeTroughTheMainDoor}]. Schaffst du es nicht, gehe zu [\ref{noEscapeFromSpiderEvenWithStealth}].

\block{openDoorWithElf}{Ich laufe nicht mehr weg}

Du erkennst sofort, dass es wenig Sinn macht die Tür mit Gewalt aufzuzerren. Da du die Elfe bei dir hast, wirst du niemals rechtzeitig die Tür geöffnet haben und mit ihr entkommen, bevor sich die Spinne auf dich stürzt. Nein, das wäre wie ein Geschenk für diese Bestie und du hast nicht vor jetzt aufzugeben!

Vorsichtig lässt du die Elfe von deiner Schulter auf den Boden sinken und machst dich kampfbereit. Es ist schwer abzuschätzen, wo sich die Bestie in diesem riesigen Netz befindet, doch du versuchst dein Glück und schleichst einige Meter durch den Raum, bis zur nächsten Wand. Du hebst einen der kleinen Steine auf und wirfst ihn gegen die Tür. Sofort danach presst du dich dicht an die Wand!

Du hörst ein Rascheln im Netz und kurz darauf trifft der schwere Leib des Untiers auf dem harten Boden auf. Dein Plan geht auf, der Hinterleib des Untiers ist dir zugewandt und du hast den ersten Angriff!

\textit{Du hast Vorteil auf deinen Geschicklichkeitswurf.}

\monsterSpider{ikilledthedamnspiderAfterDescent}{poisonedAndDying}{theSpiderKilledMe}

\block{theSpiderKnowsWhereIAmAfterDescent}{Erschütternde Nachrichten}

Behutsam bewegst du dich Zentimeter um Zentimeter zur Tür, die Decke des Raums immer im Blick. Dabei übersiehst du leider einen kleinen Kieselstein, auf den du trittst. Das entstehende Schleifgeräusch ist alles andere als laut. Trotzdem sackt dir für einen Moment das Herz zusammen, als du Bewegung im Spinnennetz erkennst. Du wurdest entdeckt!

Wenn Ereignis \getEvent{iGotTheElf} eingetreten ist, gehe zu \goto{haveToFightSpiderBecauseElf}. Ansonsten gehe zu [\ref{sprintToTheMainDoor}], wenn du zum Tor rennen willst, oder zu [\ref{comeAtMeSpiderling}], wenn du dich dem Kampf stellen willst!

\block{haveToFightSpiderBecauseElf}{Nur über deinen Kadaver}

Du erkennst sofort, dass es mit der Elfe keinen Sinn macht zur Tür zu rennen und dem Biest den Rücken zuzuwenden. Nein, das wäre wie ein Geschenk für diese Bestie und du hast nicht vor jetzt aufzugeben!

Schnell lässt du die Elfe von deiner Schulter auf den Boden sinken und machst dich kampfbereit. In diesem Augenblick trifft der schwere Leib des Untiers auf dem harten Boden auf. Mit erhobenen Greifarmen stürmt es kreischend auf dich zu!

\monsterSpider{ikilledthedamnspiderAfterDescent}{poisonedAndDying}{theSpiderKilledMe}

\block{longRoadHomeAfterDescent}{Ich rieche frische Luft}

Dein Herz macht einen Freudensprung, als dir die feuchte Nachtluft ins Gesicht schlägt. Du bist wieder in Freiheit! Es besteht kein Zweifel, das war ein echtes Abenteuer und du warst mehr als einmal kurz davor dein Leben zu verlieren. Nachdem du ein paar Meter gelaufen bist, siehst du dich noch einmal um. Noch immer ragt der Turm bedrohlich in den Himmel, doch du wirst ihn sicher nicht mehr betreten.

Wenn Ereignis \getEvent{iGotTheElf} eingetreten ist, gehe zu \goto{theEndWithElf}. Ansonsten gehe gehe zu [\ref{theEndAlone}].

\block{theEndAlone}{Heimweg nach Greifenheim}

Vor dir liegt ein langer und beschwerlicher Fußmarsch zurück nach Greifenheim. Doch angetrieben vom Gedanken an ein gemütliches Bett und eine warme Mahlzeit machst du dich auf den Weg. Du hast dir Ruhe und Erholung verdient. Und sicherlich wirst du mit der Geschichte deines Abenteuers das ein oder andere Freigetränk in den Gasthäusern erwerben können. Ganz zu schweigen davon, was du vielleicht noch bei den Händlern verdienen könntest, wenn du deine Beute verkaufst. Doch das sind Aufgaben für einen anderen Tag. Du gehst nach Hause.

\textbf{Ende.}

\block{theEndWithElf}{Ich habe es geschafft}

Erschöpft schleppst du dich zu einem nahen Baum und lehnst die Elfe behutsam gegen den Stamm. Du setzt dich neben sie und atmest kurz tief durch, bevor du sie beobachtest. Ihr Brustkorb hebt und senkt sich und du hast das Gefühl, dass ihre Wangen wieder etwas rosiger aussehen. Dem ersten Eindruck nach scheint es ihr gut zu gehen und du seufzst erleichtert.

Etwas erschöpft lehnst du dich auch gegen den Baumstamm. Du hast ein ganzes schönes Abenteuer hinter dir. Auch wenn du keine zweihundert Meter vom Turm entfernt bist, hast du das seltsame Gefühl, dass du hier sicher bist. Du versuchst noch einen Moment wach zu bleiben und kämpfst gegen den Schlaf an, doch keine zwei Minuten später sinken deine Augenlieder schwer nach unten.
Gehe zu \goto{awakenOutsideTower}.
