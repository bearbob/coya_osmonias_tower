\chapter*{Der Maschinengott}
%im obersten Zimmer des Turms verbirgt sich ein Maschinengott, der dem Hexer Befehle erteilt
%die maschine ist eine magische kriegsmaschine und benötigt seelenenergie um wieder zu funktionieren


\block{theUpperLevelWithWarlock}{Die oberen Stockwerke}

Irgendwann kommt ihr an einer weiteren Tür an, vor der ihr stehen bleibt. Dein neuer Meister wühlt in seinem Umhang herum und zückt schließlich einen kleinen Schlüssel, mit dem er die Tür öffnet. Du hörst, wie sich ein schwerer Riegel hinter dem Holz bewegt.

Der nachfolgende Raum ist kreisrund und muss ungefähr so groß sein wie die Eingangshalle. Allerdings ist er nicht leer, sondern gefüllt mit knapp einem Dutzend gläsernern Säulen. Im Inneren dieser Säulen wabert eine vioelette Flüssigkeit, die den Raum in ein bedrohliches Licht taucht. Als der Hexer den Raum betritt, flammen an den Wänden Fackeln auf und erhellen die Szenerie. Du kannst nun deutlich erkennen, dass sich in der violetten Flüssigkeit humanoide Formen befinden, die in ihr zu schweben scheinen.

Wenn Ereignis \getEvent{ihaveseentheirfaces} eingetreten ist, gehe zu [\ref{iKnowTheseFaces}]. Ansonsten gehe zu [\ref{weproceedThroughTheRoom}].

\block{weproceedThroughTheRoom}{Durch den Raum}

Entschlossen geht der Hexer durch den Raum und du folgst ihm. Als du einigen der Säulen näher kommst, erkennst du, dass die humanoiden Formen, die du zuvor gesehen hast, Männer mittleren Alters mit breiten Schultern sind. Es ist schwer mehr Details zu erkennen, aber du glaubst, dass die Männer sich alle sehr stark ähneln. Beinahe als wären sie Brüder. Allerdings glaubst du nicht, dass du eine große Familie vor dir hast. Vielmehr muss hier Magie im Spiel sein.

Auf der anderen Seitedes Raums befindet sich ein Fenster und daneben eine weitere Treppe, die höher führt. Ihr folgt dieser Treppe.
Gehe zu [\ref{togetherToLevel2}].

\block{iKnowTheseFaces}{Ich kenne diese Gesichter}

Entschlossen geht der Hexer durch den Raum und du folgst ihm. Als du einigen der Säulen näher kommst, erkennst, dass die humanoiden Formen die du zuvor gesehen hast Männer mittleren Alters mit breiten Schultern sind. Es ist schwer mehr Details zu erkennen, aber du bist dir völlig sicher: es sind immer wieder die gleichen zwei Gesichter. Die Gesichter deiner Entführer, Torlof und sein Kumpane.

Wirf eine Arkanaprobe mit DC 14. Wenn du es nicht schaffst, gehe zu [\ref{arcanaFailedMe}]. Wenn du Erfolg hast, und das Ereignis \getEvent{oneKidnapperDiedA}, \getEvent{oneKidnapperDiedB} oder \getEvent{bothKidnappersDied} eingetreten ist, gehe zu [\ref{theseAreClonesAndOneIsMissing}]. Wenn keins dieser Ereignisse eingetreten ist, gehe zu [\ref{theseAreClones}].

\block{arcanaFailedMe}{Ich kann das nicht zuordnen}

Du kannst nur vermuten, dass es einen dunklen Zauber oder eine ähnliche Hexerei gibt, mit der das alles zu tun hat. Aber was für Pläne hat der Hexer, wenn er sich offensichtlich eine Schar von Handlangern heranzüchtet?

Während du noch darüber nachdenkst, erreicht ihr die andere Seite des Raums. Hier befindet sich ein Fenster und eine weitere Treppe, die höher führt. Ihr folgt dieser Treppe.
Gehe zu [\ref{togetherToLevel2}].

\block{theseAreClones}{Ich habe davon gehört}

Du kannst dich düster erinnern, dass es einen Zauber gibt, mit dem man Klone erschaffen kann. Wenn einer der beiden Entführer stirbt, fährt seine Seele in einen der schlafenden Körper. Aber was für Pläne hat der Hexer, wenn er nicht nur einmal, sondern so oft auf den Tod seiner Handlanger vorbereitet sein will?

Während du noch darüber nachdenkst, erreicht ihr die andere Seite des Raums. Hier befindet sich ein Fenster und eine weitere Treppe, die höher führt. Ihr folgt dieser Treppe.
Gehe zu [\ref{togetherToLevel2}].

\block{theseAreClonesAndOneIsMissing}{Mir fehlt etwas}

Du kannst dich düster erinnern, dass es einen Zauber gibt, mit dem man Klone erschaffen kann. Wenn einer der beiden Entführer stirbt, fährt seine Seele in einen der schlafenden Körper. Diese Theorie wird bestätigt, als du eine Säule entdeckst, die keinen Inhalt hat. Vermutlich wurde einer der beiden nach dem Angriff im Wald wieder zum Leben erweckt. Aber was für Pläne hat der Hexer, wenn er nicht nur einmal, sondern so oft auf den Tod seiner Handlanger vorbereitet sein will?

Während du noch darüber nachdenkst, erreicht ihr die andere Seite des Raums. Hier befindet sich ein Fenster und eine weitere Treppe, die höher führt. Ihr folgt dieser Treppe.
Gehe zu [\ref{togetherToLevel2}].

\block{stairsToTheUpperLevels}{Die Treppe nach oben}

Du folgst den Treppenstufen nach oben. Wenig später kommst du zu einer kleinen Tür, die deinen Weg blockiert. Ein kurzer Versuch am Türgriff zeigt, dass die Tür verschlossen ist.

Wenn du \getItem{warlocksKey} besitzt, kannst du zu [\ref{unlockTheDoorWithWarlocksKey}] gehen. Ansonsten musst du umkehren. Gehe zu [\ref{goToCrossroom}].

\block{togetherToLevel2}{Höher hinauf}

Als ihr am Fenster vorbei kommt, siehst du, dass es inzwischen dunkel geworden ist. Noch immer scheinen die dichten Wolken den Himmel zu bedecken, denn du kannst in diesem flüchtigen Moment keine Sterne oder den Schein des Mondes erkennen.

Euer Weg führt euch die Treppe hinauf in die nächste Etage des Turms. Es geht einen langen Gang entlang, an dessen Ende sich ein kleiner Runde Raum mit 4 Türen befindet, eine in jede Himmelsrichtung. Der Hexer greift nach dem Türgriff der östlichen Tür und öffnet sie. Ihr durchquert ein kleines Lager, du siehst einige Fässer, Säcke und gefüllte Gläser. Am Ende wartet eine weitere Treppe darauf erklommen zu werden. Vom Ende der Treppe strahlt euch bereits ein rötliches Licht entgegen, das wild flackert und furchterregende Schatten an die Wand wirft.

Ihr befindet euch nun im obersten Zimmer des Turms. Keine weitere Treppe, keine Türen, keine Fenster. Nur ein großer, runder Raum, der ungefähr 10 Meter im Durchmesser misst. An der Außenwand sind in kurzem Abstand hunderte schwarzer Kerzen aufgereiht, deren kleine Flammen sich zuckend hin und her bewegen.
Doch die größte Lichtquelle befindet sich mitten im Raum. Auf einem Podest, dass sich wie eine Pyramide zuspitzt, befindet sich in Brusthöhe ein faustgroßer Edelstein. Er ist in eine kleine Metallkralle gefasst, von der Metallfäden verschiedenster Farbe ausgehen. Du hast selten größere Juwelen gesehen, doch besonders das rote Licht, welches dieser Stein ausstrahlt, zieht dich in seinen Bann.

Wirf einen Charismarettungswurf mit DC 16 oder mit DC 15, wenn du \getItem{itemBurialRing} besitzt. Wenn du es nicht schaffst, gehe zu [\ref{cannotresistthegem}]. Wenn du Erfolg hast, gehe zu [\ref{resistingTheGem}].

\block{resistingTheGem}{Ich kann mich abwenden}

Für einen Moment bist du wie verzaubert. Als deine Augen den Edelstein erblicken, vergisst du die Welt um dich herum. Doch dann schaffst du es, deine Gedanken wegzulenken und abwärts zu sehen. Erst jetzt fällt dir auf, dass sich am Podest und auf dem Boden schwarze Flecken befinden. Unter den Flecken, über den gesamten Boden verteilt, befindet sich ein riesiger Runenkreis mit merkwürdigen Symbolen, die mit Sicherheit keinem guten Zweck gewidmet sind.

Als du den Blick zur Decke schweifen lässt, erkennst du eine Frau, eine Elfe, die dort in einigen Metern Höhe kopfüber aufgehangen ist. Sie scheint bewusstlos, ihr blondes Haar ist blutverschmiert und hängt nach unten. Immer wieder fallen kleine Bluttropfen von ihr hinab auf den Edelstein, zweifelsohne absichtlich.

Die krächzende Stimme des Hexers reißt dich aus deinen Gedanken. ``Nur keine Scheu, fass den Stein an.'', sagt er mit einem bösartigen Unterton.
Wenn Ereignis \getEvent{deceivedTheWarlock} eingetreten ist, gehe zu [\ref{idonthavetotouchthegem}]. Wenn nicht, gehe zu [\ref{hewantsmetotouchthegem}].

\block{idonthavetotouchthegem}{Die Scharade hat ein Ende}

Es reicht, das Schauspiel endet jetzt! Ruckartig drehst du dich um und gehst auf den Hexer los. Der Ausdruck der Überraschung in seinen Augen weicht schnell dem Schock, als er realisiert, dass er keine Macht über dich hat!

\monsterWarlock

Wenn du den Kampf gewinnst, gehe zu [\ref{ikilledthewarlockInTheHighTower}].
Wenn du verlierst, gehe zu [\ref{anEvilWarlockKilledMe}].

\block{ikilledthewarlockInTheHighTower}{Nur einer von uns wird leben}

Der Hexer ist dem Tode geweiht, daran besteht für keinen von euch beiden ein Zweifel. Ein atemloser Schrei versucht sich aus seinem Mund zu quälen, doch er bringt nicht mehr als ein Keuchen hervor. Während du zusiehst kehrt sich das Weiße in seinen Augen nach oben und er sinkt auf die Knie. Noch bevor sein Gesicht auf dem harten Steinboden aufschlägt, hört er auf zu zucken. Reglos bleibt er liegen. Er ist tot.

Markiere Ereignis \getEvent{theWarlockDied}.

Wenn du den Hexer durchsuchen willst, gehe zu [\ref{lootTheWarlockInTheTower}].
Wenn du das Podest mit dem Edelstein untersuchen willst, gehe zu [\ref{inspectTheGem}].
Wenn du die Elfe genauer betrachten willst, gehe zu [\ref{inspectTheElf}].
Wenn du den Raum verlassen willst, gehe zu [\ref{downToLayer2}].

\block{lootTheWarlockInTheTower}{Was sein war ist mein}

Du zögerst keine Sekunde und durchsuchst den Toten nach Dingen, die dir helfen könnten.
Kurze Zeit später kannst du einen Zauberstab \getItem{warlockStaff}, einen geschwungenden Dolch \getItem{warlocksDagger}, einen kleinen Kupferschlüssel \getItem{warlocksKey} und einen goldenen Ring \getItem{goldenWarlockRing} dein Eigen nennen.

Außerdem ist dir aufgefallen, dass der linke Arm des Mannes von seltsamen Tätowierungen übersät ist, die dir wie Runen vorkommen. Du erkennst die gleichen Muster auf dem Boden um den Edelstein wieder, offensichtlich besteht ein Zusammenhang. Viele kleine Narben und Schnitte durchbrechen die Muster, einige scheinen noch recht frisch und gerade erst verheilt zu sein.

Wenn du das Podest mit dem Edelstein untersuchen willst, gehe zu \goto{inspectTheGem}.
Wenn du die Elfe genauer betrachten willst, gehe zu \goto{inspectTheElf}.
Wenn du den Raum verlassen willst, gehe zu \goto{downToLayer2}.

\block{inspectTheElf}{Ich sehe mir die Elfe genauer an}

Du betrachtest die Elfe über die genauer. Sie ist mit deinem dicken Seil gefesselt und mit den Füßen nach oben an der Decke aufgehangen. Ihre sonst feinen Gesichtszüge wirken wie von Schmerzen verzerrt und immer wieder tropft Blut von ihr herab auf den Edelstein.
Du kannst nicht erkennen, wo sie verwundet ist, die Teile ihres blauen Seidengewands, die du durch die Fesseln sehen kannst, wirken sauber und unbeschadet.

Zwischen euch liegen mehr als zwei Körperlängen, sodass du keine Chance hast, sie zu erreichen. Wenn du dir das Podest genauer ansehen willst, gehe zu \goto{inspectTheGem}.
Wenn du den Raum verlassen willst, gehe zu \goto{downToLayer2}.

\block{inspectTheGem}{Ich sehe mir den Edelstein genauer an}

Du widmest deine Aufmerksamkeit dem Edelstein. Als du ihn ansiehst, spürst du sofort wieder das Verlangen ihn zu berühren. Schnell wendest du deine Augen nach unten ab. Dein Blick fällt auf das Podest. Erst jetzt bemerkst du, dass es aus Metall besteht und scheinbar nicht auf dem Boden steht, sondern \textit{durch} den Boden geht. Knapp über dem Boden siehst du eine kleine Platte, auf der die Zeichen ``F.R.E-Y4'' eingraviert sind.

Wirf eine Weisheitsprobe mit DC 12. Wenn du es schaffst, gehe zu \goto{iHaveMachineKnowledge}.
Wenn nicht, gehe zu \goto{skippingMachineKnowledge}.

\block{skippingMachineKnowledge}{Ich kann das nicht zuordnen}

Du kannst mit dieser Gravur nicht viel anfangen. Dir ist klar, dass das Gebilde vor dir nicht nur ein einfacher Podest ist, aber wer kann schon sagen, welchem Zweck es wirklich dient. Tatsache ist, dass das Leuchten des Edelsteins noch immer ein mulmiges Gefühl in deiner Magengegend auslöst. Vielleicht ist es besser zu gehen.

Wenn du versuchen willst dir den Edelstein zu nehmen, gehe zu \goto{grabbingTheGem}.
Wenn du lieber die Finger von der ganzen Sache lassen und den Raum verlassen willst, gehe zu \goto{downToLayer2}.

\block{iHaveMachineKnowledge}{Das kommt mir bekannt vor}

Du kennst diese Art von Gravur. Du hast eine ähnliche Platte am Bauch des Nimmerleer in Greifenheim gesehen! In deinem Kopf fügen sich gerade einige Puzzelteile zusammen - du stehst auf einem Maschinengott! Mit Sicherheit gehört dieses Podest bereits zu dem Konstrukt. Du kannst zwar nicht sagen, was der Hexer vorhatte, aber mit Sicherheit war es nichts Gutes. Und du bist dir genauso sicher, dass dieser Maschinengott keine guten Neuigkeiten bringen wird.

Wenn du den Podest genauer untersuchen willst, gehe zu \goto{inspectingThePodest}.
Wenn du versuchen willst dir den Edelstein zu nehmen, gehe zu \goto{grabbingTheGem}.
Wenn du lieber die Finger von der ganzen Sache lassen und den Raum verlassen willst, gehe zu \goto{downToLayer2}.

\block{grabbingTheGem}{Ich nehme den Edelstein}

Du hast zwar kein gutes Gefühl dabei, aber du willst auch sicher gehen, dass niemand das Werk des Hexers fortführen kann. Und dieser Edelstein scheint eine Schlüsselrolle zu spielen. Eilig streckst du die Hand aus und packst den Stein. Sofort umgibt dich Stille. Alles ist schwarz.

Gehe zu \goto{deathStranding}.

\block{hewantsmetotouchthegem}{Ich soll es berühren}

Du stehst noch immer unter dem Zauber des Hexers, der deinen Körper kontrolliert. Gegen deinen Willen bewegen sich deine Füße in Richtung des Edelsteins! Du hörst den Hexer noch sagen: ``Nur keine Scheu, erfülle deinen Zweck!''

Nenn es Todesahnung, nenn es schlechtes Bauchgefühl, doch dein Geist sträubt sich vehement gegen den Befehl! Wirf einen Charismarettungswurf mit DC 12 oder mit DC 11, wenn du \getItem{itemBurialRing} besitzt. Wenn du erfolgreich bist, gehe zu [\ref{noiwillnottouchthisgem}]. Wenn du versagst, gehe zu [\ref{deathbyGem}].

\block{noiwillnottouchthisgem}{Ich werde das nicht tun}

Du bist dir sicher, dass es dein Todesurteil ist, wenn du den Stein berührst. Magische Kontrolle oder nicht, du wirst dein Leben nicht ohne einen Kampf opfern! Während sich deine Füße wie von allein weiter bewegen, tobt in dir ein Gefecht mit der dunklen Magie, die dich beherrscht. Schlussendlich kommst du wieder zu Sinnen und hast die volle Kontrolle zurück.

Du bist frei. Und das keinen Moment zu spät, als du nach unten siehst, hast du schon die Hand ausgestreckt um den Edelstein zu berühren. Doch dazu wird es nicht kommen.

Gehe zu [\ref{idonthavetotouchthegem}]

\block{cannotresistthegemAlone}{Ich sollte es berühren}

Du bist wie verzaubert. Als deine Augen den Edelstein erblicken, vergisst du alles um dich herum. Du achtest gar nicht darauf, was sich noch im Raum befindet.
In deinen Gedanken ist nur dieser riesige, wunderschöne Edelstein.
Du musst ihn haben, du sollst ihn haben! Das Schicksal hat dich hier her geführt, damit du dir diesen Stein nimmst. Dein ganzes Leben lief auf diesen Moment zu. Und gleich gehört er dir. Du musst nur noch zupacken. Ja, du musst den Edelstein berühren. Dann hast du es geschafft. Gehe zu [\ref{deathbyGem}].

\block{cannotresistthegem}{Ich will es berühren}

Du bist wie verzaubert. Als deine Augen den Edelstein erblicken, vergisst du alles um dich herum. Du achtest gar nicht darauf, was sich noch im Raum befindet. Unterbewusst bemerkst du, dass der Hexer mit dir spricht, doch es ist wie ein unverständliches Rauschen in deinen Ohren. Und du willst ihm nicht zuhören. Du musst ihm nicht zuhören. Nein, du musst gar nichts. Nur zum Edelstein, ja. Du musst den Edelstein berühren. Gehe zu [\ref{deathbyGem}].

\block{deathbyGem}{Ich habe es berührt}

Wie in Trance kommst du dem Stein näher, Schritt um Schritt, bis deine Finger ihn berühren können. Irgendetwas in dir wehrt sich, ist unruhig. Doch niemand wird dich jetzt noch aufhalten, nicht einmal du selbst! Du streckst die Hand aus und berührst den Stein. Sofort umgibt dich Stille. Die Unruhe hört auf. Alles ist schwarz.

Gehe zu \goto{deathStranding}.

\block{deathStranding}{Ich bin verloren}

Du weißt nicht, wie viel Zeit vergeht. Dein Körper ist wie betäubt, du spürst nichts, siehst nichts und hörst nichts. Dann wird es plötzlich wärmer. Erst jetzt bemerkst du, wie kalt dir war. Doch auch die Kälte ist bald vergessen, als die Wärme nicht mehr abnimmt und sich zu Hitze entwickelt. Und schon bald wird aus der Hitze ein schmerzvolles Brennen. Du kannst das Gefühl nicht beschreiben, doch es will einfach nicht aufhören. Wenn du sterben könntest, wärst du inzwischen gestorben. So viel Schmerz kann keine Seele ertragen. Doch irgendetwas hält dich auf! Du kannst nicht in die nächste Welt, der Weg ist versperrt!

Du weißt nicht, wie viel Zeit vergeht. Tage? Jahre? Jahrhunderte? Deine Welt besteht nur noch aus Verzweiflung und Schmerz, der jeden klaren Gedanken verhindert.

\textbf{Ende.}

\block{reachingTheWindow}{Wenn sich eine Tür schließt, öffnet sich ein Fenster}

Du erreichst mit schmerzenden Fingern den Fenstersims und kannst dich endlich wieder richtig festhalten. Dein Kletterpfad war vermutlich nicht der einfachste, aber er hat dich ans Ziel geführt. Über die beiden Momente, in denen du fast den Halt verloren und metertief gefallen wärst, denkst du nicht weiter nach.

Du atmest tief ein und ziehst dich nach oben. Vor dir liegt ein großer, kreisrunder Raum, der von einem violetten Schimmer erhellt wird. Knapp ein Dutzend gläserner Säulen sind die Quellen dieses Lichts. In ihnen wabert eine dunkle Flüssigkeit, in der du jeweils ein menschliche Form erkennen kannst. Irgendetwas... oder irgendjemand steckt in diesen Säulen.

Auf deiner rechten Seite befindet sich eine Treppe, die entlang der Außenwand in ein höheres Stockwerk führt. Auf der dir gegenüber liegenden Seite des Raums kannst du eine kleine Tür erkennen.

Wenn du der Treppe nach oben folgen willst, gehe zu [\ref{goToLevel2Alone}].
Wenn du die Säulen genauer betrachten willst, gehe zu [\ref{inspectTheVioletPillars}].
Wenn du zur Tür gegen willst, gehe zu [\ref{theLockedDoorFromTheOtherSide}].

\block{theLockedDoorFromTheOtherSide}{Ich sehe mir die Tür an}

Du gehst zielorientiert durch den Raum und gelangst zu der Tür. Ein kurzer Griff am Türknauf zeigt dir, dass sie verschlossen ist. Da das gute Stück sehr stabil aussieht, wirst du hier ohne passendes Werkzeug oder den richtigen Schlüssel keine Chance haben.

Wenn du der Treppe nach oben folgen willst, gehe zu [\ref{goToLevel2Alone}].
Wenn du die Säulen genauer betrachten willst, gehe zu [\ref{inspectTheVioletPillars}].

\block{unlockTheDoorWithWarlocksKey}{Jede Tür hat einen Schlüssel}

Wie es das Schicksal will, scheint der Schlüssel, den du dem Hexer genommen hast, genau in das Schloss zu passen. Weder ein Zufall, noch sonderlich überraschend. Du öffnest die Tür.

Vor dir liegt ein großer, kreisrunder Raum, der von einem violetten Schimmer erhellt wird. Knapp ein Dutzend gläserner Säulen sind die Quellen dieses Lichts. In ihnen wabert eine dunkle Flüssigkeit, in der du jeweils ein menschliche Form erkennen kannst. Irgendetwas... oder irgendjemand steckt in diesen Säulen.

Auf der dir gegenüberliegenden Seite, vorbei an diesen Säulen, befindet sich ein Fenster. Links neben dem Fenster kannst du Treppenstufe erkennen, die entlang der Außenwand in ein höheres Stockwerk führen.

Wenn du der Treppe nach oben folgen willst, gehe zu [\ref{goToLevel2Alone}].
Wenn du die Säulen genauer betrachten willst, gehe zu [\ref{inspectTheVioletPillars}].

\block{inspectTheVioletPillars}{Ich untersuche die Säulen}

Als du einigen der Säulen näher kommst, erkennst, dass die humanoiden Formen, die du zuvor gesehen hast, Männer mittleren Alters mit breiten Schultern sind. Es ist schwer mehr Details zu erkennen, aber du bist dir völlig sicher: es sind immer wieder die gleichen zwei Gesichter. Die Gesichter deiner Entführer!

Wirf eine Arkanaprobe mit DC 14. Wenn du es nicht schaffst, gehe zu [\ref{arcanaFailedMeAndImAlone}]. Wenn du Erfolg hast, gehe zu [\ref{theseAreClonesAndSomeAreMissing}].

\block{arcanaFailedMeAndImAlone}{Ich kann damit nichts anfangen}

Du kannst nur vermuten, dass es einen dunklen Zauber oder eine ähnliche Hexerei gibt, mit der das alles zu tun hat. Aber um was für einen Plan kann es sich handeln, wenn man dafür offensichtlich eine Schar von Handlangern heranzüchten muss?

Du beschließt, dass es wenig Sinn macht zu lange darüber nachzudenken. Du durchquerst den Raum und gelangst zur Treppe, die nach oben führt.
Gehe zu [\ref{goToLevel2Alone}].

\block{theseAreClonesAndSomeAreMissing}{Hier fehlt etwas}

Du kannst dich düster erinnern, dass es einen Zauber gibt, mit dem man Klone erschaffen kann. Wenn einer der beiden Entführer stirbt, fährt seine Seele in einen der hier schlafenden Körper. Diese Theorie wird bestätigt, als du zwei Säulen entdeckst, die keinen Inhalt haben. Vermutlich wurden die beiden nach dem Angriff im Wald hier wieder zum Leben erweckt. Aber um was für einen Plan kann es sich handeln, wenn man dafür offensichtlich eine Schar von Handlangern heranzüchten muss?

Du beschließt, dass es wenig Sinn macht zu lange darüber nachzudenken. Du durchquerst den Raum und gelangst zur Treppe, die nach oben führt.
Gehe zu [\ref{goToLevel2Alone}].

\block{goToLevel2Alone}{Ich steige hinauf}

Ein kurzer Blick aus dem Fenster zeigt dir, dass es inzwischen dunkel geworden ist. Noch immer scheinen die dichten Wolken den Himmel zu bedecken, denn du kannst in diesem flüchtigen Moment keine Sterne oder den Schein des Mondes erkennen.
Du gehst über die Treppenstufen weiter nach oben, bis du im nächsten Stockwerk angekommen bist.

Vor dir liegt ein langer Gang, an dessen Ende sich ein kleiner Runde Raum mit 4 Türen befindet, eine in jede Himmelsrichtung.
Wenn du die östliche Tür öffnen willst, gehe zu \goto{stockDoor}.
Wenn du die südliche Tür öffnen willst, gehe zu \goto{mastersBedroom}.
Wenn du die westliche Tür öffnen willst, gehe zu \goto{bathRoom}.
Wenn du die nördliche Tür öffnen willst, gehe zu \goto{libray}.

\block{mastersBedroom}{Ich gehe nach Süden}

Du entschließt dich dazu die südliche Tür zu öffnen.
Wenn Ereignis \getEvent{theWarlockDied} bereits eingetreten ist, gehe zu \goto{emptyMastersBedroom}. Ansonsten gehe zu \goto{dangerMastersBedroom}.

\block{dangerMastersBedroom}{Ich zögere}

Du streckst die Hand aus, um den Türknauf der südlichen Tür zu drehen, als du plötzlich ein kurzes Klappern hinter der Tür hörst. Unter dem Türschlitz leuchtet der Boden kurz grünlich auf und deine Nackenhaare stellen sich auf. Du hast plötzlich wieder dieses Gefühl, das dir im Laufe deines Lebens schon ein paar Mal die Haut gerettet hat. Wer oder was auch immer sich hinter dieser Tür verbirgt, wenn du sie jetzt öffnest, würde es vermutlich kein gutes Ende nehmen. Du hast gelernt, in solchen Momenten auf dein Bauchgefühl zu hören.

Wenn du die Tür trotzdem öffnen willst, gehe zu \goto{openDoorToWarlock}.
Wenn du lieber die östliche Tür öffnen willst, gehe zu \goto{stockDoor}.
Wenn du lieber die westliche Tür öffnen willst, gehe zu \goto{bathRoom}.
Wenn du lieber die nördliche Tür öffnen willst, gehe zu \goto{libray}.

\block{openDoorToWarlock}{Ich komme unangekündigt}

Du öffnest die Tür und siehst sofort den alten Mann in der dunklen Robe vor dir. Alles an ihm schreit nach ``Hexer''. Die knochigen Arme, die schwarze, lange Robe, die trotzdem nicht die jämmerliche Statur verbergen kann, die eingefallene, blasse Haut und der grünlich leuchtende Zauberstab in seiner Hand. Er hat sofort bemerkt, dass du eingetreten bist und sagt krächzend: ``Ich hatte euch doch befohlen...''

Als er sich umsieht, realisiert er erschrocken, dass du nicht die Person bist, mit der er gerechnet hatte. Du spürst sofort, dass ein Kampf unvermeidbar ist und nutzt diese Überraschung, um zuerst anzugreifen!

\monsterWarlock

Wenn du den Kampf gewinnst, gehe zu [\ref{ikilledthewarlockInHisRoom}].
Wenn du verlierst, gehe zu [\ref{anEvilWarlockKilledMe}].

\block{ikilledthewarlockInHisRoom}{Ich sterbe hier nicht}

Der Hexer ist dem Tode geweiht, daran besteht für keinen von euch beiden ein Zweifel. Ein atemloser Schrei versucht sich aus seinem Mund zu quälen, doch er bringt nicht mehr als ein Keuchen hervor. Während du zusiehst, kehrt sich das Weiße in seinen Augen nach oben und er sinkt auf die Knie. Noch bevor sein Gesicht auf dem harten Steinboden aufschlägt, hört er auf zu zucken. Reglos bleibt er liegen. Er ist tot.

Markiere Ereignis \getEvent{theWarlockDied}.

Wenn du den Hexer durchsuchen willst, gehe zu [\ref{lootTheWarlockInHisRoom}].
Wenn du dich in dem Raum umsehen willst, gehe zu [\ref{emptyMastersBedroom}].

\block{lootTheWarlockInHisRoom}{Er braucht das nicht mehr}

Du zögerst keine Sekunde und durchsuchst den Toten nach Dingen, die dir helfen könnten.
Kurze Zeit später kannst du einen Zauberstab \getItem{warlockStaff}, einen geschwungenden Dolch \getItem{warlocksDagger}, einen kleinen Kupferschlüssel \getItem{warlocksKey} und einen goldenen Ring \getItem{goldenWarlockRing} dein Eigen nennen.

Außerdem ist dir aufgefallen, dass der linke Arm des Mannes von seltsamen Tätowierungen übersät ist, die dir wie Runen vorkommen. Viele kleine Narben und Schnitte durchbrechen die Muster, einige scheinen noch recht frisch und gerade erst verheilt zu sein.

Gehe zu [\ref{emptyMastersBedroom}].

\block{emptyMastersBedroom}{Ich sehe mich um}

Vor dir liegt ein kleiner Raum, der sowohl als Schlafgemach, als auch als Labor zu dienen scheint.
Das ausladende Bett mit hölzernem Baldachin war einst sicher ein kostbares Stück, doch es ist in die Jahre gekommen und wurde nicht gepflegt. Viele kleine Löcher weisen auf Holzwürmer hin und die Tücher sind von Motten zerfressen. Daneben steht ein langer, robuster Holztisch, auf dem dutzende Phiolen und Gläser mit Tränken in verschiedensten Farben aufgestellt sind. Eine kleine Kerze erhitzt eines der Gläser, in dem eine blaue Flüssigkeit vor sich hinblubbert.

All das ist zwar interessant, für dich aber nicht von sonderlich großer Bedeutung. Spannender sind die Papierfetzen, die auf dem Boden verstreut liegen. Du bist dir nicht sicher, was du dort siehst, denn das Gekrakel wirkt wirr und ist mit unleserlichen Runen übersäht. Doch auf fast jedem der Blätter kannst du den Schriftzug ``F.R.E-Y4'' erkennen. Auf einem der Blätter ist eine Art Tier abgebildet, mit harten Kanten. Wenn du raten müsstest, würdest du vermuten, dass dies die Abbildung eines Maschinengottes ist. Auch die Karte von Greifenheim, auf der einige Gebäude mit einem X markiert sind, lässt dich nichts Gutes vermuten.

Während du dich bückst um die Karte aufzuheben, bemerkst du eine große Truhe direkt unter dem Labortisch. Sie sieht alt und schwer aus, hat aber kein Schloss.

Wenn du versuchen willst die Truhe zu öffnen, gehe zu \goto{openTheDamnChest}. Wenn nicht, gehe zu \goto{leaveTheBedroom}.

\block{openTheDamnChest}{Ich öffne die Truhe}

Es kostet dich einiges an Mühe die schwere Truhe vorzuziehen. Behutsam öffnest du den Deckel, um nicht irgendeinen Sicherheitsmechanismus auszulösen. Doch deine Sorge ist unbegründet, die kannst die Truhe ohne Probleme öffnen.
%TODO
Gehe zu \goto{leaveTheBedroom}.

\block{stockDoor}{Ich gehe nach Osten}

Du versuchst die östliche Tür zu öffnen und kommst in einen kleinen Lagerraum, an dessen Außenwand sich eine weitere Treppe nach oben befindet. Vom Ende der Treppe strahlt dir bereits ein rötliches Licht entgegen, das wild flackert und furchterregende Schatten an die Wand wirft.
Du erkennst links von dir einige Fässer und Leinensäcke, während sich rechts ein Regal mit einer Vielzahl von vollen und leeren Gläsern und Flaschen befindet.

Wenn du der Treppe folgen willst, gehe zu \goto{goToLevel3Alone}.
Wenn du den Raum durchsuchen willst, gehe zu \goto{searchStockRoom}.
Wenn du eine der anderen Türen ausprobieren willst, gehe zu \goto{takeAnotherDoor}.

\block{goToLevel3Alone}{Ich gehe weiter}

Du gehst die Treppe hinauf und befindest dich nun im obersten Zimmer des Turms. Keine weitere Treppe, keine Türen, keine Fenster. Nur ein großer, runder Raum, der ungefähr 10 Meter im Durchmesser misst. An der Außenwand sind in kurzem Abstand hunderte schwarzer Kerzen aufgereiht, deren kleine Flammen sich zuckend hin und her bewegen.
Doch die größte Lichtquelle befindet sich mitten im Raum. Auf einem Podest, dass sich wie eine Pyramide zuspitzt, befindet sich in Brusthöhe ein faustgroßer Edelstein. Er ist in eine kleine Metallkralle gefasst, von der Metallfäden verschiedenster Farbe ausgehen. Du hast selten größere Juwelen gesehen, doch besonders das rote Licht, welches dieser Stein ausstrahlt, zieht dich in seinen Bann.

Wirf einen Charismarettungswurf mit DC 16 oder mit DC 15, wenn du \getItem{itemBurialRing} besitzt. Wenn du es nicht schaffst, gehe zu [\ref{cannotresistthegemAlone}]. Wenn du Erfolg hast, gehe zu [\ref{resistingTheGemAllAlone}].

\block{resistingTheGemAllAlone}{Ich muss mich abwenden}

Für einen Moment bist du wie verzaubert. Als deine Augen den Edelstein erblicken, vergisst du die Welt um dich herum. Doch dann schaffst du es, deine Gedanken wegzulenken und abwärts zu sehen. Der Zwang wird schwächer und du bekommst langsam wieder einen klaren Kopf. Erst jetzt fällt dir auf, dass sich am Podest und auf dem Boden schwarze Flecken befinden. Unter den Flecken, über den gesamten Boden verteilt, befindet sich ein riesiger Runenkreis mit merkwürdigen Symbolen, die mit Sicherheit keinem guten Zweck gewidmet sind.

Als du den Blick zur Decke schweifen lässt, erkennst du eine Frau, eine Elfe, die dort in einigen Metern Höhe kopfüber aufgehangen ist. Sie scheint bewusstlos, ihr blondes Haar ist blutverschmiert und hängt nach unten. Immer wieder fallen kleine Bluttropfen von ihr hinab auf den Edelstein, zweifelsohne absichtlich.

Wenn du das Podest mit dem Edelstein genauer untersuchen willst, gehe zu [\ref{inspectTheGem}].
Wenn du die Elfe genauer betrachten willst, gehe zu [\ref{inspectTheElf}].
Wenn du den Raum lieber wieder verlassen willst, gehe zu [\ref{downToLayer2}].

\block{inspectingThePodest}{Ich sehe mir das ganz genau an}

Wenn es sich wirklich um eine Maschine handelt, dann kannst du vielleicht mit viel Glück... ja, da ist ein kleiner Schalter. Als du mit dem Finger über die kleine Platte fährst, gibt diese leicht nach. Du suchst noch etwas weiter, kannst aber keine anderen Schalter finden. Du könntest natürlich auch an einem der bunten Metalldrähte ziehen.

Wenn du den Schalter drücken willst, gehe zu \goto{pushTheButtonAtTheGem}.
Wenn du dich an den Drähten versuchen willst, gehe zu \goto{secretMechanic}.
Wenn du lieber nichts machst und den Raum verlassen willst, gehe zu [\ref{downToLayer2}].

\block{secretMechanic}{Ich weiß, was ich tue}

Um das Podest herum sind verschiedene Metalldrähte zu erkennen, die vom Boden zum Podest verlaufen. Du siehst zwei rote, einen schwarzen, vier graue und einen gelben Draht.

Wenn du einen roten Draht herausreißen willst, gehe zu \goto{redHeringCable}.
Wenn du am schwarzen Draht ziehen willst, gehe zu \goto{pullingBlackCable}.
Wenn du einen der grauen Drähte auswählen willst, gehe zu \goto{pullinGreyCable}.
Wenn du dein Glück mit dem gelben Draht versuchen willst, gehe zu \goto{pullingYellowCable}.
Wenn du doch lieber den Schalter drücken willst, gehe zu \goto{pushTheButtonAtTheGem}.
Wenn du lieber nichts machst und den Raum verlassen willst, gehe zu [\ref{downToLayer2}].

\block{pushTheButtonAtTheGem}{Ich drücke zu}

Du siehst nicht viel Sinn darin an den Drähten herumzuspielen. Am Ende richtest du noch größeren Schaden an, als du dir vorstellen kannst. Aber der kleine Schalter scheint dir ein vertretbares Risiko zu sein. Wenn es einen Schalter gibt, wird er dafür gedacht sein, dass man ihn benutzt.

Behutsam legst du deinen Finger auf die kleine Metallplatte und atmest tief ein und aus. Dann drückst du zu. Sofort hörst du ein leichtes Surren, das den Raum erfüllt. Dann beginnt das Podest langsam im Boden zu versinken. Bevor du dich aber genauer damit befassen kannst, hörst du plötzlich ein lautes Knacken über dir. Als du hochsiehst, siehst du, dass dir die gefesselte Elfe entgegenfällt.

Wirf eine Athletikprobe mit DC 14. Wenn du Erfolg hast, gehe zu \goto{catchingTheElf}. Wenn nicht, gehe zu \goto{notCatchingTheElf}.

\block{redHeringCable}{Ich sehe rot}

Du zögerst kurz und greifst dann nach einem der roten Drähte. Es erfordert mehr Kraft, als du gedacht hättest, doch als du kräftig an dem Draht ziehst, reißt er aus der Verbindung am Podest. Erwartungsvoll hälst du die Luft an.

Es passiert nichts. Sekunden vergehen und du bist dir nicht sicher, ob das etwas bewirkt hat. Du weißt auch nicht genau, was du erwartet hast. Vielleicht einfach... \textit{irgendetwas}. Doch es passiert nichts. Also greifst du zum zweiten roten Draht und reißt auch diesen aus der Halterung, wieder ohne erkennbaren Effekt.

Wenn du jetzt am schwarzen Draht ziehen willst, gehe zu \goto{pullingBlackCable}.
Wenn du einen der grauen Drähte herausreißen willst, gehe zu \goto{pullinGreyCable}.
Wenn du dich am gelben Draht versuchen willst, gehe zu \goto{pullingYellowCable}.
Wenn du jetzt den Schalter drücken willst, gehe zu \goto{pushTheButtonAtTheGem}.
Wenn du es dabei belassen und den Raum verlassen willst, gehe zu [\ref{downToLayer2}].

\block{pullingBlackCable}{Ich sehe schwarz}

Du ziehst kräftig am schwarzen Draht. Noch während du das Ende des Drahts ansiehst, nachdem du es aus seiner Halterung gelöst hast, spürst du ein Zittern im Boden. Innerhalb eines Herzschlags wandert dein Blick nach unten und du siehst, wie dir einer der riesigen Steine geradewegs entgegen kommt! Benommen fällts du rückwärts, doch du landest nicht. Unter dir tut sich die Pforte zur Hölle auf, eine sengende Flammenwand schießt nach oben und verwandelt alles in ihrem Weg zu Asche. Auch dich.

\textbf{Ende.}

\block{pullinGreyCable}{Ich sehe grau}

Wahllos nimmst du einen der grauen Drähte und ziehst an ihm, sodass er sich aus der Halterung am Podest löst. Noch während du das Ende des Drahts ansiehst, spürst du ein Zittern im Boden. Kein Zweifel, der ganze Turm bebt!

Du verschwendest keinen Moment und rennst sofort zur Treppe. Wirf einen Geschicklichkeitsrettungswurf mit DC 16. Wenn du Erfolg hast, gehe zu \goto{barelyEscaping}. Ansonsten gehe zu \goto{deathByStones}.

\block{pullingYellowCable}{Ich sehe gelb}

Wie heißt es im Volksmund? ``Alle guten Dinge sind gelb''... oder so ähnlich? Du greifst nach dem gelben Draht und ziehst kräftig, sodass er aus der Halterung am Podest gerissen wird. Ein kurzes Zittern geht durch das Podest, dann siehst du, wie sich die Metallkralle um den Edelstein langsam zurückzieht. Der Stein verliert seinen Halt und fällt zu Boden. Du willst instinktiv zugreifen und ihn auffangen, hälst dann aber doch kurz inne. Ein lautes Klirren durchdringt den Raum, als der Edelstein auf dem harten Steinboden auftrifft und mehrmals auf und ab springt. Als er endlich ruhig liegen bleibt, siehst du, wie das rötliche Leuchten langsam verschwindet und der Steine eine weißliche, trübe Färbung annimmt.

Wenn du den Edelstein vorsichtig aufheben willst, gehe zu \goto{pickUpTheGem}.
Wenn du den Schalter drücken willst, gehe zu \goto{pushTheButtonWithoutGem}.
Wenn du es dabei belassen und den Raum verlassen willst, gehe zu [\ref{downToLayer2}].

\block{pickUpTheGem}{Ich hebe ihn auf}

Du überlegst kurz, ob es eine gute Idee ist den Stein aufzuheben. Allerdings hast du nicht mehr dieses unbändige Verlangen ihn anzufassen, wenn du ihn betrachtest. Und ohne das Leuchten wirkt er auch nicht mehr so bedrohlich. Vorsichtig tippst du den Stein mit dem Finger an und zuckst sofort zurück.

Nachdem eine Weile nichts passiert, atmest du tief ein und greifst nach dem Stein. Es passiert nichts und du steckst da gute Stück ein \getItem{itemGiantGemstone}.

Wenn du den Schalter drücken willst, gehe zu \goto{pushTheButtonWithoutGem}.
Wenn du es dabei belassen und den Raum verlassen willst, gehe zu [\ref{downToLayer2}].

\block{pushTheButtonWithoutGem}{Ich drücke endlich drauf}

Du siehst nicht viel, was du hier noch tun könntest. Da ist nur noch der Schalter am Podest. Du bist darauf gefasst, dass gleich etwas passiert, als sich dein Finger dem Schalter nähert. Du siehst dich noch einmal im Raum um, dann drückst du den Schalter nach Innen.

Du wartest einen Moment. Doch nichts passiert. Also drückst du erneut. Und nochmal. Und nochmal. Doch das Ergebnis bleibt das gleiche, deine Aktion hat keinen für dich erkennbaren Effekt. Du siehst noch einmal hoch zur Elfe, die noch immer an der Decke aufgehangen ist. Dann wieder zum Podest. Ein wenig enttäuscht machst du dich auf den Weg um den Raum zu verlassen. Gehe zu [\ref{downToLayer2}].

\block{deathByStones}{Ich renne um mein Leben}

Du rast die Treppe hinab und durch den kleinen Lagerraum. Um dich herum fallen Steine aus den Wänden, die Decke und der Boden scheinen sich in ihre Einzelteile aufzulösen und du hörst ein ohrenbetäubendes Dröhnen. Im Gang angekommen hast du die nächste Treppe im Blick, als dir plötzlich ein großer Stein vor den Fuß fällt und dich ins Stolpern bringt. Verzweifelt kommst du wieder auf die Beine, doch da trifft dich der nächste Stein, diesmal auf die Schulter. Und noch einer. Dir bleibt keine Zeit, dich zu erholen.

Es dauert einige Minuten, bis sich die gigantische Staubwolke legt. Vom einst bedrohlichen schwarzen Turm ist nur noch ein riesiger Trümmerhaufen geblieben. Für dich gibt es keine Hilfe mehr. Dein lebloser Körper liegt unter mehreren Metern Gestein begraben.

\textbf{Ende.}

\block{barelyEscaping}{Ich fliehe sofort}

Du rast die Treppe hinab und durch den kleinen Lagerraum. Um dich herum fallen Steine aus den Wänden, die Decke und der Boden scheinen sich in ihre Einzelteile aufzulösen und du hörst ein ohrenbetäubendes Dröhnen. Im Gang angekommen hast du die nächste Treppe im Blick, als dir plötzlich ein großer Stein vor den Fuß fällt. Im letzten Moment kannst du ausweichen und rennst weiter. Bei der nächsten Treppe gehst du mehr Risiko ein und nimmst mehrere Stufen auf einmal. Mit Erfolg, du kommst in den Raum mit den leuchtenden Säulen. Von diesen ist nicht mehr viel geblieben, überall liegen große Glassplitter und die reglosen, schleimbedeckten Körper von nackten Männern auf dem Boden in einer riesigen Lache aus der seltsamen Flüssigkeit.

Du siehst durch den Raum und begreifst sofort, dass du den Turm nicht lebend verlassen wirst. Plötzlich gerät das offene Fenster wieder in dein Blickfeld. Du siehst nicht einmal nach, ob der Sprung sicher wäre, du springst einfach. Dir bleibt keine Wahl, jede noch so kleine Chance ist besser als der sichere Tod!

Du landest in einem Dornenbusch und spürst den heftigen Aufprall in jedem Knochen. Wirf 3D6 und zieh dir das Ergebnis von den Lebenspunkten ab. Wenn du den Aufprall überlebst, gehe zu \goto{survivedTheJump}. Andernfalls gehe zu \goto{killedByTheJump}.

\block{killedByTheJump}{Ich bin heftig gestürzt}

Du bist dir nicht sicher, was es am Ende war. Welcher Knochen gebrochen ist und dein Schicksal besiegelt hat. Während der Turm hinter dir in einer riesigen Staubwolke versinkt, die dich komplett umhüllt, kriechst du voller Schmerzen über den Boden und versuchst zu entkommen. Einige Minuten kämpfst du noch qualvoll mit deinem Schicksal, bis deine Kräfte schwinden und du erschöpft die Augen schließt.

\textbf{Ende.}

\block{survivedTheJump}{Ich bin tief gefallen}

Du bist dir sicher, dass du dir einige Knochen gebrochen hast, als du auf dem Boden aufkommst. Mit letzter Kraft zerrst du dich vom Turm weg und kriechst so weit du kommst, als dich die riesige Staubwolke des Zusammenbruchs einhüllt. Es dauert eine Weile, bis sich der Staub legt und der Krach abebbt. Vom vormals furchteinflößenden schwarzen Turm ist nichts als ein gigantiger Schutthaufen geblieben.

Du bist noch nicht außer Gefahr. Hier draußen bist du leichte Beute für Wölfe, besonders in deinem angeschlagenen Zustand. Verzweifelt versuchst du zum nächsten Baum zu kommen, doch dich verlassen vorher die Kräfte. Ohnmächtig sackst du zusammen.

Als du wieder zu dir kommst, starrst du eine Zimmerdecke aus Holz an. Es braucht einen Moment, bis du dir deiner Situation bewusst wirst. Du liegst in einem weichen Bett in einem Zimmer im... ja, im ``Zum Nimmerleer'' in Greifenheim! Deine verletzten Glieder wurden versorgt, du bist in dicke Bandagen gehüllt. Als du den Kopf nach rechts drehst und leicht stöhnst, hörst du plötzlich eine helle Stimme: ``Bei den Göttern, was für ein Glück, ihr lebt!''. Du siehst den Sohn des Wirts, der an gerade durch die Tür kam und dich anstrahlt. ``Vater war sehr besorgt, dass ihr die Nacht nicht überstehen würdet.''

Nach einem kurzen Gespräch bringst du in Erfahrung, dass ein anderer Abenteurer, eine große Schildkröte, dich gefunden und nach Greifenheim gebracht hat. Er hat auch für deine Versorgung und das Zimmer bezahlt für eine ganze Woche bezahlt. Leider weiß niemand, wie dein mysteriöser Retter heißt, denn nachdem er dich ablieferte verschwand er sogleich wieder. Scheinbar ist es wie der junge Wirtssohn gesagt hat, die Götter waren dir gnädig gestimmt. Du hast dieses Abenteuer überlebt und kannst die Geschichte weitererzählen.

\textbf{Ende.}

\block{notCatchingTheElf}{Ich reagiere nicht schnell genug}

Du fängst die Elfe in letzter Sekunde, allerdings mit deinem vollen Körper. Du verlierst 2 Lebenspunkte, als sie aus mehreren Metern Höhe auf dich fällt. Nachdem du dich davon erholt hast, prüfst du die Atmung der Elfe. Endlich eine gute Nachricht, sie lebt, ist aber bewusstlos. Eilig löst du die Fesseln und entdeckst dabei ein dicke Nadel, die mittig in ihrer Brust steckt. Vorsichtig entfernst du das Werkzeug \getItem{itemBloodyNeedle} und drückst auf die kleine Wunde. Sie ist tief, aber nicht besonders breit, sodass du dir keine großen Sorgen machst. Es ist offensichtlich, dass das Ziel war die Elfe so langsam wie möglich ausbluten zu lassen.

Wenn \getEvent{theWarlockDied} bereits eingereten ist, gehe zu \goto{noIntermissionByWarlock}. Wenn keines der Ereignisse eingetreten ist, gehe zu \goto{warlockFoundMe}.

\block{catchingTheElf}{Ich reagiere sofort}

Du kannst gerade noch die Arme ausstrecken und die Elfe abfangen, bevor sie auf dem Boden aufschlägt. Zum Glück ist sie relativ leicht, ansonsten wäre dir dieses Kunststück vermutlich nicht ohne weiteres gelungen. Eilig prüfst du die Atmung der Elfe. Endlich eine gute Nachricht, sie lebt, ist aber bewusstlos. Du löst ihre Fesseln und entdeckst dabei ein dicke Nadel, die mittig in ihrer Brust steckt und aus der frisches Blut tropft. Vorsichtig entfernst du das Werkzeug \getItem{itemBloodyNeedle} und drückst auf die kleine Wunde. Sie ist tief, aber nicht besonders breit, sodass du dir keine großen Sorgen machst. Es ist offensichtlich, dass das Ziel war die Elfe so langsam wie möglich ausbluten zu lassen.

Wenn \getEvent{theWarlockDied} bereits eingereten ist, gehe zu \goto{noIntermissionByWarlock}. Wenn keines der Ereignisse eingetreten ist, gehe zu \goto{warlockFoundMe}.

\block{noIntermissionByWarlock}{Wir sollten gehen}

Du siehst dich im Raum um. Hier gibt es nichts mehr, was du noch tun könntest.
Gehe zu \goto{downToLayer2WithElf}, um die Turmspitze mit der Elfe zu verlassen.

\block{warlockFoundMe}{Ich wurde entdeckt}

Bevor du weiter darüber nachdenken kannst, was du jetzt tun solltest, hörst du plötzlich einen entsetzten Schrei von der Treppe. Du erkennst einen dürren, alten Mann in einer langen, dunklen Robe, der dich wütend anschreit: ``WIE KANNST DU ES WAGEN?''. Das hagere, eingefallene Gesicht ist von tiefen Augenringen gekennzeichnet, die bei der blassen Haut besonders auffallen. Auch die weite Robe kann nicht verbergen, welch knochiger Körper sich unter ihr befindet. Du hast selten einen Mann gesehen, auf den die Beschreibung ``Hexer'' so gut gepasst hat.

Es ist offensichtlich, dass der Hexer nicht einverstanden ist mit dem was du getan hast. Mit wackligen Schritten rennt er auf dich zu und hebt dabei seinen Arm. In seiner Hand hält er einen kleinen Holzstab, dessen Spitze grün zu leuchten beginnt. Doch du lässt dich nicht einfach überraschen, du bist längst aufgestanden und zum Kampf bereit!

\monsterWarlock

Wenn du den Kampf gewinnst, gehe zu [\ref{ikilledthewarlockInTheHighTowerAfterHeSuprisedMe}].
Wenn du verlierst, gehe zu [\ref{anEvilWarlockKilledMe}].

\block{ikilledthewarlockInTheHighTowerAfterHeSuprisedMe}{Ich lasse mich nicht überraschen}

Der Hexer ist dem Tode geweiht, daran besteht für keinen von euch beiden ein Zweifel. Ein atemloser Schrei versucht sich aus seinem Mund zu quälen, doch er bringt nicht mehr als ein Keuchen hervor. Während du zusiehst, kehrt sich das Weiße in seinen Augen nach oben und er sinkt auf die Knie. Noch bevor sein Gesicht auf dem harten Steinboden aufschlägt, hört er auf zu zucken. Reglos bleibt er liegen. Er ist tot.

Markiere Ereignis \getEvent{theWarlockDied}.

Wenn du den Hexer durchsuchen willst, gehe zu [\ref{lootTheWarlockInTheTowerAfterTheSurpise}].
Wenn du den Raum mit der Elfe verlassen willst, gehe zu [\ref{downToLayer2WithElf}].

\block{lootTheWarlockInTheTowerAfterTheSurpise}{Was sein war soll mir gehören}

Du zögerst keine Sekunde und durchsuchst den Toten nach Dingen, die dir helfen könnten.
Kurze Zeit später kannst du einen Zauberstab \getItem{warlockStaff}, einen geschwungenden Dolch \getItem{warlocksDagger}, einen kleinen Kupferschlüssel \getItem{warlocksKey} und einen goldenen Ring \getItem{goldenWarlockRing} dein Eigen nennen.

Außerdem ist dir aufgefallen, dass der linke Arm des Mannes von seltsamen Tätowierungen übersät ist, die dir wie Runen vorkommen. Du erkennst die gleichen Muster auf dem Boden wieder, offensichtlich besteht ein Zusammenhang. Viele kleine Narben und Schnitte durchbrechen die Muster, einige scheinen noch recht frisch und gerade erst verheilt zu sein.

Gehe zu \goto{downToLayer2WithElf}, um den Raum mit der Elfe zu verlassen.

\block{downToLayer2WithElf}{Wir sollten gehen}

Vorsichtig hebst du die Elfe über deine Schulter, sodass du sie einigermaßen tragen kannst. Du kannst so zwar nicht rennen, aber du kannst halbwegs sicher laufen. Du trägst jetzt die Verantwortung für euch beide, wortwörtlich. Markiere Ereignis \getEvent{iGotTheElf} und gehe zu \goto{downToLayer2}.

\block{downToLayer2}{Es geht die Treppe hinab}

Du gehst zurück zur Treppe und bist kurz darauf ein Stockwerk tiefer. Vor dir liegt der kleine Lagerraum mit der Tür auf der anderen Seite. Rechts von dir befinden sich einige Fässer und braune Leinensäcke. Auf der linken Seite steht ein größeres Regal, in dem sich dutzende Gläser befinden, manche befüllt, manche Leer.
%TODO continue writing
%TODO can finde two healing potions \getItem{itemRegularHealingPotion}
