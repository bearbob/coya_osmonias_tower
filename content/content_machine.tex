\chapter*{Der Maschinengott}
%im obersten Zimmer des Turms verbirgt sich ein Maschinengott, der dem Hexer Befehle erteilt
%die maschine ist eine magische kriegsmaschine und benötigt seelenenergie um wieder zu funktionieren


\block{theUpperLevelWithWarlock}{Die oberen Stockwerke}

Irgendwann kommt ihr an einer weiteren Tür an, vor der ihr stehen bleibt. Dein neuer Meister wühlt in seinem Umhang herum und zückt schließlich einen kleinen Schlüssel, mit dem er die Tür öffnet. Du hörst, wie sich ein schwerer Riegel hinter dem Holz bewegt.

Der nachfolgende Raum ist kreisrund und muss ungefähr so groß sein wie die Eingangshalle. Allerdings ist er nicht leer, sondern gefüllt mit knapp einem Dutzend gläsernern Säulen. Im Inneren dieser Säulen wabert eine vioelette Flüssigkeit, die den Raum in ein bedrohliches Licht taucht. Als der Hexer den Raum betritt, flammen an den Wänden Fackeln auf und erhellen die Szenerie. Du kannst nun deutlich erkennen, dass sich in der violetten Flüssigkeit humanoide Formen befinden, die in ihr zu schweben scheinen.

Wenn Ereignis \getEvent{ihaveseentheirfaces} eingetreten ist, gehe zu [\ref{iKnowTheseFaces}]. Ansonsten gehe zu [\ref{weproceedThroughTheRoom}].

\block{weproceedThroughTheRoom}{Durch den Raum}

Entschlossen geht der Hexer durch den Raum und du folgst ihm. Als du einigen der Säulen näher kommst, erkennst du, dass die humanoiden Formen, die du zuvor gesehen hast, Männer mittleren Alters mit breiten Schultern sind. Es ist schwer mehr Details zu erkennen, aber du glaubst, dass die Männer sich alle sehr stark ähneln. Beinahe als wären sie Brüder. Allerdings glaubst du nicht, dass du eine große Familie vor dir hast. Vielmehr muss hier Magie im Spiel sein.

Auf der anderen Seitedes Raums befindet sich ein Fenster und daneben eine weitere Treppe, die höher führt. Ihr folgt dieser Treppe.
Gehe zu [\ref{togetherToLevel2}].

\block{iKnowTheseFaces}{Ich kenne diese Gesichter}

Entschlossen geht der Hexer durch den Raum und du folgst ihm. Als du einigen der Säulen näher kommst, erkennst, dass die humanoiden Formen die du zuvor gesehen hast Männer mittleren Alters mit breiten Schultern sind. Es ist schwer mehr Details zu erkennen, aber du bist dir völlig sicher: es sind immer wieder die gleichen zwei Gesichter. Die Gesichter deiner Entführer, Torlof und sein Kumpane.

Wirf eine Arkanaprobe mit DC 14. Wenn du es nicht schaffst, gehe zu [\ref{arcanaFailedMe}]. Wenn du Erfolg hast, und das Ereignis \getEvent{oneKidnapperDiedA}, \getEvent{oneKidnapperDiedB} oder \getEvent{bothKidnappersDied} eingetreten ist, gehe zu [\ref{theseAreClonesAndOneIsMissing}]. Wenn keins dieser Ereignisse eingetreten ist, gehe zu [\ref{theseAreClones}].

\block{arcanaFailedMe}{Ich kann das nicht zuordnen}

Du kannst nur vermuten, dass es einen dunklen Zauber oder eine ähnliche Hexerei gibt, mit der das alles zu tun hat. Aber was für Pläne hat der Hexer, wenn er sich offensichtlich eine Schar von Handlangern heranzüchtet?

Während du noch darüber nachdenkst, erreicht ihr die andere Seite des Raums. Hier befindet sich ein Fenster und eine weitere Treppe, die höher führt. Ihr folgt dieser Treppe.
Gehe zu [\ref{togetherToLevel2}].

\block{theseAreClones}{Ich habe davon gehört}

Du kannst dich düster erinnern, dass es einen Zauber gibt, mit dem man Klone erschaffen kann. Wenn einer der beiden Entführer stirbt, fährt seine Seele in einen der schlafenden Körper. Aber was für Pläne hat der Hexer, wenn er nicht nur einmal, sondern so oft auf den Tod seiner Handlanger vorbereitet sein will?

Während du noch darüber nachdenkst, erreicht ihr die andere Seite des Raums. Hier befindet sich ein Fenster und eine weitere Treppe, die höher führt. Ihr folgt dieser Treppe.
Gehe zu [\ref{togetherToLevel2}].

\block{theseAreClonesAndOneIsMissing}{Mir fehlt etwas}

Du kannst dich düster erinnern, dass es einen Zauber gibt, mit dem man Klone erschaffen kann. Wenn einer der beiden Entführer stirbt, fährt seine Seele in einen der schlafenden Körper. Diese Theorie wird bestätigt, als du eine Säule entdeckst, die keinen Inhalt hat. Vermutlich wurde einer der beiden nach dem Angriff im Wald wieder zum Leben erweckt. Aber was für Pläne hat der Hexer, wenn er nicht nur einmal, sondern so oft auf den Tod seiner Handlanger vorbereitet sein will?

Während du noch darüber nachdenkst, erreicht ihr die andere Seite des Raums. Hier befindet sich ein Fenster und eine weitere Treppe, die höher führt. Ihr folgt dieser Treppe.
Gehe zu [\ref{togetherToLevel2}].

\block{stairsToTheUpperLevels}{Die Treppe nach oben}

Du folgst den Treppenstufen nach oben. Wenig später kommst du zu einer kleinen Tür, die deinen Weg blockiert. Ein kurzer Versuch am Türgriff zeigt, dass die Tür verschlossen ist.

Wenn du \getItem{warlocksKey} besitzt, kannst du zu [\ref{unlockTheDoorWithWarlocksKey}] gehen. Ansonsten musst du umkehren. Gehe zu [\ref{goToCrossroom}].

\block{togetherToLevel2}{Höher hinauf}

Als ihr am Fenster vorbei kommt, siehst du, dass es inzwischen dunkel geworden ist. Noch immer scheinen die dichten Wolken den Himmel zu bedecken, denn du kannst in diesem flüchtigen Moment keine Sterne oder den Schein des Mondes erkennen.

Euer Weg führt euch die Treppe hinauf in die nächste Etage des Turms. Es geht einen langen Gang entlang, an dessen Ende sich ein kleiner Runde Raum mit 4 Türen befindet, eine in jede Himmelsrichtung. Der Hexer greift nach dem Türgriff der östlichen Tür und öffnet sie. Ihr durchquert ein kleines Lager, du siehst einige Fässer, Säcke und gefüllte Gläser. Am Ende wartet eine weitere Treppe darauf erklommen zu werden. Vom Ende der Treppe strahlt euch bereits ein rötliches Licht entgegen, das wild flackert und furchterregende Schatten an die Wand wirft.

Ihr befindet euch nun im obersten Zimmer des Turms. Keine weitere Treppe, keine Türen, keine Fenster. Nur ein großer, runder Raum, der ungefähr 10 Meter im Durchmesser misst. An der Außenwand sind in kurzem Abstand hunderte schwarzer Kerzen aufgereiht, deren kleine Flammen sich zuckend hin und her bewegen.
Doch die größte Lichtquelle befindet sich mitten im Raum. Auf einem Podest, dass sich wie eine Pyramide zuspitzt, befindet sich in Brusthöhe ein faustgroßer Edelstein. Er ist in eine kleine Metallkralle gefasst, von der Metallfäden verschiedenster Farbe ausgehen. Du hast selten größere Juwelen gesehen, doch besonders das rote Licht, welches dieser Stein ausstrahlt, zieht dich in seinen Bann.

Wirf einen Charismarettungswurf mit DC 16 oder mit DC 15, wenn du \getItem{itemBurialRing} besitzt. Wenn du es nicht schaffst, gehe zu [\ref{cannotresistthegem}]. Wenn du Erfolg hast, gehe zu [\ref{resistingTheGem}].

\block{resistingTheGem}{Ich kann mich abwenden}

Für einen Moment bist du wie verzaubert. Als deine Augen den Edelstein erblicken, vergisst du die Welt um dich herum. Doch dann schaffst du es, deine Gedanken wegzulenken und abwärts zu sehen. Erst jetzt fällt dir auf, dass sich am Podest und auf dem Boden schwarze Flecken befinden. Unter den Flecken, über den gesamten Boden verteilt, befindet sich ein riesiger Runenkreis mit merkwürdigen Symbolen, die mit Sicherheit keinem guten Zweck gewidmet sind.

Als du den Blick zur Decke schweifen lässt, erkennst du eine Frau, eine Elfe, die dort in einigen Metern Höhe kopfüber aufgehangen ist. Sie scheint bewusstlos, ihr blondes Haar ist blutverschmiert und hängt nach unten. Immer wieder fallen kleine Bluttropfen von ihr hinab auf den Edelstein, zweifelsohne absichtlich.

``Nur keine Scheu, fass den Stein an.'', sagt er Hexer mit einem bösartigen Unterton.
Wenn Ereignis \getEvent{deceivedTheWarlock} eingetreten ist, gehe zu [\ref{idonthavetotouchthegem}]. Wenn nicht, gehe zu [\ref{hewantsmetotouchthegem}].

\block{idonthavetotouchthegem}{Die Scharade hat ein Ende}

Es reicht, das Schauspiel endet jetzt! Ruckartig drehst du dich um und gehst auf den Hexer los. Der Ausdruck der Überraschung in seinen Augen weicht schnell dem Schock, als er realisiert, dass er keine Macht über dich hat!

\monsterWarlock

Wenn du den Kampf gewinnst, gehe zu [\ref{ikilledthewarlockInTheHighTower}].
Wenn du verlierst, gehe zu [\ref{anEvilWarlockKilledMe}].

\block{ikilledthewarlockInTheHighTower}{Nur einer von uns wird leben}

Der Hexer ist dem Tode geweiht, daran besteht für keinen von euch beiden ein Zweifel. Ein atemloser Schrei versucht sich aus seinem Mund zu quälen, doch er bringt nicht mehr als ein Keuchen hervor. Während du zusiehst kehrt sich das Weiße in seinen Augen nach oben und er sinkt auf die Knie. Noch bevor sein Gesicht auf dem feuchten Erdboden aufschlägt, hört er auf zu zucken. Reglos bleibt er liegen. Er ist tot.

Markiere Ereignis \getEvent{theWarlockDiedAtTheTop}.

Wenn du den Hexer durchsuchen willst, gehe zu [\ref{lootTheWarlockInTheTower}].
Wenn du das Podest mit dem Edelstein untersuchen willst, gehe zu [\ref{inspectTheGem}].
Wenn du die Elfe genauer betrachten willst, gehe zu [\ref{inspectTheElf}].
Wenn du den Raum verlassen willst, gehe zu [\ref{downToLayer2}].

\block{lootTheWarlockInTheTower}{Was sein war ist mein}

Du zögerst keine Sekunde und durchsuchst den Toten nach Dingen, die dir helfen könnten.
Kurze Zeit später kannst du einen Zauberstab \getItem{warlockStaff}, einen geschwungenden Dolch \getItem{warlocksDagger}, einen kleinen Kupferschlüssel \getItem{warlocksKey} und einen goldenen Ring \getItem{goldenWarlockRing} dein Eigen nennen.

Außerdem ist dir aufgefallen, dass der linke Arm des Mannes von seltsamen Tätowierungen übersät ist, die dir wie Runen vorkommen. Du erkennst die gleichen Muster auf dem Boden um den Edelstein wieder, offensichtlich besteht ein Zusammenhang. Viele kleine Narben und Schnitte durchbrechen die Muster, einige scheinen noch recht frisch und gerade erst verheilt zu sein.

Wenn du das Podest mit dem Edelstein untersuchen willst, gehe zu \goto{inspectTheGem}.
Wenn du die Elfe genauer betrachten willst, gehe zu \goto{inspectTheElf}.
Wenn du den Raum verlassen willst, gehe zu \goto{downToLayer2}.

\block{inspectTheGem}{Ich sehe mir den Edelstein genauer an}

Du widmest deine Aufmerksamkeit dem Edelstein. Als du ihn ansiehst, spürst du sofort wieder das Verlangen ihn zu berühren. Schnell wendest du deine Augen nach unten ab. Dein Blick fällt auf das Podest. Erst jetzt bemerkst du, dass es aus Metall besteht und scheinbar nicht auf dem Boden steht, sondern \textit{durch} den Boden geht. Knapp über dem Boden siehst du eine kleine Platte, auf der die Zeichen ``F.R.E-Y4'' eingraviert sind.

Wirf eine Weisheitsprobe mit DC 12. Wenn du es schaffst, gehe zu \goto{iHaveMachineKnowledge}.
Wenn nicht, gehe zu \goto{skippingMachineKnowledge}.

\block{hewantsmetotouchthegem}{Ich soll es berühren}

Du stehst noch immer unter dem Zauber des Hexers, der deinen Körper kontrolliert. Gegen deinen Willen bewegen sich deine Füße in Richtung des Edelsteins! Du hörst den Hexer noch sagen: ``Nur keine Scheu, erfülle deinen Zweck!''

Nenn es Todesahnung, nenn es schlechtes Bauchgefühl, doch dein Geist sträubt sich vehement gegen den Befehl! Wirf einen Charismarettungswurf mit DC 12 oder mit DC 11, wenn du \getItem{itemBurialRing} besitzt. Wenn du erfolgreich bist, gehe zu [\ref{noiwillnottouchthisgem}]. Wenn du versagst, gehe zu [\ref{deathbyGem}].

\block{noiwillnottouchthisgem}{Ich werde das nicht tun}

Du bist dir sicher, dass es dein Todesurteil ist, wenn du den Stein berührst. Magische Kontrolle oder nicht, du wirst dein Leben nicht ohne einen Kampf opfern! Während sich deine Füße wie von allein weiter bewegen, tobt in dir ein Gefecht mit der dunklen Magie, die dich beherrscht. Schlussendlich kommst du wieder zu Sinnen und hast die volle Kontrolle zurück.

Du bist frei. Und das keinen Moment zu spät, als du nach unten siehst, hast du schon die Hand ausgestreckt um den Edelstein zu berühren. Doch dazu wird es nicht kommen.

Gehe zu [\ref{idonthavetotouchthegem}]

\block{cannotresistthegem}{Ich will es berühren}

Du bist wie verzaubert. Als deine Augen den Edelstein erblicken, vergisst du alles um dich herum. Du achtest gar nicht darauf, was sich noch im Raum befindet. Unterbewusst bemerkst du, dass der Hexer mit dir spricht, doch es ist wie ein unverständliches Rauschen in deinen Ohren. Und du willst ihm nicht zuhören. Du musst ihm nicht zuhören. Nein, du musst gar nichts. Nur zum Edelstein, ja. Du musst den Edelstein berühren. Gehe zu [\ref{deathbyGem}].

\block{deathbyGem}{Ich habe es berührt}

Wie in Trance kommst du dem Stein näher, Schritt um Schritt, bis deine Finger ihn berühren können. Irgendetwas zerrt an dir. Doch niemand wird dich jetzt noch aufhalten! Du streckst die Hand aus und berührst den Stein. Sofort umgibt dich Stille. Das Zerren hört auf. Alles ist schwarz.

Du weißt nicht, wie viel Zeit vergeht. Dein Körper ist wie betäubt, du spürst nichts, siehst nichts und hörst nichts. Dann wird es plötzlich wärmer. Erst jetzt bemerkst du, wie kalt dir war. Doch auch die Kälte ist bald vergessen, als die Wärme nicht mehr abnimmt und sich zu Hitze entwickelt. Und schon bald wird aus der Hitze ein schmerzvolles Brennen. Du kannst das Gefühl nicht beschreiben, doch es will einfach nicht aufhören. Wenn du sterben könntest, wärst du inzwischen gestorben. So viel Schmerz kann keine Seele ertragen. Doch irgendetwas hält dich auf! Du kannst nicht in die nächste Welt, der Weg ist versperrt!

Du weißt nicht, wie viel Zeit vergeht. Tage? Jahre? Jahrhunderte? Deine Welt besteht nur noch aus Verzweiflung und Schmerz, der jeden klaren Gedanken verhindert.

\textbf{Ende.}

\block{reachingTheWindow}{Wenn sich eine Tür schließt, öffnet sich ein Fenster}

Du erreichst mit schmerzenden Fingern den Fenstersims und kannst dich endlich wieder richtig festhalten. Dein Kletterpfad war vermutlich nicht der einfachste, aber er hat dich ans Ziel geführt. Über die beiden Momente, in denen du fast den Halt verloren und metertief gefallen wärst, denkst du nicht weiter nach.

Du atmest tief ein und ziehst dich nach oben. Vor dir liegt ein großer, kreisrunder Raum, der von einem violetten Schimmer erhellt wird. Knapp ein Dutzend gläserner Säulen sind die Quellen dieses Lichts. In ihnen wabert eine dunkle Flüssigkeit, in der du jeweils ein menschliche Form erkennen kannst. Irgendetwas... oder irgendjemand steckt in diesen Säulen.

Auf deiner rechten Seite befindet sich eine Treppe, die entlang der Außenwand in ein höheres Stockwerk führt. Auf der dir gegenüber liegenden Seite des Raums kannst du eine kleine Tür erkennen.

Wenn du der Treppe nach oben folgen willst, gehe zu [\ref{goToLevel3Alone}].
Wenn du die Säulen genauer betrachten willst, gehe zu [\ref{inspectTheVioletPillars}].
Wenn du zur Tür gegen willst, gehe zu [\ref{theLockedDoorFromTheOtherSide}].

\block{unlockTheDoorWithWarlocksKey}{Jede Tür hat einen Schlüssel}

Wie es das Schicksal will, scheint der Schlüssel, den du dem Hexer genommen hast, genau in das Schloss zu passen. Weder ein Zufall, noch sonderlich überraschend. Du öffnest die Tür.

Vor dir liegt ein großer, kreisrunder Raum, der von einem violetten Schimmer erhellt wird. Knapp ein Dutzend gläserner Säulen sind die Quellen dieses Lichts. In ihnen wabert eine dunkle Flüssigkeit, in der du jeweils ein menschliche Form erkennen kannst. Irgendetwas... oder irgendjemand steckt in diesen Säulen.

Auf der dir gegenüberliegenden Seite, vorbei an diesen Säulen, befindet sich ein Fenster. Links neben dem Fenster kannst du Treppenstufe erkennen, die entlang der Außenwand in ein höheres Stockwerk führen.

Wenn du der Treppe nach oben folgen willst, gehe zu [\ref{goToLevel3Alone}].
Wenn du die Säulen genauer betrachten willst, gehe zu [\ref{inspectTheVioletPillars}].
