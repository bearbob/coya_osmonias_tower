\chapter{Die Reise beginnt...}

\block{startpage}{Unsanftes Erwachen}
Leise rauscht der warme Herbstwind durch das vielfarbige Blätterdach des Waldes. Viele Bäume haben bereits den Großteil ihrer Blätter abgeworfen und färben so den schmalen Weg in zarte Rot- und Goldtöne.\\
Das ganze Bild strahl eine tiefe Ruhe aus, die du jedoch nur schwerlich genießen kannst. Du liegst auf einem Pferdekarren, gelenkt von zwei muskelbepackten Männern. Ihre breiten Rücken sind dir zugewandt, sodass du ihre Gesichter nicht erkennen kannst. Beide haben schulterlange blonde Haare, die der rechte zu einem kleinen Knoten geformt hat. Scheinbar haben sie noch nicht bemerkt, dass du wieder zu dir gekommen bist, denn sie unterhalten sich angeregt. Leider sind ihre Worte durch die Geräusche des Karrens schwer zu verstehen. Dir ist noch nicht ganz klar, wie du in diese Situation gekommen bist, aber die dröhnenden Kopfschmerzen lassen nichts gutes vermuten. Schnell stellst du fest, dass deine Hände hinter dem Rücken zusammengebunden sind und deine Knöchel ebenfalls gefesselt sind. Dein Kiefer schmerzt ein wenig von dem dicken Stoffknebel.
\\Wenn du dich bemerkbar machen willst, gehe zu [\ref{getAttention}].
\\Wenn du dich weiter umsehen willst, gehe zu [\ref{lookAtCart}].
\\Wenn du abwarten möchtest, gehe zu [\ref{waitInCart}].
\\Wenn du versuchen willst zu verstehen, worüber die beiden reden, gehe zu [\ref{listenInCart}].

\block{getAttention}{Ich hätte gern Aufmerksamkeit}

Es wird Zeit herauszufinden mit wem du hier eigentlich unterwegs bist. Leider werden deine Worte durch den Knebel zu stark gedämpft, um beim Knarzen und Ächzen des Wagens gehört zu werden.
\\Wenn du gegen den Kutschbock treten möchtes, gehe zu [\ref{kickCart}].
\\Wenn du dich umsehen möchtest, gehe zu [\ref{lookAtCart}].
\\Wenn du abwarten möchtest, gehe zu [\ref{waitInCart}].

\block{kickCart}{Ich brauche Aufmerksamkeit}

Da dir im Moment nichts besseres einfällt, drehst du dich mit den Beinen zum Kutschbock und trittst beherzt gegen das alte Holz. Wie erwartet sorgt das dumpfe Geräusch dafür, dass sich deine Entführer umdrehen. Die beiden könnten Zwillinge sein, so ähnlich sehen sie sich. Beide haben grobschlächtige Gesichter und breiten Nasen. Kleine Bartstoppeln umrahmen den Mund, der bei beiden ein leichtes Grinsen formt. Trotzdem sind sie leicht zu unterscheiden, denn beim rechten Halunken zieht sich eine große Narbe von der linken Wange über die Nase zur rechten Augenbraue.\\
``Sieh dir das an, Torlof, unsere Prinzessin ist aufgewacht!'', sagt er laut mit tiefer Stimme. Torlof antwortet nicht, sondern brummt nur belustigt vor sich hin. Sein Partner fügt hinzu: ``Na, dann tu mal den Rest deiner Reise genießen, ein paar Stunden wird es noch dauern bis wir da sind!''\\
Dein fragender Blick wird keiner Antwort gewürdigt. Die beiden Männer lachen kurz und drehen sich dann wieder nach vorn.
\\Wenn du dich umsehen möchtest, gehe zu [\ref{lookAtCart2}].
\\Wenn du warten möchtest, gehe zu [\ref{waitInCart}].

\block{waitInCart}{Ich akzeptiere mein Schicksal}

Du bist dir nicht sicher ob es nur ein paar Stunden sind, doch es fühlt sich wie eine Ewigkeit an. Du liegst mit dem Rücken im Wagen und beobachtest den Himmel, an dem hin und wieder dicke weiße Wolken vorbeiziehen. Scheinbar seid ihr inzwischen in einem Wald, denn um euch herum sind immer mehr Bäume zu sehen und der Weg ist noch holpriger geworden als zuvor. Man muss kein Genie sein um zu wissen, dass es sich nicht um eine vielbefahrene Handelsstraße handelt.\\
Wirf einen W20. Ist das Ergebnis mindestens 16, gehe zu [\ref{banditEncounter}].\\
Ansonsten gehe zu [\ref{arrivalAtTower}].

\block{lookAtCart}{Ich sehe mich um}

Du lässt deinen Blick schweifen und versuchst deine aktuelle Lage besser zu erfassen. Der Karren ist aus Holz und hat schon bessere Tage gesehen. Am hinteren Ende befindet sich eine Ladeklappe, die von Metallriegeln auf beiden Seiten gehalten wird. Neben dir befinden sich zwei gefüllte braune Leinensäcke.
\\Wenn du zur Ladeklappe kriechen willst, gehe zu [\ref{loadramp}].
\\Wenn du warten möchtest, gehe zu [\ref{waitInCart}].
\\Wenn du versuchen willst herauszufinden was sich in den Säcken befindet, gehe zu [\ref{inspectCargo}].

\block{loadramp}{Ich krieche zur Ladeklappe}

Es ist ein kleines Kunststück und dauert einige Minuten, aber du schaffst es dich zu drehen und zur Klappe zu kriechen. Die Bretter werden von Nägeln zusammengehalten, denen Wind und Wetter stark zugesetzt haben. Trotzdem glaubst du, dass es kein Kinderspiel wäre die Klappe mit Gewalt zu öffnen - was vor allem an deiner eingeschränkten Bewegungsmöglichkeit liegt.
\\Wenn du versuchen willst die Klappe aufzutreten, gehe zu [\ref{forceLoadramp}].
\\Wenn du versuchen willst die Riegel zu öffnen, gehe zu [\ref{unlockLoadramp}].
\\Wenn du versuchen willst herauszufinden was sich in den Säcken befindet, gehe zu [\ref{inspectCargo}].
\\Wenn du warten möchtest, gehe zu [\ref{waitInCart}].

\block{forceLoadramp}{Ich trete gegen die Ladeklappe}

Es braucht wieder einige Minuten der Vorbereitung, bis du dich auf den Rücken gedreht hast und mit angewinkelten Knien vor der Holzklappe liegst.\\
Lege eine Stärkeprobe mit DC 16 ab. Wenn du erfolgreich bist, gehe zu [\ref{forceLoadrampsuccess}].
Wenn du scheiterst, gehe zu [\ref{forceLoadrampfailed}].

\block{forceLoadrampsuccess}{Ich bin stärker als Holz}

Du konzentrierst dich und sammelst deine Kräfte für einen gezielten Tritt gegen die Seite der Holzklappe. Das Glück ist endlich auf deiner Seite, das Material gibt nach und bricht am Metallriegel. Ein nervöser Blick auf deine Entführer zeigt dir, dass sie überhaupt nicht bemerken, was du da treibst. Schnellst sammelst du dich für einen zweiten Tritt auf der anderen Seite und die Klappe fällt nach unten ab. Einige Herzschläge später rollst du seitwärts vom Wagen und fällst auf den steinigen Boden. Dabei verlierst du 2 Lebenspunkte.\\
Während du versuchst nicht vor Schmerz zu stöhnen, rollt der Wagen weiter den Weg entlang. Die beiden Halunken ahnen nicht einmal, dass du nicht wie ein Lamm auf die Schlachtung wartest. Schnell kriechst du vom Weg zu einem der Bäume, während sich der Wagen Meter um Meter von dir entfernt.

Es dauert einige Stunden, doch irgendwann kannst du deine Fesseln lösen. Du machst dich auf den Rückweg, wobei du darauf achtest dem Weg fern zu bleiben.
Du wirst zwar nie erfahren, warum du entführt wurdest, aber manche Geheimnisse können auch ungelöst bleiben.

\textbf{Ende.}

\block{forceLoadrampfailed}{Das Holz hat gewonnen}

Du konzentrierst dich und sammelst deine Kräfte für einen gezielten Tritt gegen die Seite der Holzklappe. Zu deinem Pech rührt sich überhaupt nichts, woran auch ein zweiter und ein dritter Tritt nichts ändern. Als du zum vierten Tritt ansetzt, bemerkst du plötzlich einen Schatten über dir und hörst eine tiefe Stimme ``Na, sowas sehen wir hier aber gar nicht gern!'' brummen, bevor du einen harten Schlag auf den Kopf bekommst und das Bewusstsein verlierst.\\
Du verlierst 5 Lebenspunkte.
\\Gehe zu [\ref{wakeUpInCell}].

\block{lookAtCart2}{Ich sehe mich um}
% copy of lookAtCart after the hero has seen the kidnappers
Du lässt deinen Blick schweifen und versuchst deine aktuelle Lage besser zu erfassen. Der Karren ist aus Holz und hat schon bessere Tage gesehen. Am hinteren Ende befindet sich eine Ladeklappe, die von Metallriegeln auf beiden Seiten gehalten wird. Neben dir befinden sich zwei gefüllte braune Leinensäcke. Torlof und sein Kumpane scheinen wieder in ihr Gespräch vertieft zu sein.
\\Wenn du zur Ladeklappe kriechen willst, gehe zu [\ref{loadramp}].
\\Wenn du warten möchtest, gehe zu [\ref{waitInCart}].
\\Wenn du versuchen willst herauszufinden was sich in den Säcken befindet, gehe zu [\ref{inspectCargo2}].

\block{arrivalAtTower}{Die Ankunft}

In der Ferne drückt sich ein mächtiger Turm aus schwarzem Stein wie ein Dorn in den Himmel. Allein der Anblick sorgt für ein mulmiges Gefühl in der Bauchgegend, was dadurch verstärkt wird, dass dieser Turm offensichtlich das Ziel der Reise ist.

Je näher ihr kommt, desto mehr beunruhigende Details kommen zum Vorschein. Die Bäume in der Nähe sind kahl und haben eine blässliche, tote Farbe angenommen. Das zuvor hörbare Zwitschern der Vögel ist verstummt und ein leichter Wind trägt kalte Luft zu dir, sodass du kurz ungewollt erschauderst. Für einen Moment sieht es so aus, als sei der Torbogen um die große hölzerne Tür aus Schädeln konstruiert. Als du erneut hinsiehst, erkennst du, dass es sich um kleine Skulpturen handelt, die schreckliche entstellte Fratzen abbilden.

Ohne anzuhalten fährt der Wagen auf die Tür zu, die wie von Geisterhand und mit hörbarem Ächzen nach links und rechts aufschwingt und Einlass gewährt. Im Inneren befindet sich ein kreisrunder Raum, der ungefähr 10 Meter im Durchmesser misst. Gegenüber des Eingangstors befindet sich eine weitere Holztür. Während du die Decke des Raums betrachtest, an der ein Kronleuchter aus schwarzem Metall hängt, rüttelt der Wagen kurz. Deine Entführer sind abgestiegen und nur Sekunden später wird dir ein Leinensack über den Kopf gezogen. Du hörst eine fremde Stimme, die nicht zu deinen beiden Entführern gehört, sagen: ``Sehr gut... in den Kerker...''. Kurz darauf wirst du hochgehoben.
\\Wenn du dich wehren willst, gehe zu [\ref{resistCell}].
\\Wenn du abwarten willst, gehe zu [\ref{arriveInCell}].

\block{resistCell}{Nicht ohne Widerstand}

Ein Kerker ist selten etwas Gutes. Du fängst an dich heftig zu wehren und schlägst mit den gefesselten Gliedmaßen um dich. Leider nur mit mäßigem Erfolg, doch immerhin bewegst du dich so ungünstig, dass dein Entführer nicht anders kann als dich fallen zu lassen. Du hörst wie er sagt: ``Nun aber genug mit den Spielchen!''. Kurz darauf trifft dich etwas hartes an der Schläfe und du verlierst das Bewusstsein.\\
Du verlierst 4 Lebenspunkte.
\\Gehe zu [\ref{wakeUpInCell}].

\block{arriveInCell}{Ohne Widerstand}

Du spürst, dass es kühler wird und ihr abwärts geht, vermutlich eine Treppe hinab. Modriger Geruch steigt dir in die Nase und du hörst das Quietschen alter Metallgitter. Wenige Sekunden später wirst du auf den Boden geworfen und der Leinensack wird entfernt.
\\Gehe zu [\ref{welcomeToTheCell}].

\block{wakeUpInCell}{Unsanftes Erwachen}

Als du wieder zu dir kommst, steigt dir ein modriger Geruch in die Nase. Es ist kalt und feucht und du bist noch immer gefesselt. Offensichtlich liegst du nicht mehr auf dem Wagen, denn der Boden ist weicher und riecht nach Erde.
\\Gehe zu [\ref{welcomeToTheCell}].
