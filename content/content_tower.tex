\chapter{Der schwarze Turm}

\block{arrivalAtTower}{Die Ankunft}

In der Ferne drückt sich ein mächtiger Turm aus schwarzem Stein wie ein Dorn in den Himmel. Allein der Anblick sorgt für ein mulmiges Gefühl in der Bauchgegend, was dadurch verstärkt wird, dass dieser Turm offensichtlich das Ziel der Reise ist.

Je näher ihr ihm kommt, desto mehr beunruhigende Details kommen zum Vorschein. Die Bäume in der Nähe sind kahl und haben eine blässliche, tote Farbe angenommen. Das zuvor hörbare Zwitschern der Vögel ist verstummt und ein leichter Wind trägt kalte Luft zu dir, sodass du kurz ungewollt erschauderst. Für einen Moment sieht es so aus, als sei der Torbogen um die große hölzerne Tür aus Schädeln konstruiert. Als du erneut hinsiehst, erkennst du, dass es sich um kleine Skulpturen handelt, die schreckliche entstellte Fratzen abbilden. Es ist schwer zu sagen, wie hoch der Turm wirklich ist, denn seine Spitze wird von einem Trichter aus dunklen Wolken verdeckt, die sich wie in einer Spirale langsam drehen.

Ohne anzuhalten fährt der Wagen auf die Tür zu, die wie von Geisterhand und mit hörbarem Ächzen nach links und rechts aufschwingt und Einlass gewährt. Im Inneren befindet sich ein kreisrunder Raum, der ungefähr 10 Meter im Durchmesser misst. Gegenüber des Eingangstors befindet sich eine weitere Holztür. Während du die Decke des Raums betrachtest, an der ein Kronleuchter aus schwarzem Metall hängt, rüttelt der Wagen kurz. Deine Entführer sind abgestiegen und nur Sekunden später wird dir ein Leinensack über den Kopf gezogen. Du hörst eine fremde Stimme krächzen: ``Sehr gut... in den Kerker...''. Kurz darauf wirst du hochgehoben.
\\Wenn du dich wehren willst, gehe zu [\ref{resistCell}].
\\Wenn du abwarten willst, gehe zu [\ref{arriveInCell}].

\block{infiltrateTower}{Der Einbruch}

Nach knapp zwei weiteren Stunden Fußmarsch drückt sich in der Ferne ein mächtiger Turm aus schwarzem Stein wie ein Dorn in den Himmel. Allein der Anblick sorgt für ein mulmiges Gefühl in der Bauchgegend, was dadurch verstärkt wird, dass dieser Turm offensichtlich das Ziel deiner Reise ist.

Je näher du ihm kommst, desto mehr beunruhigende Details kommen zum Vorschein. Die Bäume in der Nähe sind kahl und haben eine blässliche, tote Farbe angenommen. Das zuvor hörbare Zwitschern der Vögel ist verstummt und ein leichter Wind trägt kalte, modrige Luft zu dir, sodass du kurz ungewollt erschauderst. Du stehst nun vor der rechten Seite des Turms. Zu deiner linken kannst du das Eingangstor sehen. Für einen Moment sieht es so aus, als sei der Torbogen um die große hölzerne Tür aus Schädeln konstruiert. Als du erneut hinsiehst, erkennst du, dass es sich um kleine Skulpturen handelt, die schreckliche entstellte Fratzen abbilden.

Es ist schwer zu sagen, wie hoch der Turm wirklich ist, denn seine Spitze wird von einem Trichter aus dunklen Wolken verdeckt, die sich wie in einer Spirale langsam drehen. Du kannst in ungefähr 10 Metern Höhe ein Fenster erkennen. Am Fuß des Turms befinden sich, außer vor dem Tor, dichte Dornenbüsche. 

%TODO Option

\block{resistCell}{Nicht ohne Widerstand}

Ein Kerker ist selten etwas Gutes. Du fängst an dich heftig zu wehren und schlägst mit den gefesselten Gliedmaßen um dich. Leider nur mit mäßigem Erfolg, doch immerhin bewegst du dich so ungünstig, dass dein Entführer nicht anders kann als dich fallen zu lassen. Du hörst wie er wütend sagt: ``Nun aber genug mit den Spielchen!''. Kurz darauf trifft dich etwas hartes an der Schläfe und du verlierst das Bewusstsein.\\
Du verlierst 4 Lebenspunkte.
\\Gehe zu [\ref{wakeUpInCell}].

\block{arriveInCell}{Ohne Widerstand}

Du spürst, dass es kühler wird und ihr abwärts geht, vermutlich eine Treppe hinab. Modriger Geruch steigt dir in die Nase und du hörst das Quietschen alter Metallgitter. Wenige Sekunden später wirst du auf den Boden geworfen und der Leinensack wird entfernt.
\\Gehe zu [\ref{welcomeToTheCell}].

\block{arrivalAtTowerSingle}{Die verspätete Ankunft}

In der Ferne drückt sich ein mächtiger Turm aus schwarzem Stein wie ein Dorn in den Himmel. Allein der Anblick sorgt für ein mulmiges Gefühl in der Bauchgegend, was dadurch verstärkt wird, dass dieser Turm offensichtlich das Ziel der Reise ist. Dein verbleibender Entführer hat sich ab und an zu dir umgedreht und dich hasserfüllt angesehen, aber kein weiteres Wort verloren.

Je näher ihr dem Turm kommt, desto mehr beunruhigende Details kommen zum Vorschein. Die Bäume in der Nähe sind kahl und haben eine blässliche, tote Farbe angenommen. Das zuvor hörbare Zwitschern der Vögel ist verstummt und ein leichter Wind trägt kalte Luft zu dir, sodass du kurz ungewollt erschauderst. Für einen Moment sieht es so aus, als sei der Torbogen um die große hölzerne Tür aus Schädeln konstruiert. Als du erneut hinsiehst, erkennst du, dass es sich um kleine Skulpturen handelt, die schreckliche entstellte Fratzen abbilden. Es ist schwer zu sagen, wie hoch der Turm wirklich ist, denn seine Spitze wird von einem Trichter aus dunklen Wolken verdeckt, die sich wie in einer Spirale langsam drehen.

Ohne anzuhalten fährt der Wagen auf die Tür zu, die wie von Geisterhand und mit hörbarem Ächzen nach links und rechts aufschwingt und Einlass gewährt. Im Inneren befindet sich ein kreisrunder Raum, der ungefähr 10 Meter im Durchmesser misst. Gegenüber des Eingangstors befindet sich eine weitere Holztür. Während du die Decke des Raums betrachtest, an der ein Kronleuchter aus schwarzem Metall hängt, rüttelt der Wagen kurz. Dein Entführer ist abgestiegen und nur Sekunden später wird dir ein Leinensack über den Kopf gezogen. Eine fremde Stimme fragt krächzend: ``Ihr... seid spät. Wo ist Wladan?''. Du hörst ein leises Gemurmel als Antwort und die Reaktion des Fremden: ``Ein Schande... in den Kerker...''. Kurz darauf wirst du hochgehoben.
\\Wenn du dich wehren willst, gehe zu [\ref{resistCell}].
\\Wenn du abwarten willst, gehe zu [\ref{arriveInCell}].

\block{wakeUpInCell}{Unsanftes Erwachen}

Als du wieder zu dir kommst, steigt dir ein modriger Geruch in die Nase. Es ist kalt und feucht und du bist noch immer gefesselt. Offensichtlich liegst du nicht mehr auf dem Wagen, denn der Boden ist weicher und riecht nach Erde.
\\Gehe zu [\ref{welcomeToTheCell}].
