\chapter*{Der schwarze Turm}

\block{arrivalAtTower}{Die Ankunft}

In der Ferne drückt sich ein mächtiger Turm aus schwarzem Stein wie ein Dorn in den Himmel. Allein der Anblick sorgt für ein mulmiges Gefühl in der Bauchgegend, was dadurch verstärkt wird, dass dieser Turm offensichtlich das Ziel der Reise ist.

Je näher ihr ihm kommt, desto mehr beunruhigende Details kommen zum Vorschein. Die Bäume in der Nähe sind kahl und haben eine blässliche, tote Farbe angenommen. Das zuvor hörbare Zwitschern der Vögel ist verstummt und ein leichter Wind trägt kalte Luft zu dir, sodass du kurz ungewollt erschauderst. Für einen Moment sieht es so aus, als sei der Torbogen um die große hölzerne Tür aus Schädeln konstruiert. Als du erneut hinsiehst, erkennst du, dass es sich um kleine Skulpturen handelt, die schreckliche entstellte Fratzen abbilden. Viele der Mauersteine haben große Risse und Teile der Fassade sind abgesplittert.
Es ist schwer zu sagen, wie hoch der Turm wirklich ist, denn seine Spitze wird von einem Trichter aus dunklen Wolken verdeckt, die sich wie in einer Spirale langsam drehen.

Ohne anzuhalten fährt der Wagen auf die Tür zu, die wie von Geisterhand und mit hörbarem Ächzen nach links und rechts aufschwingt und Einlass gewährt. Im Inneren befindet sich ein kreisrunder Raum, der ungefähr 10 Meter im Durchmesser misst. Gegenüber des Eingangstors befindet sich eine weitere Holztür. Während du die spinnwebenverhangene Decke des Raums betrachtest, an der ein Kronleuchter aus schwarzem Metall hängt, rüttelt der Wagen kurz. Deine Entführer sind abgestiegen und nur Sekunden später wird dir ein Leinensack über den Kopf gezogen. Du hörst eine fremde Stimme krächzen: ``Sehr gut... in den Kerker...''. Kurz darauf wirst du hochgehoben.
\\Wenn du dich wehren willst, gehe zu [\ref{resistCell}].
\\Wenn du abwarten willst, gehe zu [\ref{arriveInCell}].

\block{resistCell}{Nicht ohne Widerstand}

Ein Kerker ist selten etwas Gutes. Du fängst an dich heftig zu wehren und schlägst mit den gefesselten Gliedmaßen um dich. Leider nur mit mäßigem Erfolg, doch immerhin bewegst du dich so ungünstig, dass dein Entführer nicht anders kann als dich fallen zu lassen. Du hörst wie er wütend sagt: ``Nun aber genug mit den Spielchen!''. Kurz darauf trifft dich etwas Hartes an der Schläfe und du verlierst das Bewusstsein.

Du verlierst 4 Lebenspunkte.

Gehe zu [\ref{wakeUpInCell}].

\block{arriveInCell}{Ohne Widerstand}

Du spürst, dass es kühler wird und ihr abwärts geht, vermutlich eine Treppe hinab. Modriger Geruch steigt dir in die Nase und du hörst das Quietschen alter Metallgitter. Wenige Sekunden später wirst du auf den Boden geworfen und der Leinensack wird entfernt.

Gehe zu [\ref{welcomeToTheCell}].

\block{arrivalAtTowerSingle}{Die verspätete Ankunft}

In der Ferne drückt sich ein mächtiger Turm aus schwarzem Stein wie ein Dorn in den Himmel. Allein der Anblick sorgt für ein mulmiges Gefühl in der Bauchgegend, was dadurch verstärkt wird, dass dieser Turm offensichtlich das Ziel der Reise ist. Dein verbleibender Entführer hat sich ab und an zu dir umgedreht und dich hasserfüllt angesehen, aber kein weiteres Wort verloren.

Je näher ihr dem Turm kommt, desto mehr beunruhigende Details kommen zum Vorschein. Die Bäume in der Nähe sind kahl und haben eine blässliche, tote Farbe angenommen. Das zuvor hörbare Zwitschern der Vögel ist verstummt und ein leichter Wind trägt kalte Luft zu dir, sodass du kurz ungewollt erschauderst. Für einen Moment sieht es so aus, als sei der Torbogen um die große hölzerne Tür aus Schädeln konstruiert. Als du erneut hinsiehst, erkennst du, dass es sich um kleine Skulpturen handelt, die schreckliche entstellte Fratzen abbilden. Viele der Mauersteine haben große Risse und Teile der Fassade sind abgesplittert. Es ist schwer zu sagen, wie hoch der Turm wirklich ist, denn seine Spitze wird von einem Trichter aus dunklen Wolken verdeckt, die sich wie in einer Spirale langsam drehen.

Ohne anzuhalten fährt der Wagen auf die Tür zu, die wie von Geisterhand und mit hörbarem Ächzen nach links und rechts aufschwingt und Einlass gewährt. Im Inneren befindet sich ein kreisrunder Raum, der ungefähr 10 Meter im Durchmesser misst. Gegenüber des Eingangstors befindet sich eine weitere Holztür. Während du die spinnwebenverhangene Decke des Raums betrachtest, an der ein Kronleuchter aus schwarzem Metall hängt, rüttelt der Wagen kurz. Dein Entführer ist abgestiegen und nur Sekunden später wird dir ein Leinensack über den Kopf gezogen. Eine fremde Stimme fragt krächzend: ``Ihr... seid spät. Wo ist Wladan?''. Du hörst ein leises Gemurmel als Antwort und die Reaktion des Fremden: ``Ein Schande... in den Kerker...''. Kurz darauf wirst du hochgehoben.
\\Wenn du dich wehren willst, gehe zu [\ref{resistCell}].
\\Wenn du abwarten willst, gehe zu [\ref{arriveInCell}].

\block{infiltrateTower}{Der Einbruch}

Nach knapp zwei weiteren Stunden Fußmarsch und mit dem Einbruch der Dämmerung drückt sich in der Ferne ein mächtiger Turm aus schwarzem Stein wie ein Dorn in den Himmel. Allein der Anblick sorgt für ein mulmiges Gefühl in der Bauchgegend, was dadurch verstärkt wird, dass dieser Turm offensichtlich das Ziel deiner Reise ist.

Je näher du ihm kommst, desto mehr beunruhigende Details kommen zum Vorschein. Die Bäume in der Nähe sind kahl und haben eine blässliche, tote Farbe angenommen. Das zuvor hörbare Zwitschern der Vögel ist verstummt und ein leichter Wind trägt kalte, modrige Luft zu dir, sodass du kurz ungewollt erschauderst. Du stehst nun vor der rechten Seite des Turms. Zu deiner linken kannst du das Eingangstor sehen. Für einen Moment sieht es so aus, als sei der Torbogen um die große hölzerne Tür aus Schädeln konstruiert. Als du erneut hinsiehst, erkennst du, dass es sich um kleine Skulpturen handelt, die schreckliche entstellte Fratzen abbilden. Viele der Mauersteine haben große Risse und Teile der Fassade sind abgesplittert.

Es ist schwer zu sagen, wie hoch der Turm wirklich ist, denn seine Spitze wird von einem Trichter aus dunklen Wolken verdeckt, die sich wie in einer Spirale langsam drehen. Du kannst in ungefähr 10 Metern Höhe ein Fenster erkennen. Am Fuß des Turms befinden sich, außer vor dem Tor, dichte Dornenbüsche. Es ist in der Dunkelheit schwer zu erkennen, aber die Steine des Turms sehen so aus, als könnte man die Wand mit etwas Erfahrung hinaufklettern.

Wenn du zum Tor gehen willst, gehe zu [\ref{invadeTheDoor}].
\\Wenn du es am Fenster probieren willst, gehe zu [\ref{upTheWindow}].
\\Wenn du deine Reise abbrechen und aufgeben willst, gehe zu [\ref{letsquitnow}].

\block{invadeTheDoor}{Ich nehme das Tor}

Du schätzt die Wand bei diesen Lichtverhältnissen als zu großes Risiko ein. Nein, man sieht der Gefahr offen ins Gesicht. Du läufst zum Tor. Die beiden Flügeltüren sind aus schwerem, dunklen Holz und von Kratzern übersäät. Auf jeder Seite hängt ein schwerer Ring im Mund einer metallenen Dämonenfratze.

Wenn du versuchen willst die Tür aufzuziehen, gehe zu [\ref{strengthTestDoor}].
\\Wenn du den Metallring schlagen willst, gehe zu [\ref{knockknockHereIAm}].
\\Wenn du es doch am Fenster probieren willst, gehe zu [\ref{upTheWindow}].
\\Wenn du deine Reise abbrechen und aufgeben willst, gehe zu [\ref{letsquitnow}].

\block{strengthTestDoor}{Ich öffne das Tor}

Du beschließt, dass Anklopfen keine Option ist. Nein, dieses Tor muss mit Gewalt geöffnet werden! Du drückst deine Schulter gegen das Holz und drückst mit aller Kraft!

Lege eine Athletikprobe mit DC 18 ab. Wenn du Erfolg hast, gehe zu [\ref{forceOpenDoor}].
Wenn nicht, gehe zu [\ref{doorStaysShut}].

\block{doorStaysShut}{Das Tor gewinnt}

Du mühst dich einige Minuten vergeblich ab, bevor du einsiehst, dass du hier nichts ausrichten kannst. Die Tür hat sich keinen Milimeter bewegt.

Wenn du den Metallring schlagen willst, gehe zu [\ref{knockknockHereIAm}].
\\Wenn du es doch am Fenster probieren willst, gehe zu [\ref{upTheWindow}].
\\Wenn du deine Reise abbrechen und aufgeben willst, gehe zu [\ref{letsquitnow}].

\block{forceOpenDoor}{Ich gewinne}

Du mühst dich einige Minuten ab und Milimeter für Milimeter bewegt sich das Holz ins Innere des Turms. Bald hast du einen ausreichend großen Spalt geöffnet, um hindurch zu schlüpfen.

Im Inneren befindet sich ein kreisrunder Raum, der ungefähr 10 Meter im Durchmesser misst. An den Wänden sind Fackeln in die Wand eingelassen und erhellen die Szenerie dürftig. Gegenüber des Eingangstors befindet sich eine weitere Holztür. An der spinnwebenverhangenen Decke des Raums befindet sich ein bedrohlich wirkender Kronleuchter aus schwarzem Metall.

Wenn du zur anderen Tür gehen willst, gehe zu [\ref{walkTowardsTheNextDoor}].
\\Wenn du zur anderen Tür schleichen willst, gehe zu [\ref{stealthilyGifford}].
\\Wenn du versuchen willst eine der Fackeln zu nehmen, gehe zu [\ref{grabLight}].

\block{grabLight}{Ich greife nach dem Licht}

Du betrittst den Raum und streckst dich, um eine der Fackeln zu erreichen. Plötzlich hörst du über dir ein lautes Knacken und das Rasseln einer Metallkette.

Gehe zu [\ref{someMonsterAttacks}].

\block{knockknockHereIAm}{Ich klopfe an}

Du nimmst den Eisenring und hämmerst ihn dreimal fest gegen das Tor. Der Hall deiner Schläge scheint durch das ganze Gemäuer zu hallen. Einige Sekunden lang geschieht nichts. Als du den Ring anfassen willst, um ihn erneut zu schlagen, schwingen die beiden Türen wie von Geisterhand auf. Im Inneren befindet sich ein kreisrunder Raum, der ungefähr 10 Meter im Durchmesser misst. An den Wänden sind Fackeln in die Wand eingelassen und erhellen die Szenerie dürftig. Gegenüber des Eingangstors befindet sich eine weitere Holztür. An der spinnwebenverhangene Decke des Raums befindet sich ein bedrohlich wirkender Kronleuchter aus schwarzem Metall.

``Welch unverhoffter Gast zu so später Stunde.'' hörst du eine Stimme auf der anderen Seite des Raums krächzen. Die Shilouette eines Mannes in einer langen, schwarzen Robe tritt aus dem Schatten. Bevor du antworten kannst, hebt der Mann seine Hand, in der er einen kleinen Stab hält. Als dieser beginnt bläulich zu leuchten, spürst wie dich eine plötzliche Erschöpfung überkommt.

Lege einen Rettungswurf auf Konstitution mit DC 18 ab. Wenn du erfolgreich bist, gehe zu [\ref{noSleepTillBrooklyn}].
\\Wenn du es nicht schaffst, gehe zu [\ref{mamaIwillSleep}].

\block{stealthilyGifford}{Ich schleiche zur Tür}

Ein leerer Raum und viele Spinnweben an der Decke? Keine Wachen? Das riecht förmlich nach einer Falle. Trotzdem willst du wissen, was sich hinter Tür Nummer Zwei verbirgt. Vorsichtig schleichst du durch den Raum.

Wirf eine Probe auf Schleichen mit DC 16. Wenn du Erfolg hast, gehe zu [\ref{stealthedThroughTheLobby}]. Wenn du es nicht schaffst, gehe zu [\ref{nopeNotSilentEnough}].

\block{nopeNotSilentEnough}{Ich habe etwas übersehen}

Behutsam tastest du dich Zentimeter um Zentimeter vorwärts. Leider übersiehst du einen kleinen Kieselstein, auf den du trittst. Das entstehende Schleifgeräusch ist alles andere als laut. Trotzdem sackt dir für einen Moment das Herz zusammen, als du plötzlich über dir ein tiefes Brüllen und das Rasseln einer Metallkette hörst.

Gehe zu [\ref{someMonsterAttacks}].

\block{walkTowardsTheNextDoor}{Ich gehe zur Tür}

Du betrittst den Raum und läufst zur gegenüber liegenden Tür. Plötzlich hörst du über dir ein tiefes Brüllen und das Rasseln einer Metallkette.

Gehe zu [\ref{someMonsterAttacks}].

\block{stealthedThroughTheLobby}{Ich erreiche die Tür}

Behutsam tastest du dich Zentimeter um Zentimeter vorwärts. Als du kurz ein Rascheln über dir hörst, hälst du inne und siehst vorsichtig zur Decke. Minutenlang harrst du so aus. Nachdem nichts passiert, setzt du vorsichtig deinen Weg fort und erreichst wenig später die Tür. Sie lässt sich problemlos und ohne Geräusche öffnen.

Gehe zu [\ref{TheCrossroom}].

\block{TheCrossroom}{Ich sehe Treppen}

Du betrittst einen kleinen, kreisrunden Raum, von dem zwei Treppen abgehen. Die rechte Treppe führt nach unten und die linke in die oberen Stockwerke. An der Decke hängt eine kleine Öllampe, die den sonst so kargen Raum spärlich beleuchtet.

Wenn du der Treppe nach unten folgen willst, gehe zu [\ref{stairsToTheDungeon}]. Wenn du die oberen Stockwerke betreten willst, gehe zu [\ref{stairsToTheUpperLevels}].

\block{mamaIwillSleep}{Ich bin sehr erschöpft}

Du versuchst gegen die Erschöpfung anzukämpfen, doch der Zauber ist zu stark. Bevor du einen weiteren Schritt machen kannst, spürst du wie deine Augen schwer werden und du zu Boden sinkst.

Wenn du \getItem{itemSimpleDagger}, \getItem{itemShortSword} und \getItem{GreaterHealingPotion} hast, verlierst du diese Gegenstände.

Als du wieder zu dir kommst, steigt dir ein modriger Geruch in die Nase. Es ist kalt und feucht und du bist wieder gefesselt. Du merkst sofort, dass du dich nicht mehr in der Eingangshalle befindest, denn der Boden ist weicher und es riecht nach Erde.

Gehe zu [\ref{welcomeToTheCell}].

\block{goToCrossroom}{Richtungswechsel}

Da du keinen Weg siehst die verschlossene Tür zu öffnen, läufst du die Treppe wieder hinunter. Du kommst wieder in den kleinen Zwischenraum neben der Eingangshalle.

Wenn du in die unteren Stockwerke gehen möchtest, gehe zu [\ref{stairsToTheDungeon}]. Wenn du zurück in die Eingangshalle willst, um den Turm zu verlassen, gehe zu [\ref{doorToLobby}].

\block{noSleepTillBrooklyn}{Keine Zeit für Schlaf}

Du versuchst gegen die Erschöpfung anzukämpfen. Für einen Moment glaubst du, dass du es nicht schaffen wirst und deine Augen sinken schwer nach unten. Du nimmst all deine Kraft zusammen und reißt sie wieder auf! Du siehst, wie der Mann erschrocken die Hand sinken lässt. Er ruft: ``Ah es wirkt nicht! Schnapp ihn dir!'' und dreht sich um, um durch die entfernte Tür zu gehen.

Du willst ihm folgen, als du plötzlich über dir ein lautes Knacken und das Rasseln einer Metallkette hörst.

Gehe zu [\ref{someMonsterAttacks}].

\block{upTheWindow}{Ich nehme das Fenster}

Es wird sicher kein Kinderspiel diese Wand zu erklimmen, aber es sieht auch nicht unmöglich aus. Allerdings musst du dir vorher überlegen, wie du durch die Dornenbüsche kommst.

Wenn du hindurchkriechen willst, gehe zu [\ref{dornyBushes}].
\\Wenn du \getItem{itemShortSword} hast, kannst du zu [\ref{dornyBushesWithSword}] gehen.

\block{dornyBushes}{Meine Haut ist wie Leder}

Du fürchtest dich doch nicht vor ein paar Kratzern. Natürlich kannst du nicht einfach durch die Büsche laufen, aber am Boden sieht es einfacher aus.

Wenige Sekunden später steckst du tief in den Dornenbüschen. Es ist nicht mehr weit bis zur Wand, doch deine Haut leidet deutlich mehr als du gedacht hättest. Durch die ganzen Kratzer und Schnitte verlierst du 2 Lebenspunkte. Aber du kommst an der Außenwand des Turms an.

Gehe zu [\ref{climbTheWall}].

\block{dornyBushesWithSword}{Ich schneide mich durch}

Du fürchtest dich nicht vor ein paar Kratzern. Aber warum sollst du auf dem Boden kriechen, wenn du ein Schwert zur Hand hast? Schlag um Schlag kämpfst du dich durch den Busch und schneidest dir deinen Weg frei. Schließlich kommst du an der Außenwand des Turms an.

Gehe zu [\ref{climbTheWall}].

\block{climbTheWall}{Ich habe eine Wand vor mir}

Wie du vermutet hattest, weist die Wand größere Sprünge und abgebrochene Stellen auf. Das ist gut für dich. Prüfend legst du deine Hand auf einen der Vorsprünge und ziehst dich ein Stück aufwärts. Die Steine tragen dein Gewicht, nun gilt es nur noch einen guten Pfad zum Fenster zu finden.

Wirf eine Akrobatikprobe mit DC 14. Wenn du erfolgreich bist, gehe zu [\ref{reachingTheWindow}]. Wenn du es nicht schaffst, gehe zu [\ref{fallingFromTheWall}].

\block{fallingFromTheWall}{Die Schwerkraft ist gegen mich}

Du kommst dem Fenster immer näher. Ein kurzer Kontrollblick zeigt dir, dass du dich bereits einige Meter über dem Boden befindest, als du plötzlich ein seltsames Gefühl im Finger hast. Gerade noch rechtzeitig kannst du dein Gewicht zur anderen Hand verlagern, als ein Teil des Steins, an dem du dich gehalten hast, aus der Wand bricht. Du kannst dich halten, aber dein gedachter einfacher Kletterpfad zum Fenster ist damit Geschichte. Verzweifelt siehst du dich nach einer Alternative um. Und entdeckst sie.

Wenn du weiterklettern willst, gehe zu [\ref{dontStopMeNow}]. Wenn dir das Risiko zu groß ist und du dich lieber an den Abstieg machen willst, gehe zu [\ref{returnFromTheWall}].

\block{dontStopMeNow}{Ich bin kurz vor dem Ziel}

Du kannst schon fast nach dem Fenstersims greifen, warum solltest du jetzt aufgeben? Am Ende rutschst du beim Abstieg ab. Nein, für dich geht es weiter aufwärts!

Du schaffst die fehlenden Meter und hast deine Hand am Fenster, als dein rechter Fuß unerwartet den Halt verliert. An einem Arm hängend suchst du verzweifelt nach einem anderen Griff, doch die Zeit reicht nicht. Deine Finger verkrampfen und du wirst Richtung Erdboden gerissen. Dein Aufprall ist unfassbar schmerzhaft, es hilft auch nicht, dass die Dornenbüsche die Wucht etwas abmildern. Du verlierst 19 Lebenspunkte.

Stöhnend windest du dich aus dem Gestrüpp. Deine Gelenke schmerzen und dein Rückrat ist heftig getroffen. Du hast dir vermutlich einige Rippen und wer weiß was noch gebrochen. Du kannst auf keinen Fall nochmal versuchen zum Fenster zu klettern.

Wenn du zum Tor gehen willst, gehe zu [\ref{invadeTheDoor}].
Wenn du deine Reise abbrechen und aufgeben willst, gehe zu [\ref{letsquitnowiamhurt}].

\block{returnFromTheWall}{Mein Leben ist das Risiko nicht wert}

Du entschließt dich das Schicksal an dieser Stelle nicht weiter herauszufordern. Dass du dich eben noch halten konntest war ein Zeichen. Behutsam machst du dich an den Abstieg und bist wenige Minuten später wieder auf sicherem Boden. Erschöpft, aber unverletzt.

Wenn du \getItem{itemShortSword} hast, kannst du zu [\ref{icutthedornyBushesWithSword}] gehen. Wenn nicht, gehe zu [\ref{dornyBushesAgain}].

\block{letsquitnowiamhurt}{Ich kann so nicht weitermachen}

In deiner aktuellen Verfassung wäre es blanker Selbstmord an eine Fortsetzung des Abenteuers zu denken. Du bist verletzt, besitzt keine richtige Ausrüstung und bist eigentlich auch nicht vorbereitet. Warum solltest du ein solches Risiko eingehen?

Das macht keinen Sinn und du hast im Laufe der Jahre gelernt Situationen zu vermeiden, die keinen Sinn machen. Nein, es wird Zeit nach Hause zu gehen. Humpelnd kehrst du zurück in den Wald und machst dich auf den qualvollen Heimweg. Eine warme Suppe, ein Bett und ein Besuch bei einem Heiler warten auf dich.

\textbf{Ende.}

\block{dornyBushesAgain}{Ich bin voller Kratzer}

Du bist vielleicht unverletzt wieder auf den Boden gekommen, aber der Blick zurück ruft dir die Dornenbüsche wieder ins Gedächtnis. Dir bleibt nichts anderes übrig, als diese Tortur erneut über dich ergehen zu lassen. Erneut steckst du tief in den Dornenbüschen. Es ist genauso schmerzhaft wie beim ersten Mal, wieder verlierst du 2 Lebenspunkte und eine Menge neuer Kratzer bleiben als Andenken.

Wenn du zum Tor gehen willst, gehe zu [\ref{invadeTheDoor}].
Wenn du deine Reise abbrechen und aufgeben willst, gehe zu [\ref{letsquitnow}].

\block{icutthedornyBushesWithSword}{Ich wandle auf alten Pfaden}

Da du die Dornenbüsche bereits so zerhackt hast, dass ein bequemer Weg frei wurde, kannst du problemlos wieder hindurchlaufen.
Wenn du zum Tor gehen willst, gehe zu [\ref{invadeTheDoor}].
Wenn du deine Reise abbrechen und aufgeben willst, gehe zu [\ref{letsquitnow}].

\block{letsquitnow}{Ich habe es mir anders überlegt}

Jetzt, wo du vor dem Turm stehst, kommt dir das ganze doch nicht mehr sonderlich schlau vor. Nein, eigentlich überhaupt nicht. Warum solltest du jetzt, ohne richtige Ausrüstung und Vorbereitung, ein solches Risiko eingehen?

Das macht keinen Sinn und du hast im Laufe der Jahre gelernt Situationen zu vermeiden, die keinen Sinn machen. Nein, es wird Zeit nach Hause zu gehen. Du kehrst zurück in den Wald und machst dich auf den mühevollen Heimweg. Eine warme Suppe und ein Bett warten auf dich.

\textbf{Ende.}

\block{someMonsterAttacks}{Ich sehe nach oben}

Markiere Ereignis \setEvent{SpiderAttacks}. \\Das Rasseln der Kette lässt dich nach oben blicken. Dein Magen zieht sich zusammen als du erkennst, dass dort gar kein Kronleuchter hing sondern eine \textbf{riesige Wolfspinne}, die sich gerade zu dir herablässt! Ein Kampf ist unvermeidbar...

\monsterSpider{26}{poisonedAndOut}

Wenn du die Spinne getötet hast, gehe zu [\ref{diespiderdie}].
\\Wenn die Spinne dich tötet, gehe zu [\ref{theSpiderKilledMe}].

\block{poisonedAndOut}{Ich spüre meine Finger nicht mehr}

Du schlägst dich wacker gegen die Spinne. Leider scheint sich ihr Gift immer weiter durch deinen Körper auszubreiten. Du verlierst langsam das Gefühl in deinen Fingern und deine Sicht verschwimmt.
Kurze Zeit später versagen deine Beine ihren Dienst und du kippst hilflos nach vorn. Immerhin musst du nicht miterleben, wie du gefressen wirst, denn du verlierst das Bewusstsein...

Du verlierst alle gesammelten Gegenstände.

Gehe zu [\ref{wakeUpInCellAfterSpider}].

\block{wakeUpInCellAfterSpider}{Ich lebe}

Der stechende Schmerz in deinem Nacken verrät dir, dass du nicht gefressen wurdest. Als du langsam wieder zu Sinnen kommst, steigt dir ein modriger Geruch in die Nase. Es ist kalt und feucht und du bist schon wieder gefesselt. Offensichtlich befindest du dich nicht mehr in der Eingangshalle des Turms, denn der Boden ist weicher und riecht nach Erde.

Gehe zu [\ref{welcomeToTheCell}].

\block{wakeUpInCell}{Unsanftes Erwachen}

Als du wieder zu dir kommst, steigt dir ein modriger Geruch in die Nase. Es ist kalt und feucht und du bist noch immer gefesselt. Offensichtlich liegst du nicht mehr auf dem Wagen, denn der Boden ist weicher und riecht nach Erde.

Gehe zu [\ref{welcomeToTheCell}].

\block{theSpiderKilledMe}{Ich bin leider köstlich}

Du lieferst dir den Kampf deines Lebens mit der Spinne. Leider ist es auch der letzte Kampf deines Lebens. Du weichst Biss um Biss der Kreatur aus, doch für einen kleinen Moment bist du unaufmerksam. Sofort bohren sich die dicken Zangen der Spinne in deinen Körper. Aus dieser Situation gibt es kein Entkommen.

Keine überraschende Wendung. Kein Happy End. Immerhin nimmt dir das Gift der Spinne langsam die Sinne. Der Schmerz lässt nach und eine tiefe Müdigkeit überkommt dich. Die Götter sind gnädig und lassen dich nicht miterleben, wie du gefressen wirst.

\textbf{Ende.}

\block{diespiderdie}{Ich hasse Spinnen}

Du schaffst es die Spinne zu erschlagen! Mit einem jämmerlichen Quieken weicht das Tier zurück, doch seine Wunden sind zu schwer. Es zuckt noch ein paar Mal, bevor es die hässlichen, behaarten Beine an den mächtigen Leib zieht und zusammensackt. Dieses Vieh wird niemandem mehr Probleme machen!

Während du schwer atmest siehst du dich im Raum um. Nach einigen Sekunden ist klar, dass du allein bist. Du kannst nur die Holztür gegenüber des Eingangstors erkennen, ansonsten ist der Raum leer. Dein Abenteurerinstinkt sagt dir außerdem, dass du dir keine Hoffnung machen musst bei dem Kadaver irgendetwas wertvolles zu finden.

Markiere Ereignis \setEvent{ikilledthedamnspiderMarker}.

Wenn du den Turm verlassen und nach Hause gehen möchtest, gehe zu [\ref{spidersmakemequit}].
\\Wenn du durch die Tür gehen willst, gehe zu [\ref{TheCrossroom}].

\block{spidersmakemequit}{Ich bin ziemlich kaputt}

Es ist keine Schande ein Abenteuer abzubrechen. Das unterscheidet einen guten Abenteurer von einem toten Abenteurer. Du hast dich tapfer geschlagen, aber du kennst deine Grenzen. Es wird Zeit nach Hause zu gehen, vielleicht ist das die letzte Chance dafür.

Ohne einen weiteren Gedanken an das Innere des Turms oder den Hexer zu verschwenden gehst du durch das Tor und machst dich auf den Weg zurück nach Greifenheim.

Eine warme Suppe und ein Bett warten auf dich.

\textbf{Ende.}

\block{welcomeToTheCell}{Ich bin eingesperrt}

Es dauert einen Moment, bis sich deine Augen an die schlechten Lichtverhältnisse gewöhnt haben. Du befindest dich in einem kleinen Raum, ohne Fenster. Die Tür besteht aus einem dicken Eisengitter, das zwar alt, aber sehr robust wirkt. Von der Decke tropft Wasser und hat auf dem weichen Erdboden eine kleine Pfütze gebildet. An der Wand findest du einige kleine Knochen, von denen du nicht sagen kannst zu welchem Lebewesen sie vorher gehört hatten.

Wenn du dich in die Ecke setzen und warten willst, gehe zu [\ref{waitingForDestinyToDoSomething}].
\\Der Boden sieht locker aus. Wenn du versuchen willst dich unter den Gitterstäben hindurchzugraben, gehe zu [\ref{schauflerGo}].
\\Wenn du gegen die Stäbe schlagen willst, gehe zu [\ref{hitYourBars}].
\\Wenn du versuchen willst aus den Knochen Werkzeug herzustellen, gehe zu [\ref{craftingBones}].

\block{craftingBones}{Das Schlüsselbein}

Du durchsuchst die Knochen erneut und findest ein Exemplar, der mit etwas Hilfe vermutlich eine passable Waffe abgeben würde. Andererseits könntest du auch dein Glück auf die Probe stellen und versuchen mit dem Knochen das Schloss zu knacken.

Wenn du den Knochen als Waffe benutzen willst, gehe zu [\ref{boneDagger}].
\\Wenn du versuchen willst das Schloss zu knacken, gehe zu [\ref{crackTheLockWithBones}].

\block{boneDagger}{Ich habe spitze Knochen}

Die reibst den Knochen an der Wand, bis eine Seite eine Spitze bildet. Du kannst zufrieden mit dir sein, im Zweifelsfall ist der Knochen eine passable Waffe.
Du erhälst einen Knochendolch \getItem{itemBoneDagger}.

Wenn du dich in die Ecke setzen und warten willst, gehe zu [\ref{waitingForDestinyToDoSomething}].
\\Der Boden sieht locker aus. Wenn du versuchen willst dich unter den Gitterstäben hindurchzugraben, gehe zu [\ref{schauflerGo}].
\\Wenn du gegen die Stäbe schlagen willst, gehe zu [\ref{hitYourBars}].

\block{hitYourBars}{Ich halte das nicht aus}

Verzweifelt und wütend schlägst du gegen die schweren Gitterstäbe der Zellentür. Das dumpfe Geräusch hallt für einen Moment durch die Gänge, dann kehrt Ruhe ein. Du versuchst es erneut, diesmal wirfst du dich mit deiner Schulter gegen die Tür, doch vergebens. Seufzend lässt du dich auf den Boden sinken und sitzt mit dem Rücken an der Tür.

Dann hörst du ein Klacken. Du drehst dich um und versuchst ganz still zu sein. Da, da ist es wieder. Es kann nicht der Hall deiner Schläge sein. Als wieder Stille einkehrt, schlägst du erneut gegen deine Zellentür. Und erhälst eine Antwort. Irgendjemand... oder irgendetwas ist hier auch eingesperrt und hat dich gehört. Und schlägt nun selbst gegen die Gitterstäbe.

Bevor du überlegen kannst, wie du das zu deinem Vorteil nutzen kannst, hörst du Schritte, die sich nähern. Zuerst leise, dann immer lauter, bis jemand vor deinem Gefängnis auftaucht. Instinktiv machst du einen Schritt zurück. Mit einer simplen Handbewegung wird das Schloss deiner Tür quietschend geöffnet.

Gehe zu [\ref{meetTheWarlock}].

\block{schauflerGo}{Ich grabe mich frei}

Der Boden der Zelle ist wirklich recht locker. Du fängst an zu graben, in der Hoffnung vielleicht unter der Tür genügend Platz für dich machen zu können. Zu deiner Enttäuschung zeigt sich schnell, dass die Steinwände unter der Erde weiterführen. Und sie sind robust. Dafür machst du eine andere Entdeckung. In der Erde findest du, in einen Stofffetzen gewickelt, einen silbernen Ring \getItem{itemBurialRing}.

Bevor du überlegen kannst, wie du das zu deinem Vorteil nutzen kannst, hörst du Schritte, die sich nähern. Schnell schiebst du dein Erdloch wieder zu und steckst dir den Ring auf den linken Ringfinger. Zuerst nähern sich die Schritte leise, dann immer lauter, bis jemand vor deinem Gefängnis auftaucht. Instinktiv machst du einen Schritt zurück. Mit einer simplen Handbewegung wird das Schloss deiner Tür quietschend geöffnet.

Gehe zu [\ref{meetTheWarlock}].

\block{waitingForDestinyToDoSomething}{Ich übe mich in Geduld}

Du setzt dich in die Ecke und wartest geduldig ab. Das langsame Tropfen des Wassers von der Decke hat beinahe etwas hypnotisches. Irgendwann werden deine Augen schwer und du schläfst ein.

Als du das Quietschen der Tür hörst, schreckst du hoch. Gehe zu [\ref{meetTheWarlock}].

\block{meetTheWarlock}{Ich treffe meinen Gastgeber}

In der Zellentür steht ein Mann, für den die Beschreibung ``Hexer'' förmlich erfunden wurde. Das hagere, eingefallene Gesicht ist von tiefen Augenringen gekennzeichnet, die bei der blassen Haut besonders auffallen. Auch die weite Robe kann nicht verbergen, welch knochiger Körper sich unter ihr befindet. Mit langsamen, beinahe etwas wackligen Schritten kommt er auf dich zu. Seine Gestalt und sein Gang stehen im starken Kontrast zu seinem Blick. Die giftgrünen Augen des Hexers scheinen in der Dunkelheit zu leuchten. Du hast das Gefühl, dass er durch dich hindurch sieht, mitten in deine Seele hinein.

``So, so...'', krächzt er, ``...wie ich sehe hast du dich bereits an dein neues Heim gewöhnt. Doch keine Sorge, das soll nicht von Dauer sein. Ich wäre ein schlechter Gastgeber, wenn ich mich nicht um meine Gäste... kümmern würde, nicht wahr?''. In seiner Hand erkennst du einen kleinen Zauberstab, der anfängt gelblich zu leuchten. Langsam spürst du eine Kraft auf dich einwirken, die deinen Körper bewegen will.

Wirf einen Rettungswurf auf Charisma mit DC 18 oder mit DC 17, wenn du \getItem{itemBurialRing} besitzt. Wenn du erfolgreich bist, gehe zu [\ref{youShallNotControlMe}]. Wenn du es nicht schaffst, gehe zu [\ref{yesMasterControlMe}].

\block{yesMasterControlMe}{Ich habe keine Kontrolle}

Keine Fremde Macht soll dich kontrollieren! Angestrengt versuchst du den Angriff auf deinen Geist abzuwehren. Wie in Wellen prallt es auf dich ein und versucht Stück für Stück dir die Sinne zu rauben. Du hast das Gefühl tausende Stimmen zu hören, die direkt in deinem Kopf flüstern und dir befehlen einfach aufzugeben. Zuerst verspürst du ein seltsames Kribbeln in den Fingern, dann verschwimmt für einen Moment deine Sicht. Du versuchst die Stimmen aus deinem Kopf zu verbannen, doch vergebens.

Deine Muskeln entspannen sich. Als der Hexer dich zufrieden ansieht, sich umdreht und sagt ``Folge mir.'' hast du keine Wahl. Du fühlst dich wie ein Passagier in deinem eigenen Körper, der anfängt sich zu bewegen und dem Mann folgt.

Ihr verlasst die Zelle und nehmt eine Treppe um den Kerker zu verlassen. Einige Stufen später kommt ihr an einer Tür vorbei, die der Hexer jedoch ignoriert und die Treppe weiter emportsteigt. Du folgst ihm weiter, Stufe um Stufe.

Gehe zu [\ref{theUpperLevelWithWarlock}].

\block{youShallNotControlMe}{Ich bin mein eigener Herr}

Keine Fremde Macht soll dich kontrollieren! Angestrengt versuchst du den Angriff auf deinen Geist abzuwehren. Wie in Wellen prallt es auf dich ein und versucht Stück für Stück dir die Sinne zu rauben. Du hast das Gefühl tausende Stimmen zu hören, die direkt in deinem Kopf flüstern und dir befehlen einfach aufzugeben. Zuerst verspürst du ein seltsames Kribbeln in den Fingern, dann verschwimmt für einen Moment deine Sicht. Doch du bist nicht bereit aufzugeben! Endlos scheinende Sekunden drängst du zurück, bis du plötzlich Klarheit verspürst. Das Wispern hört auf. Du bist wieder allein in deinem Kopf.

Der Magier scheint noch nicht bemerkt zu haben, dass sein Zauber fehlgeschlagen ist. Das könnte deine große Chance sein.

Wenn du \getItem{itemBoneDagger} hast, kannst du zu [\ref{iGotMeinDagger}] gehen.
\\Wenn du so tun willst als hätte der Zauber Erfolg gehabt, gehe zu [\ref{yesMasterControlMeButInTruthYouDont}].
\\Wenn du den Hexer angreifen willst, gehe zu [\ref{fightMeWarlock}].

\block{iGotMeinDagger}{Man sieht den Knochen}

Du siehst die Überraschung in den Augen des Hexers, als ihm klar wird, dass du seinem Zauber widerstanden hast. Innerhalb eines Wimpernschlags hast du nach deinem improvisierten Knochendolch gegriffen und stichst mit aller Kraft zu. Aus dem überraschten Ausdruck deines Feindes ist eine Fratze der Angst geworden.
Wirf eine Angriffsprobe gegen AC 12. Wenn du erfolgreich bist, gehe zu [\ref{dieWarlockDie}]. Wenn du verfehlst, gehe zu [\ref{iMissedTheWarlock}].

\block{dieWarlockDie}{Herzschmerz}

Der Hexer öffnet den Mund um zu schreien, doch es ist zu spät. Mit aller Gewalt rammst du ihm den Knochen in die Brust. Deine Waffe splittert dabei, doch das atemlose Ächzen deines Feindes sagt dir, dass du nicht verfehlt hast. Ungläubig starrt der Mann auf die Waffe hinunter und hebt zitternd die Hände, um sie zu entfernen. Doch so weit kommt er nicht. Während du zusiehst kehrt sich das Weiße in seinen Augen nach oben und er sinkt auf die Knie. Noch bevor sein Gesicht im feuchten Erdboden aufschlägt hört er auf zu zucken und bleibt reglos liegen. Er ist tot.

Die Stille in der Zelle wird nur das Tropfen des Wassers von der Decke durchbrochen. Wenn du den Kerker auf schnellstem Weg verlassen willst, gehe zu [\ref{letsLeaveTheDungeon}].
\\Wenn du den Hexer durchsuchen willst, gehe zu [\ref{lootTheWarlock}].
\\Wenn du dich im Kerker umsehen willst, gehe zu [\ref{inspectTheDungeon}].

\block{crackTheLockWithBones}{Das wird ein Knochenjob}

Du schnappst dir den Knochen und einen zweiten, kleineren Knochen und fängst an dich an dem Schloss zu versuchen. Normalerweise wärst du ohne Chance gewesen, aber das Schloss an der Tür ist so alt und grobschlächtig, dass dein Plan tatsächlich Früchte tragen könnte! Du kannst den Riegel, den du umschieben musst, mit bloßem Auge erkennen. Die Frage ist nur, ob du es schaffst ihn mit den spröden Knochen zu verschieben, bevor sie brechen.

Wirf eine Probe auf Fingerfertigkeit mit DC 15. Wenn du erfolgreich bist, gehe zu [\ref{ipickedthelock}]. Wenn du es nicht schaffst, gehe zu [\ref{thelockbrokethebones}].

\block{thelockbrokethebones}{Meine Knochen brechen}

Du bist mehrmals kurz davor, doch irgendwie fehlt dir immer der letzte Zentimeter um den Riegel endlich aufzuschieben. Wenn du die Knochen noch weiter in das Schloss schiebst, könnten sie brechen, aber so fehlt dir einfach ein Stück. Ein kurzer Blick zu den anderen Knochen zeigt dir, dass diese völlig unbrauchbar sind für diese Aufgabe. Also setzt du alles in deinen nächsten Versuch und drückst den großen Knochen noch etwas tiefer in das Schloss.

Du verfehlst dein Ziel. Als wärst du mit deiner aktuellen Situation nicht schon gestraft genug, bricht der Knochen auch noch und du hast einen Haufen kleiner Splitter in der Hand, die dir nichts mehr Nützen.

Während du überlegst, ob du noch einen anderen aus dieser Lage findest, hörst du Schritte, die sich nähern. Zuerst leise, dann immer lauter, bis jemand vor deinem Gefängnis auftaucht. Instinktiv machst du einen Schritt zurück. Mit einer simplen Handbewegung wird das Schloss deiner Tür quietschend geöffnet.

Gehe zu [\ref{meetTheWarlock}].

\block{ipickedthelock}{Ich höre den Riegel umschlagen}

Du hast bereits einige Minuten und Fehlversuche hinter dir, als du endlich das Klacken des Riegels hören kannst, der zurückschwingt. Und das keinen Versuch zu früh, als du die Knochen wieder aus dem Schloss ziehst, zerbrechen sie in kleine Splitter. Doch das ist jetzt unwichtig, denn die Tür ist offen und deiner Flucht steht nichts mehr im Weg.

Wenn du den Kerker auf schnellstem Weg verlassen willst, gehe zu [\ref{letsLeaveTheDungeon}].
\\Wenn du dich im Kerker umsehen willst, gehe zu [\ref{inspectTheDungeon}].

\block{yesMasterControlMeButInTruthYouDont}{Ich bin euch zu Diensten}

Deine Muskeln entspannen sich. Du überlegst kurz, was für eine Art Zauber der Hexer wohl auf dich angewandt hast und was du tun sollst. Als du jedoch das zufriedene Gesicht des Mannes siehst, bist du dir sicher, dass du ihn täuschen konntest. Er dreht sich um und sagt ``Folge mir.''. Du beschließt das Spiel weiter mitzuspielen. Schweigend setzt du einen Fuß vor den anderen und folgst den Hexer.

Ihr verlasst die Zelle und nehmt eine Treppe um den Kerker zu verlassen. Einige Stufen später kommt ihr an einer Tür vorbei, die der Hexer jedoch ignoriert und die Treppe weiter emportsteigt. Du folgst ihm weiter, Stufe um Stufe.

Markiere Ereignis \setEvent{deceivedTheWarlock}. Gehe zu [\ref{theUpperLevelWithWarlock}].

\block{lootTheWarlock}{Was dein war ist mein}

Du zögerst keine Sekunde und durchsuchst den Toten nach Dingen, die dir helfen könnten.
Kurze Zeit später kannst du einen Zauberstab \getItem{warlockStaff}, einen geschwungenden Dolch \getItem{warlocksDagger}, einen kleinen Kupferschlüssel \getItem{warlocksKey} und einen goldenen Ring \getItem{goldenWarlockRing} dein Eigen nennen.

Außerdem ist dir aufgefallen, dass der linke Arm des Mannes von seltsamen Tätowierungen übersät ist, die dir wie Runen vorkommen. Viele kleine Narben und Schnitte durchbrechen die Muster, einige scheinen noch recht frisch und gerade erst verheilt zu sein.

Wenn du den Kerker auf schnellstem Weg verlassen willst, gehe zu [\ref{letsLeaveTheDungeon}].
\\Wenn du dich weiter im Kerker umsehen willst, gehe zu [\ref{inspectTheDungeon}].

\block{iMissedTheWarlock}{Ich verfehle mein Ziel}

Kurz bevor du dem Schurken den Knochen in die Brust rammen kannst, stolpert dieser und fällt ein Stück nach hinten. Es sind nur wenige Zentimeter, doch sie reichen aus, um deinem Angriff zu entgehen. Gehe zu [\ref{fightMeWarlock}].

\block{fightMeWarlock}{Ich gebe nicht auf}

Der Hexer mag deinem Angriff entgangen sein, doch das heißt nicht, dass du verloren hast! Sofort setzt du zum nächsten Angriff an. Leider ist dein Gegner nun gewarnt. In der kleinen Zelle kann keiner von euch beiden wegrennen und sein wütender Blick verrät dir, dass das auch ihm klar ist. Mit der freien Hand zieht er einen Dolch. Nur einer von euch beiden wird diesen Raum lebend verlassen.

\monsterWarlock

Wenn du den Hexer umbringen kannst, gehe zu [\ref{dieWarlockDie}].
Wenn du den Kampf verlierst, gehe zu [\ref{anEvilWarlockKilledMe}].

\block{anEvilWarlockKilledMe}{Ich habe keine Chance}

Du gibst alles, doch das reicht nicht. Als dir der Hexer seinen Dolch in die Rippen rammt, taumelst du etwas benommen zurück. Deine Hände packen die Klinge, um sie herauszuziehen, doch der stechende Schmerz lähmt deine Bewegung. Als du nach oben siehst, kannst du das verächtliche Grinsen des Hexers sehen, der seinen Zauberstab vor deine Augen hält. Während die Spitze anfängt in einem giftgrünen Licht zu erstrahlen, hörst du ihn sagen: ``Du unwürdiger Wurm...''

Als das grüne Licht durch deinen Körper fährt, verkrampfen deine Muskeln und ein unbeschreiblicher Schmerz schmettert in deinen Kopf. Dann wird alles um dich dunkel.

\textbf{Ende.}

\block{letsLeaveTheDungeon}{Ich verschwinde so schnell es geht}

Du verschwendest keinen zweiten Gedanken an den Hexer und rennst aus deiner Zelle, vorbei an einigen anderen Zellen, die du dir nicht ansiehst. Es ist nicht schwer die Treppe zu finden und kurze Zeit später bist du ein Stockwerk höher. Du befindest dich einem kleinen Raum. Hinter dir befindet sich die Treppe hinab in den Kerker, vor dir eine weitere Treppe, die höher führt. Auf der linken Seite siehst du eine Holztür.

Wenn du der Treppe folgen willst, gehe zu [\ref{stairsToTheUpperLevels}].
\\Wenn du durch die Tür gehen willst, gehe zu [\ref{doorToLobby}].

\block{doorToLobby}{Ich nehme die Tür}

Ohne zu Zögern gehst du zur Holztür und öffnest sie. Vor dir liegt die Eingangshalle des Turms, direkt gegenüber befindet sich das große Eingangstor, das geschlossen ist. Wenn Ereignis \getEvent{SpiderAttacks} bereits eingetreten ist, gehe zu [\ref{iKnowAboutTheSpider}]. Wenn das Ereignis \getEvent{ikilledthedamnspiderMarker} bereits eingetreten ist, gehe zu [\ref{iKnowAboutTheSpiderBecauseIKilledIt}].
Wenn beides nicht zutrifft, gehe zu [\ref{iHaveNoIdeaAboutTheSpider}].

\block{iHaveNoIdeaAboutTheSpider}{Ich gehe zum Tor}

Mit eiligen Schritten durchquerst du den Raum, um zum Tor zu gelangen. Plötzlich hörst du über dir ein lautes Knacken und das Rasseln einer Metallkette.
Das Geräusch lässt dich nach oben blicken. Dein Magen zieht sich zusammen als du erkennst, dass dort gar kein Kronleuchter hängt sondern eine \textbf{riesige Wolfspinne}, die sich gerade zu dir herablässt! Ein Kampf ist unvermeidbar...

\monsterSpider{26}{poisonedAndDying}

Wenn du die Spinne getötet hast, gehe zu [\ref{ikilledthedamnspider}].
\\Wenn die Spinne dich tötet, gehe zu [\ref{theSpiderKilledMe}].

\block{iKnowAboutTheSpider}{Mein Spinnensinn meldet sich}

Natürlich hast du die riesige Spinne noch nicht vergessen. Doch du kannst kein Zeichen des riesigen Monsters an der spinnwebenverhangenen Decke erkennen. Vorsichtig versuchst du am Rand des Raums entlang zu schleichen, um einer zweiten Konfrontation aus dem Weg zu gehen. Wirf eine Schleichenprobe mit DC 16. Wenn du erfolgreich bist, gehe zu [\ref{stealthToTheMainDoor}].
Wenn du es nicht schaffst, gehe zu [\ref{theSpiderKnowsWhereIAm}].

\block{iKnowAboutTheSpiderBecauseIKilledIt}{Acht Beine zum Himmel}

Vor dir liegt der monströse Kadaver der Wolfsspinne, deren Leben du so grausam beendet hast. Allerdings ist das keine Situation für Mitleid, sie hätte dich mit Sicherheit ohne Zögern gefresse wenn du ihr die Chance gegeben hättest. Zumindest musst du dir jetzt keine Gedanken mehr um das Untier machen.

Du gehst ungehindert zum geschlossenen Tor und ziehst mit aller Kraft. Zentimeter um Zentimeter öffnet sich der Weg in die Freiheit. Als der Spalt groß genug ist um bequem hindurch zu schlüpfen verlässt du den Turm.

Gehe zu [\ref{longRoadHome}].

\block{stealthToTheMainDoor}{Ich schwebe über den Boden}

Behutsam bewegst du dich Zentimeter um Zentimeter zur Tür, die Decke des Raums immer im Blick. Dabei übersiehst du fast einen kleinen Kieselstein, kannst aber gerade noch rechtzeitig anhalten. Nervös wandert dein Blick zu den Spinnennetzen, doch du kannst keinerlei Bewegung erkennen.
Du setzt deinen Weg fort und bist wenig später am Tor angekommen.

Um die Tür zu öffnen wirst du kräftig ziehen müssen. Dir ist klar, dass das Geräusche machen wird. Wirf eine Athletikprobe mit DC 14. Wenn du Erfolg hast, gehe zu [\ref{escapeTroughTheMainDoor}]. Schaffst du es nicht, gehe zu [\ref{noEscapeFromSpiderEvenWithStealth}].

\block{noEscapeFromSpiderEvenWithStealth}{Ich bin nicht schnell genug}

Du mobilisierst alle Kraftreserven, die dir noch bleiben und die Tür öffnet sich knarzend Milimeter um Milimeter. Das entstehende Schleifgeräusch ist alles andere als laut. Trotzdem sackt dir für einen Moment das Herz zusammen, als ein Rascheln von der Decke hörst. Du wurdest entdeckt! Hinter dir hörst du ein dumpfes Geräusch. Mit Sicherheit ist die Spinne gerade auf dem Boden aufgekommen.

Die Adern an deinem Hals treten hervor, als du versuchst die Tür schneller zu öffnen.
Leider bist du trotzdem zu langsam, die Spinne fällt dir in den Rücken und greift dich an!

Gehe zu [\ref{theSpiderAgain}].

\block{theSpiderKnowsWhereIAm}{Erschütternde Neuigkeiten}

Behutsam bewegst du dich Zentimeter um Zentimeter zur Tür, die Decke des Raums immer im Blick. Dabei übersiehst du leider einen kleinen Kieselstein, auf den du trittst. Das entstehende Schleifgeräusch ist alles andere als laut. Trotzdem sackt dir für einen Moment das Herz zusammen, als du Bewegung im Spinnennetz erkennst. Du wurdest entdeckt!

Wenn du zum Tor rennen willst, gehe zu [\ref{sprintToTheMainDoor}]!
Wenn du dich dem Kampf stellen willst, gehe zu [\ref{comeAtMeSpiderling}].

\block{sprintToTheMainDoor}{Ich renne zum Tor}

Du handelst sofort und rennst zum Tor. Hinter dir hörst du ein dumpfes Geräusch. Mit Sicherheit ist die Spinne gerade auf dem Boden aufgekommen. Eilig! Du ziehst mit aller Kraft an der schweren Holztür, das Geräusch der Spinnenbeine auf dem Steinboden hinter dir.

Wirf eine Athletikprobe mit DC 17. Wenn du Erfolg hast, gehe zu [\ref{escapeTroughTheMainDoor}]. Schaffst du es nicht, gehe zu [\ref{noEscapeFromSpider}].

\block{noEscapeFromSpider}{Ich schaffe es nicht rechtzeitig}

Du mobilisierst alle Kraftreserven, die dir noch bleiben und die Tür öffnet sich knarzend Milimeter um Milimeter. Leider ist das zu langsam, die Spinne fällt dir in den Rücken und greift dich an!

Gehe zu [\ref{theSpiderAgain}].

\block{theSpiderAgain}{Nur über meine Leiche}

Noch bevor du dich umdrehen kannst hörst du die Greifarme des Ungetüms hinter dir zuschnappen.
Unabhängig von der Iniative darf die Spinne zuerst angreifen und hat beim ersten Angriff Vorteil, da sie dir in den Rücken fällt.
\monsterSpider{letzte Lebenspunkte +2 (max. 26)}{poisonedAndDying}

Wenn du die Spinne getötet hast, gehe zu [\ref{ikilledthedamnspider}].
\\Wenn die Spinne dich tötet, gehe zu [\ref{theSpiderKilledMe}].

\block{escapeTroughTheMainDoor}{Das muss reichen}

Du mobilisierst alle Kraftreserven, die dir noch bleiben und die Tür öffnet sich Zentimeter um Zentimeter. Hinter dir hörst du das Kreischen der schrecklichen Kreatur, die auf dich zurast! Gerade noch rechtzeitig ist der Spalt der Tür groß genug um hindurchzuschlüpfen. Du drückst dich durch die Öffnung und stehst vor dem Tor, als die Spinne mit voller Wucht gegen das Holz rennt und das Tor wieder schließt. Gedämpft hörst du das Kreischen des Monsters, das mit seinen haarigen Beinen gegen das Holz kratzt. Dein Herz hämmert, doch du hast es geschafft - du bist in Sicherheit.

Gehe zu [\ref{longRoadHome}].

\block{comeAtMeSpiderling}{Nur über deine Leiche}

Du erkennst sofort, dass es keinen Sinn macht zur Tür zu rennen und dem Biest den Rücken zuzuwenden. Nein, nur einer von euch wird diesen Raum lebend verlassen!

Einen Augenblick später trifft der schwere Leib des Untiers auf dem harten Boden auf. Mit erhobenen Greifarmen stürmt es kreischend auf dich zu!

\monsterSpider{letzte Lebenspunkte +2 (max. 26)}{poisonedAndDying}

Wenn du die Spinne getötet hast, gehe zu [\ref{ikilledthedamnspider}].
\\Wenn die Spinne dich tötet, gehe zu [\ref{theSpiderKilledMe}].

\block{poisonedAndDying}{Sechs Arme zu viel}

Du lieferst dir den Kampf deines Lebens mit der Spinne. Du weichst Biss um Biss der Kreatur aus, doch langsam gewinnt die Spinne die Oberhand. Du hast nur einen einzigen unachtsamen Augenblick. Sofort bohren sich die dicken Zangen der Spinne in deinen Körper. Aus dieser Situation gibt es kein Entkommen.

Langsam sackst du auf die Knie. Das Untier tänzelt vor dir hin und her, wohl wissend, dass das Gift wirkt. Als du das Gefühl in deinen Fingern verlierst, siehst du wie sich der riesige Spinnenleib über dich schiebt. Du wirst hochgehoben und in eine klebrige weiße Flüssigkeit gewickelt. Du wirst konserviert, zweifelsohne als spätere Mahlzeit.

Keine überraschende Wendung. Kein Happy End. Immerhin nimmt dir das Gift der Spinne langsam die Sinne. Der Schmerz lässt nach und eine tiefe Müdigkeit überkommt dich.

\textbf{Ende.}

\block{ikilledthedamnspider}{Spinnenfreie Zone}

Du lieferst dir den Kampf deines Lebens mit der Spinne. Du weichst Biss um Biss der Kreatur aus, doch langsam gewinnt die Spinne die Oberhand. Plötzlich erkennst du einen unachtsamen Moment, den du sofort ausnutzt. Du triffst die Spinne schwer und sie taumelt wie benommen zurück. Die haarigen Beine zittern, als sie versucht vor dir zurückzuweichen, doch ihre Kraft reicht nicht mehr aus um die Decke emporzusteigen. Mit einem letzten Kreischen sackt der schwere Leib auf den Boden und die Beine werden angezogen. Dann Ruhe und Regungslosigkeit. Du hast die Spinne getötet.

Du wartest nicht lang, um zu sehen, was der Turm noch für dich bereit hält. Nein, dieses Abenteuer muss ein Ende finden. Mit letzter Kraft ziehst du die schwere Tür auf und rennst in den Wald.

Gehe zu [\ref{longRoadHome}].

\block{longRoadHome}{Der Weg nach Greifenheim}

Du hast bei diesem Abenteuer vielleicht nicht viel gewonnen, aber das wichtigste ist, dass du auch nicht viel verloren hast. Du bist am Leben, nur das zählt. Nachdem du ein paar Meter gelaufen bist, siehst du dich noch einmal um. Noch immer ragt der Turm bedrohlich in den Himmel, doch du wirst ihn nicht mehr betreten.

Vor dir liegt ein beschwerlicher Fußmarsch zurück nach Greifenheim. Doch angetrieben vom Gedanken an ein gemütliches Bett und eine warme Mahlzeit machst du dich auf den Weg. Du hast dir Ruhe und Erholung verdient.

\textbf{Ende.}

\block{inspectTheDungeon}{Ich bleibe im Kerker}
%Immer von meiner Zelle aus
Du willst deine Chance nutzen und dich im Kerker noch etwas genauer umsehen. Gegenüber deiner Zelle befindet sich die Treppe nach oben, auf der linken Seite geht es tiefer in den dunklen Raum. Du folgst dem Gang und kommst bald zu einer weiteren leeren Zelle. Ein flüchtiger Blick offenbart nichts von Interesse und du so biegst du nach links ab. Am Ende des Gangs befinden sich zwei weitere Zellen, von denen die linke offen steht. Die Tür der rechten Zelle ist verschlossen.

Im Dunkel des kleinen Raums kannst du eine kauernde Gestalt erkennen, kaum größer als ein Kind.
Als du näher an die Gitterstäbe gehst, kannst du ein leises Knurren hören.

Wenn du mit dem Wesen reden willst, gehe zu [\ref{talkToTheCreature}].
Wenn du weitergehen willst, gehe zu [\ref{nothingLeftInTheDungeon}].
Wenn du versuchen willst die Zellentür zu öffnen, gehe zu [\ref{openGoblinsDoor}].

\block{talkToTheCreature}{Ich versuche zu reden}

Du sprichst die Kreatur an, doch bekommst keine Reaktion außer einem leisen Wimmern und dem gelegentlichen Knurren. Vielleicht will sie nicht antworten, vielleicht kann sie nicht antworten, das kannst du nicht sagen.

Wenn du \getItem{itemSimplePotato} hast, gehe zu [\ref{givePotato}].
Wenn du weitergehen willst, gehe zu [\ref{nothingLeftInTheDungeon}].
Wenn du versuchen willst die Zellentür zu öffnen, gehe zu [\ref{openGoblinsDoor}].

\block{givePotato}{Ich kann dir das geben}

Als du das Knurren erneut hörst, erinnerst du dich an die Kartoffel. Schnell prüfst du deine Tasche und tatsächlich, du hast das gute Stück noch. Zögerlich steckst du die Knolle durch die Gitterstäbe und legst sie auf den Zellenboden.

Die Kreatur hört auf zu wimmern und beäugt die Kartoffel aus sicherer Entfernung. Schlussendlich ist der Hunger stärker als die Vorsicht und eine grüne Hand greift aus dem Schatten nach der Knolle. Vor dir befindet sich ein kleiner, abgemagerter Goblin. Mit seinen spitzen Zähnen nagt er auf der Kartoffel herum, bis nichts mehr von ihr übrig ist. Zögerlich sagt er gebrochen: ``Du Hilfe?''

Wenn du deine Hilfe anbieten willst, gehe zu [\ref{helpingTheGoblin}].
Wenn du deine Pflicht als getan siehst und weiterziehen willst, gehe zu [\ref{notHelpingTheGoblin}].

\block{notHelpingTheGoblin}{Ich habe keine Zeit}

Vermutlich will der kleine Kerl deine Hilfe, aber du ahnst schon, dass es einiges an Zeit kosten wird bis du verstehst worum es geht. Und du hast leider ohnehin keine Möglichkeit ihn aus der Zelle zu holen, was er zweifelsohne möchte. Du ersparst euch beiden viel Leid, schüttelst den Kopf und gehst weiter.

Gehe zu [\ref{nothingLeftInTheDungeon}].

\block{nothingLeftInTheDungeon}{Ich kann nichts tun}

Du bist leider nicht in der Position deine Zeit zu verschwenden wie es dir beliebt. Du musst weiter. Du siehst dich noch einmal gründlicher um, doch kannst nichts von Bedeutung finden. Es ist nicht schwer zurück zur Treppe zu finden und kurze Zeit später bist du ein Stockwerk höher. Du befindest dich einem kleinen Raum. Hinter dir befindet sich die Treppe hinab in den Kerker, vor dir eine weitere Treppe, die höher führt. Auf der linken Seite siehst du eine Holztür.

Wenn du der Treppe folgen willst, gehe zu [\ref{stairsToTheUpperLevels}].
\\Wenn du durch die Tür gehen willst, gehe zu [\ref{doorToLobby}].

\block{openGoblinsDoor}{Ich würde gern helfen}

Du würdest die verschlossene Zellentür gern öffnen, doch du hast weder einen Schlüssel noch die passende Ausrüstung um das zu tun. Nein, du kannst hier leider nicht helfen. Du musst weiter, zur Treppe. Kurze Zeit später bist du ein Stockwerk höher. Du befindest dich einem kleinen Raum. Hinter dir befindet sich die Treppe hinab in den Kerker, vor dir eine weitere Treppe, die höher führt. Auf der linken Seite siehst du eine Holztür.

Wenn du der Treppe folgen willst, gehe zu [\ref{stairsToTheUpperLevels}].
\\Wenn du durch die Tür gehen willst, gehe zu [\ref{doorToLobby}].
