\chapter{Der schwarze Turm}

\block{arrivalAtTower}{Die Ankunft}

In der Ferne drückt sich ein mächtiger Turm aus schwarzem Stein wie ein Dorn in den Himmel. Allein der Anblick sorgt für ein mulmiges Gefühl in der Bauchgegend, was dadurch verstärkt wird, dass dieser Turm offensichtlich das Ziel der Reise ist.

Je näher ihr ihm kommt, desto mehr beunruhigende Details kommen zum Vorschein. Die Bäume in der Nähe sind kahl und haben eine blässliche, tote Farbe angenommen. Das zuvor hörbare Zwitschern der Vögel ist verstummt und ein leichter Wind trägt kalte Luft zu dir, sodass du kurz ungewollt erschauderst. Für einen Moment sieht es so aus, als sei der Torbogen um die große hölzerne Tür aus Schädeln konstruiert. Als du erneut hinsiehst, erkennst du, dass es sich um kleine Skulpturen handelt, die schreckliche entstellte Fratzen abbilden. Es ist schwer zu sagen, wie hoch der Turm wirklich ist, denn seine Spitze wird von einem Trichter aus dunklen Wolken verdeckt, die sich wie in einer Spirale langsam drehen.

Ohne anzuhalten fährt der Wagen auf die Tür zu, die wie von Geisterhand und mit hörbarem Ächzen nach links und rechts aufschwingt und Einlass gewährt. Im Inneren befindet sich ein kreisrunder Raum, der ungefähr 10 Meter im Durchmesser misst. Gegenüber des Eingangstors befindet sich eine weitere Holztür. Während du die Decke des Raums betrachtest, an der ein Kronleuchter aus schwarzem Metall hängt, rüttelt der Wagen kurz. Deine Entführer sind abgestiegen und nur Sekunden später wird dir ein Leinensack über den Kopf gezogen. Du hörst eine fremde Stimme krächzen: ``Sehr gut... in den Kerker...''. Kurz darauf wirst du hochgehoben.
\\Wenn du dich wehren willst, gehe zu [\ref{resistCell}].
\\Wenn du abwarten willst, gehe zu [\ref{arriveInCell}].

\block{resistCell}{Nicht ohne Widerstand}

Ein Kerker ist selten etwas Gutes. Du fängst an dich heftig zu wehren und schlägst mit den gefesselten Gliedmaßen um dich. Leider nur mit mäßigem Erfolg, doch immerhin bewegst du dich so ungünstig, dass dein Entführer nicht anders kann als dich fallen zu lassen. Du hörst wie er wütend sagt: ``Nun aber genug mit den Spielchen!''. Kurz darauf trifft dich etwas hartes an der Schläfe und du verlierst das Bewusstsein.\\
Du verlierst 4 Lebenspunkte.
\\Gehe zu [\ref{wakeUpInCell}].

\block{arriveInCell}{Ohne Widerstand}

Du spürst, dass es kühler wird und ihr abwärts geht, vermutlich eine Treppe hinab. Modriger Geruch steigt dir in die Nase und du hörst das Quietschen alter Metallgitter. Wenige Sekunden später wirst du auf den Boden geworfen und der Leinensack wird entfernt.
\\Gehe zu [\ref{welcomeToTheCell}].

\block{arrivalAtTowerSingle}{Die verspätete Ankunft}

In der Ferne drückt sich ein mächtiger Turm aus schwarzem Stein wie ein Dorn in den Himmel. Allein der Anblick sorgt für ein mulmiges Gefühl in der Bauchgegend, was dadurch verstärkt wird, dass dieser Turm offensichtlich das Ziel der Reise ist. Dein verbleibender Entführer hat sich ab und an zu dir umgedreht und dich hasserfüllt angesehen, aber kein weiteres Wort verloren.

Je näher ihr dem Turm kommt, desto mehr beunruhigende Details kommen zum Vorschein. Die Bäume in der Nähe sind kahl und haben eine blässliche, tote Farbe angenommen. Das zuvor hörbare Zwitschern der Vögel ist verstummt und ein leichter Wind trägt kalte Luft zu dir, sodass du kurz ungewollt erschauderst. Für einen Moment sieht es so aus, als sei der Torbogen um die große hölzerne Tür aus Schädeln konstruiert. Als du erneut hinsiehst, erkennst du, dass es sich um kleine Skulpturen handelt, die schreckliche entstellte Fratzen abbilden. Es ist schwer zu sagen, wie hoch der Turm wirklich ist, denn seine Spitze wird von einem Trichter aus dunklen Wolken verdeckt, die sich wie in einer Spirale langsam drehen.

Ohne anzuhalten fährt der Wagen auf die Tür zu, die wie von Geisterhand und mit hörbarem Ächzen nach links und rechts aufschwingt und Einlass gewährt. Im Inneren befindet sich ein kreisrunder Raum, der ungefähr 10 Meter im Durchmesser misst. Gegenüber des Eingangstors befindet sich eine weitere Holztür. Während du die Decke des Raums betrachtest, an der ein Kronleuchter aus schwarzem Metall hängt, rüttelt der Wagen kurz. Dein Entführer ist abgestiegen und nur Sekunden später wird dir ein Leinensack über den Kopf gezogen. Eine fremde Stimme fragt krächzend: ``Ihr... seid spät. Wo ist Wladan?''. Du hörst ein leises Gemurmel als Antwort und die Reaktion des Fremden: ``Ein Schande... in den Kerker...''. Kurz darauf wirst du hochgehoben.
\\Wenn du dich wehren willst, gehe zu [\ref{resistCell}].
\\Wenn du abwarten willst, gehe zu [\ref{arriveInCell}].

\block{infiltrateTower}{Der Einbruch}

Nach knapp zwei weiteren Stunden Fußmarsch und mit dem Einbruch der Dämmerung drückt sich in der Ferne ein mächtiger Turm aus schwarzem Stein wie ein Dorn in den Himmel. Allein der Anblick sorgt für ein mulmiges Gefühl in der Bauchgegend, was dadurch verstärkt wird, dass dieser Turm offensichtlich das Ziel deiner Reise ist.

Je näher du ihm kommst, desto mehr beunruhigende Details kommen zum Vorschein. Die Bäume in der Nähe sind kahl und haben eine blässliche, tote Farbe angenommen. Das zuvor hörbare Zwitschern der Vögel ist verstummt und ein leichter Wind trägt kalte, modrige Luft zu dir, sodass du kurz ungewollt erschauderst. Du stehst nun vor der rechten Seite des Turms. Zu deiner linken kannst du das Eingangstor sehen. Für einen Moment sieht es so aus, als sei der Torbogen um die große hölzerne Tür aus Schädeln konstruiert. Als du erneut hinsiehst, erkennst du, dass es sich um kleine Skulpturen handelt, die schreckliche entstellte Fratzen abbilden.

Es ist schwer zu sagen, wie hoch der Turm wirklich ist, denn seine Spitze wird von einem Trichter aus dunklen Wolken verdeckt, die sich wie in einer Spirale langsam drehen. Du kannst in ungefähr 10 Metern Höhe ein Fenster erkennen. Am Fuß des Turms befinden sich, außer vor dem Tor, dichte Dornenbüsche. Es ist in der Dunkelheit schwer zu erkennen, aber die Steine des Turms sehen so aus, als könnte man die Wand mit etwas Erfahrung hinaufklettern.

Wenn du zum Tor gehen willst, gehe zu [\ref{invadeTheDoor}].
\\Wenn du es am Fenster probieren willst, gehe zu [\ref{upTheWindow}].
\\Wenn du deine Reise abbrechen und aufgeben willst, gehe zu [\ref{letsquitnow}].

\block{invadeTheDoor}{Ich nehme das Tor}

Du schätzt die Wand bei diesen Lichtverhältnissen als zu großes Risiko ein. Nein, man sieht der Gefahr offen ins Gesicht. Du läufst zum Tor. Die beiden Flügeltüren sind aus schwerem, dunklen Holz und von Kratzern übersäät. Auf jeder Seite hängt ein schwerer Ring im Mund einer metallenen Dämonenfratze.

Wenn du versuchen willst die Tür aufzuziehen, gehe zu [\ref{strengthTestDoor}].
\\Wenn du den Metallring schlagen willst, gehe zu [\ref{knockknockHereIAm}].
\\Wenn du es doch am Fenster probieren willst, gehe zu [\ref{upTheWindow}].
\\Wenn du deine Reise abbrechen und aufgeben willst, gehe zu [\ref{letsquitnow}].

\block{strengthTestDoor}{Ich öffne das Tor}

Du beschließt, dass Anklopfen keine Option ist. Nein, dieses Tor muss mit Gewalt geöffnet werden! Du drückst deine Schulter gegen das Holz und drückst mit aller Kraft!

Lege eine Athletikprobe mit DC 20 ab. Wenn du Erfolg hast, gehe zu [\ref{forceOpenDoor}].
Wenn nicht, gehe zu [\ref{doorStaysShut}].

\block{doorStaysShut}{Das Tor gewinnt}

Du mühst dich einige Minuten vergeblich ab, bevor du einsiehst, dass du hier nichts ausrichten kannst. Die Tür hat sich keinen Milimeter bewegt.

Wenn du den Metallring schlagen willst, gehe zu [\ref{knockknockHereIAm}].
\\Wenn du es doch am Fenster probieren willst, gehe zu [\ref{upTheWindow}].
\\Wenn du deine Reise abbrechen und aufgeben willst, gehe zu [\ref{letsquitnow}].

\block{forceOpenDoor}{Ich gewinne}

Du mühst dich einige Minuten ab und Milimeter für Milimeter bewegt sich das Holz ins Innere des Turms. Bald hast du einen ausreichend großen Spalt geöffnet, um hindurch zu schlüpfen.

Im Inneren befindet sich ein kreisrunder Raum, der ungefähr 10 Meter im Durchmesser misst. An den Wänden sind Fackeln in die Wand eingelassen und erhellen die Szenerie dürftig. Gegenüber des Eingangstors befindet sich eine weitere Holztür. An der Decke des Raums befindet sich ein Kronleuchter aus schwarzem Metall, an dem irgendetwas Größeres aufgehangen wurde.

Wenn du zur anderen Tür gehen willst, gehe zu [\ref{walkTowardsTheNextDoor}].
\\Wenn du zur anderen Tür schleichen willst, gehe zu [\ref{stealthilyGifford}].
\\Wenn du versuchen willst eine der Fackeln zu nehmen, gehe zu [\ref{grabLight}].

\block{grabLight}{Ich greife nach dem Licht}

Du betrittst den Raum und streckst dich, um eine der Fackeln zu erreichen. Plötzlich hörst du über dir ein tiefes Brüllen und das Rasseln einer Metallkette.

Gehe zu [\ref{someMonsterAttacks}].

\block{knockknockHereIAm}{Ich klopfe an}

Du nimmst den Eisenring und hämmerst ihn dreimal fest gegen das Tor. Der Hall deiner Schläge scheint durch das ganze Gemäuer zu hallen. Einige Sekunden lang geschieht nichts. Als du den Ring anfassen willst, um ihn erneut zu schlagen, schwingen die beiden Türen wie von Geisterhand auf. Im Inneren befindet sich ein kreisrunder Raum, der ungefähr 10 Meter im Durchmesser misst. An den Wänden sind Fackeln in die Wand eingelassen und erhellen die Szenerie dürftig. Gegenüber des Eingangstors befindet sich eine weitere Holztür. An der Decke des Raums befindet sich ein Kronleuchter aus schwarzem Metall, an dem irgendetwas Größeres aufgehangen wurde.

``Welch unverhoffter Gast zu so später Stunde.'' hörst du eine Stimme auf der anderen Seite des Raums krächzen. Die Shilouette eines Mannes in einer langen, schwarzen Robe tritt aus dem Schatten. Bevor du antworten kannst, hebt der Mann seine Hand, in der er einen kleinen Stab hält. Als dieser beginnt bläulich zu leuchten, spürst wie dich eine plötzliche Erschöpfung überkommt.

Lege einen Rettungswurf auf Konstitution mit DC 18 ab. Wenn du erfolgreich bist, gehe zu [\ref{noSleepTillBrooklyn}].
\\Wenn du es nicht schaffst, gehe zu [\ref{mamaIwillSleep}].

\block{walkTowardsTheNextDoor}{Ich gehe zur Tür}

Du betrittst den Raum und läufst zur gegenüber liegenden Tür. Plötzlich hörst du über dir ein tiefes Brüllen und das Rasseln einer Metallkette.

Gehe zu [\ref{someMonsterAttacks}].

\block{mamaIwillSleep}{Ich bin sehr erschöpft}

Du versuchst gegen die Erschöpfung anzukämpfen, doch der Zauber ist zu stark. Bevor du einen weiteren Schritt machen kannst, spürst du wie deine Augen schwer werden und du zu Boden sinkst.

Wenn du (2), (3) und (4) hast, verlierst du diese Gegenstände.

Als du wieder zu dir kommst, steigt dir ein modriger Geruch in die Nase. Es ist kalt und feucht und du bist wieder gefesselt. Du merkst sofort, dass du dich nicht mehr in der Eingangshalle befindest, denn der Boden ist weicher und es riecht nach Erde.

Gehe zu [\ref{welcomeToTheCell}].

\block{noSleepTillBrooklyn}{Keine Zeit für Schlaf}

Du versuchst gegen die Erschöpfung anzukämpfen. Für einen Moment glaubst du, dass du es nicht schaffen wirst und deine Augen sinken schwer nach unten.

%TODO Fight gegen Monster, das am Kronleuchter hing

\block{upTheWindow}{Ich nehme das Fenster}

Es wird sicher kein Kinderspiel diese Wand zu erklimmen, aber es sieht auch nicht unmöglich aus. Allerdings musst du dir vorher überlegen, wie du durch die Dornenbüsche kommst.

Wenn du hindurchkriechen willst, gehe zu [\ref{dornyBushes}].
\\Wenn du (3) hast, kannst du zu [\ref{dornyBushesWithSword}] gehen.

\block{letsquitnow}{Ich habe es mir anders überlegt}

Jetzt, wo du vor dem Turm stehst, kommt dir das ganze doch nicht mehr sonderlich schlau vor. Nein, eigentlich überhaupt nicht. Warum solltest du jetzt, ohne richtige Ausrüstung und Vorbereitung, ein solches Risiko eingehen?

Das macht keinen Sinn und du hast im Laufe der Jahre gelernt Situationen zu vermeiden, die keinen Sinn machen. Nein, es wird Zeit nach Hause zu gehen. Du kehrst zurück in den Wald und machst dich auf den mühevollen Heimweg. Eine warme Suppe und ein Bett warten auf dich.

\textbf{Ende.}

\block{wakeUpInCell}{Unsanftes Erwachen}

Als du wieder zu dir kommst, steigt dir ein modriger Geruch in die Nase. Es ist kalt und feucht und du bist noch immer gefesselt. Offensichtlich liegst du nicht mehr auf dem Wagen, denn der Boden ist weicher und riecht nach Erde.
\\Gehe zu [\ref{welcomeToTheCell}].
