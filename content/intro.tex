
\chapter{Vorwort}

\paragraph{Regeln}

``Osmonias Turm'' ist ein Solo-Abenteuer für einen Charakter auf Stufe 4 nach dem D\&D 5E-Regelwerk. Proben und Kämpfe erfolgen wie in einem normalen D\&D-Abenteuer.

Im Verlauf des Abenteuers kann es zu Kämpfen kommen. Die Werte der Gegner werden dann angegeben. Zu Beginn jedes Kampfes wird die Iniative ermittelt, es sei denn die Beschreibung legt etwas anderes fest. Sofern nicht anders beschrieben greifen die Gegner jeden Feind in Sichtweite an und kämpfen bis zu ihrem Tod.

Hin und wieder können Gegenstände gefunden werden. Jedem Gegenstand ist eine Nummer zugeordnet (beispielsweise: ``Du findest einen Schlüssel (21)''). Schreib dir den Namen des Gegenstands mit der Nummer auf. Einige Gegenstände können bestimmte Entscheidungen ermöglichen oder Ereignisse auslösen. An dieser Stelle wird auf die Nummer verwiesen, wenn du den Gegenstand mit der zugehörigen Nummer besitzt, kann diese Entscheidung gewählt werden. Andere, wie Waffen oder Tränke, können auch im Kampf verwendet werden. Diese Gegenstände besitzen die im D\&D-Spielerhandbuch angegebenen Werte und Eigenschaften.

\paragraph{Adventure Corp Staffel 4}

Dieses Abenteuer ist in der gleichen Welt angesiedelt, in der auch die vierte Staffel von ``Adventure Corp'' stattfindet. Zeitlich bewegen wir uns einige Wochen bevor die Heldengruppe das kleine Dorf Wrakenberg erreichen wird.

\chapter{Einleitung}

\paragraph{Die zerissene Welt}

Vor langer Zeit, so erzählen sich die Alten, nannte man unsere Welt "das grüne Paradies". Die Bewohner dieser Welt waren Meister der Magie, die mit einem bloßen Wort die mächtige Götter aus Stahl und Eisen befehligen konnten. Mit Hilfe dieser Maschinengötter wandelten sie das Land, sie machten Wälder aus Wüsten und bauten Städte in Sümpfen, deren Dächer an den Wolken kratzten. Wenn sie neues Land brauchten, hoben sie Inseln aus dem Meer, wenn sie über ein Gebirge reisen musste, schlugen sie Tunnel direkt durch die Berge.

Wie so oft endet auch diese Geschichte nicht glücklich. Denn obwohl sie alles hatten, um in Frieden zu leben, lagen die alten Völker ständig im Streit miteinander. Sie schufen Maschinengötter, die Zerstörung und Tod brachten. Machte der eine ein Moor bewohnbar, ließ es der nächste die Häuser wieder im Boden versacken. Niemand weiß, wer es war, doch am Ende dieser Zeit schufen sie einen Maschinengott von so schrecklicher Macht, dass sein Schrei dem Nachthimmel die schreckliche Narbe zufügte, die wir noch heute an jedem Abend über uns sehen. Und mit diesem Tag endete das Zeitalter der Maschinenherren, deren Wissen heute beinahe verloren ist. Angeblich gibt es noch eine handvoll armer Seelen, die das alte Wissen mit sich tragen.

Doch das sind nur Geschichten. Die alten Häuser sind zu Staub zerfallen, die großen Tempel liegen zerstört. Nur ein paar der Maschinengötter haben die Zeit überdauert und streifen noch heute durch die Lande, auf der Suche nach ihrem Zweck.

\paragraph{Eine anstrengende Nacht}

Du hast gerade ein kleines Abenteuer hinter dir und bist in der Stadt Greifenheim. Wer als Abenteurer etwas auf sich hält, gibt natürlich ein paar Münzen in einer Taverne aus. Eine beinahe schon berühmte Sehenswürdigkeit ist die Taverne ``Zum Nimmersatt'', in der seit Generationen eine alte Braumaschine verwendet wird. Angeblich ist die Maschine noch aus der Zeit der Maschinenherren und stand noch nie still. Doch das ist heute Abend nur eine Nebensache, denn einer der Gäste ist in Feierlaune und das Bier fließt in Strömen.

Der Gönner, Sir Lichtenstein, feiert seine Verlobung und bevorstehende Hochzeit. Du unterhälst dich eine Weile mit ihm, doch irgendwann packt dich die Müdigkeit und du wankst leicht angeheitert zu deinem Gasthaus in der Nähe. Leider wirst du dort nicht ankommen...
